
\documentclass[11pt,fleqn]{article} 
\usepackage[margin=0.8in, head=0.8in]{geometry} 
\usepackage{amsmath, amssymb, amsthm}
\usepackage{fancyhdr} 
\usepackage{palatino, url, multicol}
\usepackage{graphicx, pgfplots} 
\usepackage[all]{xy}
\usepackage{polynom} 
%\usepackage{pdfsync} %% I don't know why this messes up tabular column widths
\usepackage{enumerate}
\usepackage{framed}
\usepackage{setspace}
\usepackage{array,tikz}

\pgfplotsset{compat=1.6}

\pgfplotsset{soldot/.style={color=black,only marks,mark=*}} \pgfplotsset{holdot/.style={color=black,fill=white,only marks,mark=*}}
\pgfplotsset{my style/.append style={axis x line=middle, axis y line=
middle, xlabel={$x$}, ylabel={$y$} }}

%axis equal 
\pagestyle{fancy} 
\lfoot{}
\rfoot{Review Midterm I}

\begin{document}
\renewcommand{\headrulewidth}{0pt}
\newcommand{\blank}[1]{\rule{#1}{0.75pt}}
\newcommand{\bc}{\begin{center}}
\newcommand{\ec}{\end{center}}
\renewcommand{\d}{\displaystyle}

\vspace*{-0.7in}

%%%%%%%%%intro page
\begin{center}
  \large
  \sc{Review for Midterm I}\\
\end{center}
\noindent\textbf{Logistics}\\

You will have 90 minutes to take the midterm. You can see what your midterm will look like by going to the public webpage and looking at Midterm I from previous years. Note that your midterm will not be longer then those of previous years\\

\url{https://uaf-math251.github.io/exams.html}\\

You may bring a single 3 x 5 notecard, hand-written, front and back. You may not use a calculator. (There are not problems that require the use of one.)\\

%\textbf{For Dr. McIntyre's Section:} Our midterm will start promptly at 9:45 am and will end at 11:15 am.  Some students will finish early. To ensure that later students don't get an unfair advantage, \emph{all students are required to stay in class until 11:15}. If you know you are someone who finishes early, you should bring something to work on \emph{that does not require typing on your phone, laptop or tablet.}
%
%\textbf{For Dr. Faudree's Section:} Our midterm will start promptly at 11:30am.  Once you have finished your midterm, you may leave quietly.\\

\noindent\textbf{Taking a Math Test}

\begin{itemize}
\item The only way to earn partial credit for incorrect answers is by writing explanatory work.
\item Explanatory work needs to be written in a way that a reader can follow it.
\item Write things that are incorrect will generally result in a loss of points. So if something shouldn't be graded, cross it out.
\item Approach each problem with the underlying assumption that you have the skills and information to complete it.
\end{itemize}


\noindent\textbf{Topics}\\

\noindent \fbox{Section 2.1}\\
Secant lines and tangent lines. Average velocity and instantaneous velocity. Average rate of change and instantaneous rate of change.\\

\noindent Examples: \\
(a) Sketch the graph $y=x^3+1.$ Find the secant line between the points on the graph where $x=1$ and $x=3.$ Sketch the secant line on the graph. Find an equation of the tangent line to the graph at $x=1$ and sketch it on the graph. (NOTE: We get to answer the second part of this question using our knowledge of the derivative!) \\
(b) A salmon is swimming up stream and the position of the salmon is given by the function $d(t)=\frac{1}{2}t^2-t$ where $t$ is measured in hours and $d$ is measured in feet. Find the average velocity of the salmon over the interval from $t=2$ to $t=4.$ Find the instantaneous velocity of the salmon at $t=3.$ Include units. Explain in simple terms, in the context of the problem, what these calculations mean.\\

\noindent \fbox{Section 2.2}\\
Be able to evaluate one-sided and two-sided limits from a graph. Vertical asymptotes and limits.\\

\noindent Examples: \\
(a) Sketch a graph with \emph{all} of the following properties: \\
\begin{itemize}
\item $f(x)$ is defined for all real numbers. (ie its domain is $(-\infty,\infty).$
\item $\lim_{x \to 1^-} f(x) = 0,$ $\lim_{x \to 1^+} f(x) =4,$ $f(1) =4$
\item $\lim_{x \to 3} f(x) =4,$ $f(4)=-1.$
\item $\lim_{x \to -1^-} f(x) =\infty$
\end{itemize}
(b) Determine where the graph of $f(x)=\frac{x}{(x+2)^2}$ has a vertical asymptote and use limits to justify your answer is correct. \\

\noindent \fbox{Section 2.3}\\
Evaluating limits algebraically. You need to remember the strategies (factor, common denominator, and rationalizing). But, don't forget to always try simple first: plug in. \\

\noindent Examples: Evaluate $\d \lim_{x \to 9} \frac{3-\sqrt{x}}{9-x}$ and $\d \lim_{x \to 1/2^+} \frac{4x^2-18x}{2x-1}$\\

\noindent \fbox{Section 2.4}\\
Continuity. From a graph, determine where a graph is or is not continuous. From an algebraic description of a function, determine where a function is or is not continuous. Be able to explain why a function is not continuous at a point. Use the Intermediate Value Theorem.\\

\noindent Examples: 
(a) Look at your graph from Section 2.2. Where does it fail to be continuous and why? \\
(b) Determine where the functions $f(x)= \frac{3-\sqrt{x}}{9-x}$ and $g(x)=  \frac{4x^2-18x}{2x-1}$ fail to be continuous and show that your answer is correct algebraically.\\
(c) If $f(x)$ is piecewise determined so that $f(x) =Ce^x$ for $x\leq 0$ and $f(x)=8 \cos(x)$ for $x \geq 0,$ is $f(x)$ continuous? Is there a way to chose $C$ such that $f(x)$ is continuous?
(d) Use the Intermediate Value Theorem to show that the equation $2e^x-3x-5=0$ has a solution.\\

\noindent \fbox{Section 3.1}\\
The relationship between secant lines and the derivative.\\

\noindent Example: Explain what the expression $f'(a) = \lim_{x \to a} \frac{f(x)-f(a)}{x-a}$ means in terms of secant lines, tangent lines and the derivative. Draw a picture to illustrate you idea.\\

\noindent \fbox{Section 3.2}\\
The derivative as a function. The formal definition of the derivative. The relationship between the graph of $f(x)$ and the graph of $f'(x).$\\

\noindent Example: Sketch the derivative of your graph from the Section 2.2 example. Use the definition of the derivative to find $f'(x)$ for $f(x)=1/x^2.$\\

\noindent \fbox{Section 3.3}\\
Derivative rules: power, constant, sum/difference, product, quotient, and the derivatives of the sine and cosine functions.\\

\noindent Example: Find the derivative of $y=2x^{0.05}-\frac{x}{10}+\frac{\cos(x)}{5}$, $s=t\cos(t),$ and $f(x)=\frac{x^3}{1-x}+ \sin(x)$\\

\noindent \fbox{Section 3.4}\\
The derivative as a rate of change. Interpretations of the derivative. Velocity and acceleration.\\

\noindent Example: Assume the distance traveled by a snow machine on a straight trail is given by $s(t)$ where $t$ is in hours starting at 12 noon and $s$ is in miles. Interpret $s'(4)=10.$ Interpret $s(4)-s(0).$ Interpret $(s(4)-s(1))/(4-1).$ Using the fact that that $s''(4)=-1.2,$ estimate $s'(4.5).$

\end{document}

