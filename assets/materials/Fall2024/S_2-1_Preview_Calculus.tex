\documentclass[11pt,fleqn]{article} 
\usepackage[margin=0.8in, head=0.8in]{geometry} 
\usepackage{amsmath, amssymb, amsthm}
\usepackage{fancyhdr} 
\usepackage{palatino, url, multicol}
\usepackage{graphicx} 
\usepackage[all]{xy}
\usepackage{polynom} 
\usepackage{pdfsync}
\usepackage{enumerate}
\usepackage{framed}
\usepackage{setspace}
\usepackage{array,tikz}
\pagestyle{fancy} 
\lfoot{UAF Calculus 1}
\rfoot{Sect 2.1 }

\begin{document}
\renewcommand{\headrulewidth}{0pt}
\newcommand{\blank}[1]{\rule{#1}{0.75pt}}
\renewcommand{\d}{\displaystyle}
\vspace*{-0.7in}
\begin{center}
  \large \sc{Section 2.1: Preview of Calculuis}
\end{center}
goals: To understand
\begin{itemize}
\item the difference between a secant line and a tangent line.
\item how to use secant lines to estimate the slope of a tangent line.
\item how to use average velocity to estimate instantaneous velocity.
\item why our present tools force us to \emph{estimate} slope or instantaneous velocity and not calculate it explicitly.
\end{itemize}


\begin{enumerate}
\item REVIEW: Write the equation of the line through the points $P(-3,1)$ and $Q(2,4)$.
\vspace{1.5in}
\item The point $P(2,3)$ lies on the graph of $f(x)=x+\frac{2}{x}.$ For each value of $x$ in the table below, find the slope of the secant line between $P(2,3)$ and $Q(x,f(x)),$ \emph{if possible}.\\
		{\LARGE{\begin{center}
		\begin{tabular}{l | l | c}
		\multicolumn{2}{c}{point $Q$}& slope of secant line $PQ$\\
		$x$-value&\quad$y$-value \quad \quad& $PQ$\\
		\hline
		$x=4$&&\\
		\hline
		$x=3$&&\\
		\hline
		$x=2.5$&&\\
		\hline
		$x=2.25$&&\\
		\hline
		$x=2.1$&&\\
		\hline
		$x=2$&&\\
		\hline
		$x=1.9$&&\\
		\hline
		$x=1.75$&&\\
		\hline
		$x=1.5$&&\\
		\hline
		$x=1$&&\\
		\hline
		\end{tabular}
		\end{center}}}
		\newpage
	\begin{enumerate}
	\item  Now, use technology to sketch a rough graph $f(x)$ on the interval $(0,5]$ and add the secant lines from part $a$. (Your graph may be messy...It's ok.) Add in the tangent line to the graph at $P.$ Label the secant lines with their respective slopes. What can you conclude about the slope of the tangent line to $f(x)$ at $P$?
	\vfill
	\item Write a best guess for the equation of the line tangent to $f(x)$ at point $P$. Is your equation plausible?
	\vspace{1.5in}
	
	\end{enumerate}
	\newpage

\item The table shows the position of a cyclist after accelerating from rest.\\

\begin{tabular}{|c||c|c|c|c|c|c|c|c|c|}
$t$ (hours) &0&0.5&1&1.5&2&2.5&3&3.5&4\\
\hline
$d$ (miles) &0&6.2&13.4&23.1&33.4&44.6&54.7&62.6&70\\
\end{tabular}
\begin{enumerate}
\item What is the cyclist's average velocity on the 4 hours of the bike ride?
\vfill
\item Estimate the cyclist's average velocity in miles per hour on each of the time intervals below:
\begin{enumerate}
\item $[0,1.5]$\\ \vfill
\item $[0.5,1.5]$\\ \vfill
\item $[1,1.5]$\\ \vfill
\item $[1.5,2]$\\ \vfill
\item $[1.5,2.5]$\\ \vfill
\item $[1.5,3]$\\ \vfill
\end{enumerate}

\item The calculations above can be used to estimate the \emph{instantaneous} velocity of the cyclist at what time? What would your estimate be?\\  \vfill
\end{enumerate}
\newpage
BONUS: If you understood what we did in class today, you should be able to answer the questions below.
\item In words, what is a secant line, what is a tangent line and how are they different?
\vfill
\item Justify the assertion that the problem of finding the slope of the tangent to a graph at a point is the same problem as finding the instantaneous velocity of an object given its position.
\vfill
\item Explain why the method we use to find the slope of the secant line (i.e. $m_{sec}=\frac{y_2-y_1}{x_2-x_1�}$)  \emph{cannot} be used to find the slope of the tangent line?
\vfill
\end{enumerate}
\end{document}