
\documentclass[11pt,fleqn]{article} 
\usepackage[margin=0.8in, head=0.8in]{geometry} 
\usepackage{amsmath, amssymb, amsthm}
\usepackage{fancyhdr} 
\usepackage{palatino, url, multicol}
\usepackage{graphicx, pgfplots} 
\usepackage[all]{xy}
\usepackage{polynom} 
%\usepackage{pdfsync} %% I don't know why this messes up tabular column widths
\usepackage{enumerate}
\usepackage{framed}
\usepackage{setspace}
\usepackage{array,tikz}

\pgfplotsset{compat=1.6}

\pgfplotsset{soldot/.style={color=black,only marks,mark=*}} \pgfplotsset{holdot/.style={color=black,fill=white,only marks,mark=*}}
\pgfplotsset{my style/.append style={axis x line=middle, axis y line=
middle, xlabel={$x$}, ylabel={$y$} }}

%axis equal 
\pagestyle{fancy} 
\lfoot{}
\rfoot{3-9 Derivatives of exponentials and logs}

\begin{document}
\renewcommand{\headrulewidth}{0pt}
\newcommand{\blank}[1]{\rule{#1}{0.75pt}}
\newcommand{\bc}{\begin{center}}
\newcommand{\ec}{\end{center}}
\renewcommand{\d}{\displaystyle}

\vspace*{-0.7in}

%%%%%%%%%intro page
\begin{center}
  \large
  \sc{Section 3-9: Derivatives of Exponential Functions and Logarithms}\\
\end{center}
\begin{enumerate}
\item Recall the definition of the derivative:\\
\vspace{1in}
\item Let $f(x)=e^x.$ \textbf{Estimate} $f'(x)$ (a.k.a. the slope of the tangent line) using the limit definition for each of the values below. (Use a calculator!)
\begin{enumerate}
	\item $f'(0)$
	\vfill
	\item $f'(1)$
	\vfill
	\item $f'(2)$
	\vfill
	\item $f'(-1)$
	\vfill
\end{enumerate}
\item Derivative Rules for Exponential Functions	
\vspace{1.5in}
\newpage
\item Examples: Find the derivatives.

	\begin{enumerate}
	\begin{multicols}{2}
	\item $y=x^4e^x$
	\vfill
	\item $y=e^{x^2}$
	\vfill
	\end{multicols}
	\vfill
	\begin{multicols}{2}
	\item $y=5^{-x}$
	\vfill
	\item $f(x)=x^5+5^x$
	\vfill
	\end{multicols}
	\vfill
	\end{enumerate}

\item A population of bacteria is modeled by the equation $P(t)=100e^{0.04t}$ where $P$ is the number of bacterial and $t$ is measured in hours.
	\begin{enumerate}
	\item Find $P(0), \: P(1),$ and $P(100).$ Give units with your answers. What do these numbers represent?
	\vfill
	\item  Find $P'(0), \: P'(1),$ and $P'(100).$ Give units with your answers. What do these numbers represent?
	\vfill

	\item Find $P'(0)/P(0)$, $P'(1)/P(1)$ and $P'(100)/P(100).$ What do these numbers represent?	\vfill
	\end{enumerate}
\item Let $P(t)=P_0e^{kt}$. Find $P'(t)/P(t)$ and use this to explain what $k$ represents.
\vfill
\end{enumerate}
\end{document}
\newpage
\item Write $y=\log_2(x)$ and $y=\ln (x)$ in terms of exponential functions.
\vfill
\item Use the expressions in \#5 to find formulas for the derivatives of $y=\log_2(x)$ and $y=\ln (x).$
\vspace{3in}
\item Examples:
	\begin{enumerate}
	\item $y=x \ln(x)$
	\vfill
	\item $y=\log(x^2-5)$
	\vfill
	\item $y=\ln\left(\frac{x(x^2+1)^3}{100(x+1)} \right)$
	\vfill
	\item $y=(\sin(x))^x$
	\vspace{1.5in}
	\end{enumerate}

	
\end{enumerate}
\end{document}

