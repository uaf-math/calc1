\documentclass[11pt,fleqn]{article} 
\usepackage[margin=0.8in, head=0.8in]{geometry} 
\usepackage{amsmath, amssymb, amsthm}
\usepackage{fancyhdr} 
\usepackage{palatino, url, multicol}
\usepackage{graphicx} 
\usepackage[all]{xy}
\usepackage{polynom} 
\usepackage{pdfsync}
\usepackage{enumerate}
\usepackage{framed}
\usepackage{setspace, adjustbox}
\usepackage{array%,tikz, pgfplots
}

\usepackage{tikz, pgfplots}
\usetikzlibrary{calc}
%\pgfplotsset{my style/.append style={axis x line=middle, axis y line=
%middle, xlabel={$x$}, ylabel={$y$}, axis equal }}
%
\pagestyle{fancy} 
\lfoot{UAF Calculus I}
\rfoot{4-3-b}


\newcommand{\be}{\begin{enumerate}}
\newcommand{\ee}{\end{enumerate}}

\newcommand{\bi}{\begin{itemize}}
\newcommand{\ei}{\end{itemize}}

\begin{document}
\setlength{\parindent}{0cm}
\renewcommand{\headrulewidth}{0pt}
\newcommand{\blank}[1]{\rule{#1}{0.75pt}}
\renewcommand{\d}{\displaystyle}
\vspace*{-0.7in}
\begin{center}
 {\large{ \sc{Section 4.3: Maximums and Minimums (closed-interval method)}}}
\end{center}
 \begin{enumerate}
 \item The Extreme Value Theorem
 \vspace{2in}
 \item For each problem below, (i) find all critical numbers of the function on the given interval, (ii) use the Extreme Value Theorem to determine the absolute maximum and absolute minimum of the function, and (iii) use technology to graph the function on the interval to confirm your answer.
 \begin{enumerate}
 \item $f(x)=3x^{1/3}-x$ on $[-1,8]$
 \newpage
 \item $f(x)=\cos(x) - \frac{x}{2}$ on $[0,2\pi]$
 \vfill
 \item $g(x)=\frac{2x}{x^2+1}$ on $[0,10]$
 \vfill
 \end{enumerate}
\item (Bonus Problem) An object with a weight of $W$ is dragged along a horizontal plan by a force acting along a rope attached to the object. If the rope makes an angle of $\theta$ with the plane, then the magnitude of the force is
 $$F=\frac{\mu W}{\mu \sin \theta + \cos \theta}=\mu W (\mu \sin \theta + \cos \theta)^{-1}$$ where $\mu$ is a positive constant called the coefficient of friction. Assume $0 \leq \theta \leq \pi/2.$ Show that $F$ is minimized when $\tan \theta = \mu.$
 \vspace{1.5in}
 \end{enumerate}
\end{document}