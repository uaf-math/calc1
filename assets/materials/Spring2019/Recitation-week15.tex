
\documentclass[12pt]{article}

% Layout.
\usepackage[top=1in, bottom=0.75in, left=1in, right=1in, headheight=1in, headsep=6pt]{geometry}
\usepackage{framed}

% Fonts.
\usepackage{mathptmx}
\usepackage[scaled=0.86]{helvet}
\renewcommand{\emph}[1]{\textsf{\textbf{#1}}}

% TiKZ.
\usepackage{tikz, pgfplots}
\usetikzlibrary{calc}
\pgfplotsset{my style/.append style={axis x line=middle, axis y line=
middle, xlabel={$x$}, ylabel={$y$}, axis equal }}

% Misc packages.
\usepackage{amsmath,amssymb,latexsym}
\usepackage{graphicx}
\usepackage{array}
\usepackage{xcolor}
\usepackage{multicol}

% Commands to set various header/footer components.
\makeatletter
\def\doctitle#1{\gdef\@doctitle{#1}}
\doctitle{Use {\tt\textbackslash doctitle\{MY LABEL\}}.}
\def\docdate#1{\gdef\@docdate{#1}}
\docdate{Use {\tt\textbackslash docdate\{MY DATE\}}.}
\def\doccourse#1{\gdef\@doccourse{#1}}
\let\@doccourse\@empty
\def\docscoring#1{\gdef\@docscoring{#1}}
\let\@docscoring\@empty
\def\docversion#1{\gdef\@docversion{#1}}
\let\@docversion\@empty
\makeatother

% Headers and footers layout.
\makeatletter
\usepackage{fancyhdr}
\pagestyle{fancy}
\fancyhf{} % Clears all headers/footers.
\lhead{\baselineskip 30pt
\emph{\@doctitle\hfill\@docdate}
%\ifnum \value{page} > 1\relax\else\\
%\emph{Name: \rule{3.5in}{1pt}\hfill \@docscoring}\fi
}
%\rfoot{\emph{\@docversion}}
%\lfoot{\emph{\@doccourse}}
%\cfoot{\emph{\thepage}}
\renewcommand{\headrulewidth}{0pt}%
\makeatother

% Paragraph spacing
\parindent 0pt
\parskip 6pt plus 1pt

% A problem is a section-like command. Use \problem{5} to
% start a problem worth 5 points.
\newcounter{probcount}
\newcounter{subprobcount}
\setcounter{probcount}{0}
\newcommand{\problem}[1]{%
\par
\addvspace{4pt}%
\setcounter{subprobcount}{0}%
\stepcounter{probcount}%
\makebox[0pt][r]{\emph{\arabic{probcount}.}\hskip1ex}\emph{[#1 points]}\hskip1ex}
\newcommand{\thesubproblem}{\emph{\alph{subprobcount}.}}

% Subproblems are an enumerate-like environment with a consistent
% numbering scheme. 
% Use \begin{subproblems}\item...\item...\end{subproblems}
\newenvironment{subproblems}{%
\begin{enumerate}%
\setcounter{enumi}{\value{subprobcount}}%
\renewcommand{\theenumi}{\emph{\alph{enumi}}}}%
{\setcounter{subprobcount}{\value{enumi}}\end{enumerate}}

% Blanks for answers in normal and math mode.
\newcommand{\blank}[1]{\rule{#1}{0.75pt}}
\newcommand{\mblank}[1]{\underline{\hspace{#1}}}
\def\emptybox(#1,#2){\framebox{\parbox[c][#2]{#1}{\rule{0pt}{0pt}}}}

% Misc.
\renewcommand{\d}{\displaystyle}
\newcommand{\ds}{\displaystyle}
\def\bc{\begin{center}}
\def\ec{\end{center}}


\doctitle{Math 251: Integration Basics Worksheet}
\docdate{23 April 2019}
\doccourse{UAF Calculus I}
%\docversion{v-sample}
%\docscoring{\blank{0.8in} / 12}

\begin{document}

\small
\begin{framed}
  \textbf{Fill-in these indefinite integrals you (should) already know:}

%\vspace*{-0.35in}
  \begin{multicols}{2}
    \begin{itemize}
    \item $\d \int x^n\, dx = $
    \item $\d \int \sin x\, dx =$
    \item $\d \int \cos x\, dx = $
    \item $\d \int \sec^2 x\, dx =$
    \item $\d \int \sec x\, \tan x\, dx= $
    \item $\d \int \csc^2 x\, dx =$
\columnbreak
%\vspace*{0.32in}
    \item $\d \int \frac 1 x\, dx =$
    \item $\d \int e^x\, dx =$
    \item $\d \int a^x\, dx =$
    \item $\d \int \frac{1}{\sqrt{1-x^2}}\, dx =$ \phantom{sdlfjk asdlfkj alsdkjf asdlfkj asldj}
    \item $\d \int \frac{1}{1+x^2}\, dx =$
    \item $\d \int \csc x\, \cot x\, dx= $
    \end{itemize}
  \end{multicols}
\end{framed}
\small

\begin{itemize}
\item \textbf{Question 1:} How do you check your answers when computing integrals? For example, suppose $\int f(x) dx = F(x) + C$. How do you know you are right?
\vfill

\item \textbf{Question 2:} For what value of $n$ does the reverse power rule for the antiderivative of $x^n$ not apply?  What is the antiderivative of $x^n$ for this value of $n$?
\vfill

\item \textbf{Question 3:} What is the $u$-substitution to use for the following integral?:
    $$\int f\left(g(x)\right)\,g'(x)\,dx = \hspace{80mm}$$
\vfill

\item \textbf{Question 4:} When you check an indefinite integral which you did by substitution, what derivative rule will you always use?
\vfill

\end{itemize}
\end{document}