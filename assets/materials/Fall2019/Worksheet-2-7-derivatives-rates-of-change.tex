\documentclass[11pt,fleqn]{article} 
\usepackage[margin=0.8in, head=0.8in]{geometry} 
\usepackage{amsmath, amssymb, amsthm}
\usepackage{fancyhdr} 
\usepackage{palatino, url, multicol}
\usepackage{graphicx} 
\usepackage[all]{xy}
\usepackage{polynom} 
\usepackage{pdfsync}
\usepackage{enumerate}
\usepackage{framed}
\usepackage{setspace, adjustbox}
\usepackage{array,tikz}
\pagestyle{fancy} 
\lfoot{UAF Calculus I}
\rfoot{2-6 Limits at Infinity}


\newcommand{\be}{\begin{enumerate}}
\newcommand{\ee}{\end{enumerate}}

\newcommand{\bi}{\begin{itemize}}
\newcommand{\ei}{\end{itemize}}

\begin{document}
\setlength{\parindent}{0cm}
\renewcommand{\headrulewidth}{0pt}
\newcommand{\blank}[1]{\rule{#1}{0.75pt}}
\renewcommand{\d}{\displaystyle}
\vspace*{-0.7in}



  \begin{center}
  \large \sc{Section 2-7: Derivatives and Rates of Change}
\end{center}

\begin{enumerate}
\item  Given the curve $y=g(x),$ 
	\be
	\item Write an expression for the slope of the secant line through the points $P(8, g(8))$ and $Q(c, g(c)).$
	\vspace{1in}
	\item Write an expression for the slope of the tangent line at $P(8,g(8)).$
	\vspace{1in}
	\item Sketch a ``cartoon" including $g(x)$, $P$ and $Q$ and use it to illustrate the computations in parts (a) and (b) above.
	\vspace{2in}
	\ee
	\newpage
  \item 
  \be
  	\item \textbf{Fill in the boxes} The derivative of a function $f$ at a number $a$ is:\\
	
	\begin{tabular}{rcll}
	$ f'(a)$&$=$&${\d{\lim_{h \to 0}}}$& {\raisebox{-.5\height}{ \framebox(200,60){}}}\\
	\end{tabular}
	\vspace{.2in}
	\item Use the expression above to find $f'(2)$ for $f(x)=6x-3x^2.$\\
	
	\vfill
	\item Find $f(2)$.\\
	
	\vspace{.2in}
	\item Use the answers to parts (a) and (b) to write an equation of the line tangent to $f(x)$ when $x=2.$
	\vfill
	\item Sketch a ``cartoon" including $f(x)$ and that tangent line. Is your answer in part (c) plausible?
	\vfill
\ee
  \ee
\end{document}