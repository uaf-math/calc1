\documentclass[11pt,fleqn]{article} 
\usepackage[margin=0.8in, head=0.8in]{geometry} 
\usepackage{amsmath, amssymb, amsthm}
\usepackage{fancyhdr} 
\usepackage{palatino, url, multicol}
\usepackage{graphicx, pgfplots} 
\usepackage[all]{xy}
\usepackage{polynom,tabularx, setspace} 
%\usepackage{pdfsync} %% I don't know why this messes up tabular column widths
\usepackage{enumerate}
\usepackage{framed}
\usepackage{setspace}
\usepackage{array}
\usepackage{pgf,tikz}
\usepackage{mathrsfs}

\usepackage[parfill]{parskip}
\usetikzlibrary{arrows}
\def\ds{\displaystyle}
\usetikzlibrary{calc}

\pgfplotsset{compat=1.6}

\pgfplotsset{soldot/.style={color=blue,only marks,mark=*}} \pgfplotsset{holdot/.style={color=blue,fill=white,only marks,mark=*}}

\renewcommand{\headrulewidth}{0pt}
\newcommand{\blank}[1]{\rule{#1}{0.75pt}}
\newcommand{\bc}{\begin{center}}
\newcommand{\ec}{\end{center}}
\newcommand{\be}{\begin{enumerate}}
\newcommand{\ee}{\end{enumerate}}

\renewcommand{\d}{\displaystyle}

\newcommand{\ans}[1][2]{ \ \rule{#1 in}{.5 pt} \ }


\pagestyle{fancy} 
%\lfoot{Uses a calculator}
\rfoot{Section 4-8}

\begin{document}

\begin{center}
  \Large\sc{Section 4.8 Newton's Method}\\
\end{center}

%\vspace*{-0.7in}

\begin{enumerate}
\item Newton's Method is an iterative rule for finding roots.\\

\textbf{Given:} $F(x)$\\
\textbf{Want:} $a$ so that $F(a)=0$\\
\textbf{Guess:} $x_0$ close to $a$\\
\vspace{.2in}

\textbf{Plug in and Repeat:}
\vspace{1in}

\item Let $F(x)=x^2-2.$
	\begin{enumerate}
	\item Using elementary algebra, find $a$ such that $F(a) = 0.$ (Find $a$ exactly and find a decimal approximation with at least 9 decimal places.)
	\vspace{1in}
	\item Find a formula for $x_{k+1}.$ Simplify it.
	\vspace{1in}
	\item Using an initial guess of $x_0=2,$ complete 4 iterations of Newton's method to find $x_4$ and compare your answer to the one in part (a).
	\end{enumerate}
\newpage
\item This page is intended to illustrate \emph{how} Newton's Method works and \emph{why} it has the formula it does.\\

Again, consider the function $F(x)=x^2-2.$
	\begin{enumerate}
	\item Find the linearization $L(x)$ of $F(x)$ at $x=2$.  Leave your answer
in point-slope form.
\vfill
	\item I've graphed $F(x)$ for you below.  Mark where $\sqrt{2}$ is on this diagram and add to this diagram
the graph of $L(x)$.

\hfil\includegraphics{WSNewton.pdf}
	\item Find the number $x_1$ such that $L(x_1)=0$.
\vfill
	\item In the diagram above, label the point $x_1$ on the $x$-axis.
	\item Let's do it again!  Find the linearization $L(x)$ 
of $F(x)$ at $x=x_1$.
\vfill

	\item Add the graph of this new linearization to your diagram
on the first page.
	\item Find the number $x_2$ such that $L(x_2)=0$.  Then label the point $x=x_2$ in the diagram.
\vfill
	\item Compare your numbers for $x_1$ and $x_2$ to those on the previous page. They should be the same.
	\newpage
	\item Let's be a little more systematic.  Suppose we have an estimate
$x_k$ for $\sqrt{2}$.
\begin{itemize}
	\item Compute $F(x_k)$.
	\item Compute $F'(x_k)$.
	\item Compute the linearization of $F(x)$ at $x=x_k$.
	\vskip 0pt plus 0.3 fill
	$L(x)=$
	\vskip 0pt plus 0.3 fill
	\item Find the number $x_{k+1}$ such that $L(x_{k+1})=0$.  You
	should try to find as simple an expression as you can. Compare this to the formula we used on problem 1b from page 1. 
	\vfill
\end{itemize}
\end{enumerate}
\item Indicate on the picture below, the values of $x_1,$ $x_2$ and $x_3$ that would be obtained from Newton's Method using an initial guess of $x_0=0.5.$\\
\includegraphics{CosPic1.pdf}
\newpage	
\item  Try to solve
\[
e^{-x} - x =0
\]
by hand.  
\vskip 2cm
\item  Explain why there is a solution between $x=0$ and $x=1$.
\vskip 4cm
\item Starting with $x_0=1$, find an approximation of the solution 
of $e^{-x}-x=0$ to 6 decimal places.  During your computation, keep
track of each $x_k$ to at least 9 decimal places of accuracy.
\vfill
\end{enumerate}
\end{document}
