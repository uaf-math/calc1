\documentclass[11pt,fleqn]{article} 
\usepackage[margin=0.8in, head=0.8in]{geometry} 
\usepackage{amsmath, amssymb, amsthm}
\usepackage{fancyhdr} 
\usepackage{palatino, url, multicol}
\usepackage{graphicx} 
\usepackage[all]{xy}
\usepackage{polynom} 
\usepackage{pdfsync}
\usepackage{enumerate}
\usepackage{framed}
\usepackage{setspace, adjustbox}
\usepackage{array%,tikz, pgfplots
}

\usepackage{tikz, pgfplots}
\usetikzlibrary{calc}
%\pgfplotsset{my style/.append style={axis x line=middle, axis y line=
%middle, xlabel={$x$}, ylabel={$y$}, axis equal }}
%
\pagestyle{fancy} 
\lfoot{UAF Calculus I}
\rfoot{3-3 Derivatives of Trig Functions}


\newcommand{\be}{\begin{enumerate}}
\newcommand{\ee}{\end{enumerate}}

\newcommand{\bi}{\begin{itemize}}
\newcommand{\ei}{\end{itemize}}

\begin{document}
\setlength{\parindent}{0cm}
\renewcommand{\headrulewidth}{0pt}
\newcommand{\blank}[1]{\rule{#1}{0.75pt}}
\renewcommand{\d}{\displaystyle}
\vspace*{-0.7in}
\begin{center}
 {\large{ \sc{Section 3.3 Derivatives of Trigonometric Functions (Day 2)}}}
\end{center}
\begin{enumerate}

\item Fill in the table below. \\

\begin{framed}
  \textbf{Derivatives of Trigonometric Functions:}

\begin{multicols}{2}{
\vspace{-0.3in}
  \begin{itemize}
  \item $\frac{d}{dx} (\sin x) =$ \blank{0.85in}
  \item $\frac{d}{dx} (\cos x) = $ \blank{0.85in}
  \item $\frac{d}{dx} (\tan x) = $ \blank{0.85in}
\columnbreak
  \item $\frac{d}{dx} (\csc x) = $ \blank{0.85in}
  \item $\frac{d}{dx} (\sec x) = $ \blank{0.85in}
  \item $\frac{d}{dx} (\cot x) = $ \blank{0.85in}
  \end{itemize}}
\end{multicols}
\end{framed}

\item Derive the formula for $\frac{d}{dx}\left[ \tan (x) \right].$
\vfill
\item Find the derivative of $\displaystyle{y=\frac{\sec x}{1-x\tan x}}.$
\vfill
\newpage
%\item For what values of $x$ does the graph of $f(x) = x
%+ 2 \sin x$ have a horizontal tangent?
%\vfill
\item If $f(\theta) = e^{\theta} \cos(\theta)$, find $f '' (\theta).$ Simplify your answers here. 
\vfill
\item Find $\displaystyle{\frac{d}{dt}\left[t \sin t \cos t\right]}$. 
\vfill 
\newpage
\item An elastic band is hung on a hook and a mass is hung on the lower end of the band. When the mass is pulled down 2 cm past its rest position and then released, it vibrates vertically. The equation of motion is $$s=2 \cos t +3 \sin t, \text{\: for\:} t \geq 0,$$ where $s$ is measured in centimeters and $t$ is measured in seconds. (We are taking the positive direction to be downward.)
\begin{enumerate}
\item Why might you expect to use sines and cosines to model this particular problem?\\
\vspace{.4in}
\item Sketch a cartoon of what this problem is describing.\\
\vfill
\item Find $s(0), s'(0),$ and $s''(0)$ including units.
\vfill
\item What does $s(0)$ tell you about the mass in the context of the problem?\\
\vfill
\item What does $s'(0)$ tell you about the mass in the context of the problem?\\
\vfill
\item What does $s''(0)$ tell you about the mass in the context of the problem?\\
\vfill
\end{enumerate}
\newpage
\item A 12 foot ladder rests against a wall. Let $\theta$ be the angle between the ladder and the wall and let $x$ be the distance from the base of the ladder and the wall.
\begin{enumerate}
\item Compute $x$ as a function of $\theta$. (Drawing a picture will help.)
\vfill

\item How fast does $x$ change with respect to $\theta$ when $\theta = \pi/6$? (Get an exact answer and a decimal approximation.)
\vfill
\item Interpret your answer from part (b) in the context of the problem. (Units will help you here.)
\vfill
\item Determine how far the ladder is from the wall when $\theta = \pi/6.$
\vfill
\item If the angle $\theta$ was \emph{decreased} from $\pi/6$ radians to $\frac{\pi}{6} - \frac{1}{100}$ radians, estimate how the distance to the wall would change. Try to answer this question using only your answer from part b.
\vfill
\end{enumerate}

\end{enumerate}
\end{document}