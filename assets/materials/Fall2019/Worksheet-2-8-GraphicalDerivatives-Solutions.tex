\documentclass[11pt,fleqn]{article} 
\usepackage[margin=0.8in, head=0.8in]{geometry} 
\usepackage{amsmath, amssymb, amsthm}
\usepackage{fancyhdr} 
\usepackage{palatino, url, multicol}
\usepackage{graphicx} 
\usepackage[all]{xy}
\usepackage{polynom, cancel} 
%\usepackage{pdfsync}
\usepackage{enumerate}
\usepackage{framed}
\usepackage{setspace}
\usepackage{array,tikz,pgfplots}

\pgfplotsset{compat=1.6}

\pgfplotsset{soldot/.style={color=black,only marks,mark=*}} \pgfplotsset{holdot/.style={color=black,fill=white,only marks,mark=*}}

\pagestyle{fancy} 
\lfoot{UAF Calculus 1}
\rfoot{2-8}


\newcommand{\be}{\begin{enumerate}}
\newcommand{\ee}{\end{enumerate}}
\newcommand{\bpm}{\begin{pmatrix}}
\newcommand{\epm}{\end{pmatrix}}

\theoremstyle{definition}
\newtheorem{exercise}{Exercise}


\begin{document}


\renewcommand{\headrulewidth}{0pt}



%version 1
\begin{Large} 
\begin{center}{\sc
Worksheet \S2.8: Finding Derivatives Graphically - Solutions }\end{center} \end{Large}

{\bf The original functions are shown dashed, and the derivatives are shown thick.}
%This is due {\bf Tuesday, September 27, 2011, at the beginning of class.}

\bigskip
\begin{exercise} Sketch the derivatives of graphs 1 and 2.\end{exercise}
\begin{tabular}{c  c}
Graph 1 & Graph 2 \\
\includegraphics[width=.5\linewidth]{Graph1Ans.pdf} &\includegraphics[width=.5\linewidth]{Graph2Ans.pdf}\\
%\includegraphics[width=.5\linewidth]{EmptyGraph1.pdf} &\includegraphics[width=.5\linewidth]{EmptyGraph2.pdf}
\end{tabular}
\vfill

%\newpage

\begin{exercise}
The equation of Graph 1 is 
\[f(x) = 2 x + \frac{x^2}{4}.\]

Use the definition of the derivative to compute the derivative $f'(x)$. (Attach a separate page if you need more room.) What kind of function is $f'(x)$? How does the graph of $f'(x)$ compare to the derivative you drew?
\end{exercise}

\begin{align*}
f'(x) &= \lim_{h \to 0} \frac{\left( 2(x + h) + \frac{ (x +h)^{2}}{4} \right)  - \left(2 x + \frac{x^2}{4}\right)}{h} \\
%
&= \lim_{h \to 0} \frac{1}{h} \left(\cancel{2x} + 2h + \cancel{\frac{1}{4} x^{2} }+  \frac{1}{4}\cdot 2 x h +  \frac{1}{4}(h)^{2}  -\cancel{2x}  - \cancel{\frac{1}{4} x^{2}}\right)\\
%
&= \lim_{h \to 0} \frac{1}{\cancel{h}} \cdot \cancel{h} \left( 2+  \frac{1}{4}\cdot 2 x +  \frac{1}{4}(h)  \right)\\
&= 2+ \frac{x}{2}
\end{align*}
 Look! The derivative is a line! And hopefully you sketched something line-like for Graph 1.
\vfill
\newpage
\begin{exercise} Sketch the derivatives of graphs 3 and 4.\end{exercise}

\begin{tabular}{c  c}
Graph 3 & Graph 4 \\
\includegraphics[width=.45\linewidth]{Graph3Ans.pdf} &\includegraphics[width=.5\linewidth]{Graph6Ans.pdf}\\
%\includegraphics[width=.45\linewidth]{EmptyGraph3.pdf} &\includegraphics[width=.5\linewidth]{EmptyGraph6.pdf}
\end{tabular}

\begin{exercise}  What is an important difference between the derivative of graph 3 and the derivative of graph 4? Use terminology from calculus.\end{exercise}

\emph{The derivative of Graph 4 is not continuous.}

%\vspace{.5in}

%\newpage
%
\begin{exercise} Sketch the derivatives of graphs 5 and 6.\end{exercise}

\begin{tabular}{c  c}
Graph 5 & Graph 6 \\
\includegraphics[width=.5\linewidth]{Graph4Ans.pdf} &\includegraphics[width=.5\linewidth]{Graph5Ans.pdf}\\
%\includegraphics[width=.5\linewidth]{EmptyGraph4.pdf} &\includegraphics[width=.5\linewidth]{EmptyGraph5.pdf}
\end{tabular}

 \end{document}
 \end

 