% !TEX TS-program = pdflatexmk
\documentclass[12pt]{article}

% Layout.
\usepackage[top=1in, bottom=0.75in, left=1in, right=1in, headheight=1in, headsep=6pt]{geometry}

% Fonts.
\usepackage{mathptmx}
\usepackage[scaled=0.86]{helvet}
\renewcommand{\emph}[1]{\textsf{\textbf{#1}}}

% TiKZ.
\usepackage{tikz, pgfplots}
\usetikzlibrary{calc}
\pgfplotsset{my style/.append style={axis x line=middle, axis y line=
middle, xlabel={$x$}, ylabel={$y$}, axis equal }}

% Misc packages.
\usepackage{amsmath,amssymb,latexsym}
\usepackage{graphicx}
\usepackage{array}
\usepackage{xcolor}
\usepackage{multicol}

% Commands to set various header/footer components.
\makeatletter
\def\doctitle#1{\gdef\@doctitle{#1}}
\doctitle{Use {\tt\textbackslash doctitle\{MY LABEL\}}.}
\def\docdate#1{\gdef\@docdate{#1}}
\docdate{Use {\tt\textbackslash docdate\{MY DATE\}}.}
\def\doccourse#1{\gdef\@doccourse{#1}}
\let\@doccourse\@empty
\def\docscoring#1{\gdef\@docscoring{#1}}
\let\@docscoring\@empty
\def\docversion#1{\gdef\@docversion{#1}}
\let\@docversion\@empty
\makeatother

% Headers and footers layout.
\makeatletter
\usepackage{fancyhdr}
\pagestyle{fancy}
\fancyhf{} % Clears all headers/footers.
\lhead{\baselineskip 30pt
\emph{\@doctitle\hfill\@docdate}
\ifnum \value{page} > 1\relax\else\\
\emph{Name: \rule{3.5in}{1pt}\hfill \@docscoring}\fi}
\rfoot{\emph{\@docversion}}
\lfoot{\emph{\@doccourse}}
\cfoot{\emph{\thepage}}
\renewcommand{\headrulewidth}{0pt}%
\makeatother

% Paragraph spacing
\parindent 0pt
\parskip 6pt plus 1pt

% A problem is a section-like command. Use \problem{5} to
% start a problem worth 5 points.
\newcounter{probcount}
\newcounter{subprobcount}
\setcounter{probcount}{0}
\newcommand{\problem}[1]{%
\par
\addvspace{4pt}%
\setcounter{subprobcount}{0}%
\stepcounter{probcount}%
\makebox[0pt][r]{\emph{\arabic{probcount}.}\hskip1ex}\emph{[#1 points]}\hskip1ex}
\newcommand{\thesubproblem}{\emph{\alph{subprobcount}.}}

% Subproblems are an enumerate-like environment with a consistent
% numbering scheme. 
% Use \begin{subproblems}\item...\item...\end{subproblems}
\newenvironment{subproblems}{%
\begin{enumerate}%
\setcounter{enumi}{\value{subprobcount}}%
\renewcommand{\theenumi}{\emph{\alph{enumi}}}}%
{\setcounter{subprobcount}{\value{enumi}}\end{enumerate}}

% Blanks for answers in normal and math mode.
\newcommand{\blank}[1]{\rule{#1}{0.75pt}}
\newcommand{\mblank}[1]{\underline{\hspace{#1}}}
\def\emptybox(#1,#2){\framebox{\parbox[c][#2]{#1}{\rule{0pt}{0pt}}}}

% Misc.
\renewcommand{\d}{\displaystyle}
\newcommand{\ds}{\displaystyle}
\def\bc{\begin{center}}
\def\ec{\end{center}}


\doctitle{Math 251: Sample Derivative Proficiency for \S3.1--3.4}
\docdate{October 4, 2019}
\doccourse{UAF Calculus I}
\docversion{v-1}
\docscoring{\blank{0.8in} / 12}
\begin{document}
\begin{itemize}
\item 
There are 12 points possible on this proficiency: one point per problem with
no partial credit. 

\item You have \textbf{30} minutes to complete this proficiency.

\item No aids (book, calculator, etc.) are permitted.  

\item \fbox{Do \textbf{not} simplify your expressions}.

\item {\bf Your final answers should start with $f'(x)=$, $\frac{dy}{dx}=$} or
something similar.

\item {\fbox{Box} your final answer. }
\end{itemize}

{\it Note that this sample derivative proficiency is slightly different from the actual derivative proficiency, because there are a few functions ($\ln(x)$, inverse trig functions, implicit differentiation in general) that we haven't covered yet.} 

\problem{12}  Compute the derivatives of the following functions.
\begin{subproblems}

\item $\ds f(x)= \sqrt[5]{x} + 4x^3 + \frac{x-\sqrt{2}}{9}$
%old 1$\ds f(x)= \sqrt{5x}-\frac{e^x}{2}+ \ln 4$\\
 %old 1$\ds f(x) = \pi x^{1/8} + 7 e^x +\sqrt{5}$
\vfill

\item $\ds y=x^3 \tan(x)$
%old 2 $f(t)=\frac{8t+t^{2/3}-1}{t}$\\
%old 2$\ds f(t) = \frac{t^3-t^\frac{3}{2}+1}{\sqrt{t}}$
\vfill

\item $\ds y=\frac{\sec(x)}{1+e^{x}}$\\
%$\textcolor{red}{y=}\ds \frac{\tan(x)}{1+\ln(x)}$
%$h(x)=e^{x/3}\sin(x)$\\
%old 3$\ds f(x) = (x^3-x)\cos(x)$
\vfill
\newpage

\item $\ds y=\sin(a x)e^{b\, x^2}$ where $a$ and $b$ are fixed constants.
%$\ds y=(2x^{-1/5}+6) \ln x$\\
%old 4$\ds f(x) = \frac{\sin(x)}{1+e^{-3x}}$
\vfill

\item $\ds f(x) = %\arcsin( \cos( 7x ))$
\frac{\cos(x)}{\sin(x)}$\\
%old 5$\ds f(x) = \frac{1}{\sin(x)}$
\vfill

\item $\ds g(x) = \sqrt{2+\sin^2(6x)}$
%$\ds f(x)=x^k+e^{-kx},$ where $k$ is a fixed constant\\
%old 6$\ds f(t) = t\ln(at)$
\vfill
\newpage

\item $\ds y=\tan\left(x^3 \cdot 5^x\right)$
%old 7$\ds f(x)=\tan(x)x^{\frac{1}{2}}e^{3x}$
\vfill

\item %$\ds f(x)=\sqrt{x}\;\ln(x)\arctan(x)$
%old 8$ 
$\ds f(z) = \ds \sec(\sqrt{z})$
\vfill

\item $\ds y=\sin\left(\frac{x}{x-3}\right)$
% \textcolor{red}{Change secant to cosine.}
%old 9$\ds f(t) = \ds \sec(\ln(1+t^2))$
\vfill

\newpage

\item $\ds h(x) = %\ln \left( {\pi x^{3}} + (5x)^7 \right)$
% $\ds h(x) = 
\cos\left( e^{\pi x} - (4x)^9 \right)$
% \textcolor{red}{$\ds h(x) = \ln \left( {\pi x^2} - (4x)^9 \right)$}
%$\ds f(x)=\ln ( x + \sqrt{x^2+1} )$

%old 10$\ds f(x) = \ds \sin^5(x^2+x)$
\vfill

\item $\ds g(x)= \ds (\sin(x^2+x))^{5}$%\frac{e^5}{3-x^2}$
%old 11$\ds f(x) = \frac{1}{9x} + \left(\pi\frac{x+2}{2}\right)^3$
\vfill

\item $\ds f(x) = \frac{1}{9x}$
\vfill

%\item Compute $dy/dx$ if\quad $\ds -2x^3+x^2y^2+y^5=0$. You must solve for $dy/dx$.

%Compute $ds/dt$ if\quad $\ds s^3 e^t +5=2st^2 $. You must solve for $ds/dt$.\\
%old 12 Compute $dy/dx$ if\quad $\ds e^y\sin(x)=1-xy$. You must solve for $dy/dx$.
\vfill
\end{subproblems}
\end{document}