\documentclass[11pt,fleqn]{article} 
\usepackage[margin=0.8in, head=0.8in]{geometry} 
\usepackage{amsmath, amssymb, amsthm}
\usepackage{fancyhdr} 
\usepackage{palatino, url, multicol}
\usepackage{graphicx, pgfplots} 
\usepackage[all]{xy}
\usepackage{polynom,tabularx} 
%\usepackage{pdfsync} %% I don't know why this messes up tabular column widths
\usepackage{enumerate}
\usepackage{framed}
\usepackage{setspace}
\usepackage{array}
\usepackage{pgf,tikz}
\usepackage{mathrsfs}

\usepackage[parfill]{parskip}
\usetikzlibrary{arrows}

\usetikzlibrary{calc}

\pgfplotsset{compat=1.6}

\pgfplotsset{soldot/.style={color=blue,only marks,mark=*}} \pgfplotsset{holdot/.style={color=blue,fill=white,only marks,mark=*}}

\renewcommand{\headrulewidth}{0pt}
\newcommand{\blank}[1]{\rule{#1}{0.75pt}}
\newcommand{\bc}{\begin{center}}
\newcommand{\ec}{\end{center}}
\newcommand{\be}{\begin{enumerate}}
\newcommand{\ee}{\end{enumerate}}

\renewcommand{\d}{\displaystyle}

\newcommand{\ans}[1][2]{ \ \rule{#1 in}{.5 pt} \ }


\pagestyle{fancy} 
%\lfoot{Uses a calculator}
\rfoot{Section 4-5 (day 1)}

\begin{document}

\vspace*{-0.7in}

\begin{center}
  \Large\sc{Section 4.5 Curve Sketching (Day 1)}\\
\end{center}
\begin{center}Techniques for carefully sketching functions\end{center}

When sketching a graph of a function $f(x)$, you want to clearly indicate all the important features of the function, including: its domain, the $x$- and $y$-intercepts (maybe), intervals on which the function is increasing and decreasing, local maxima and minima, intervals where the function is concave up and concave down, inflection points, horizontal asymptotes and vertical asymptotes.

To do so, you need to do the following steps:
\be
\item {\bf Determine the domain.} Are there any places where the function is undefined? For example, is there a denominator which could be zero? Does your function include functions such as $\sqrt{x}$ or $\ln(x)$, which are not defined for negative $x$-values?

\item {\bf Determine the $y$-intercept. Determine the $x$-intercepts if it's not too hard.} To determine the $y$-intercept, evaluate $f(0)$. To determine the $x$-intercept, solve $f(x) = 0$. (This may be hard, so you may approximate or skip when appropriate. In addition, not every function has an $x$- or $y$-intercept.)

\item {\bf Are there any vertical asymptotes?} We need to determine the behavior of the function near any points where the function is not defined. Say the function is not defined at $x= a$. Then we need to compute $\displaystyle{\lim_{x\to a^{-}} f(x)}$ and $\displaystyle{\lim_{x\to a^{+}} f(x)}$ to determine whether $f$ has a vertical asymptote at $a$ and what the behavior of the function is near the asymptote.


\item {\bf Determine intervals of increase, decrease, and local maxima/minima} To do so, solve $f'(x) = 0$ for $x$ to find the critical points, and then make a first derivative chart to determine the behavior of the function in the intervals determined by the critical points. Finally, by considering where the function changes from increasing to decreasing (or vice versa), you can determine the local maxima and minima.

\item {\bf Determine intervals of concave up, concave down, and inflection points} To do so, solve $f''(x) = 0$ for $x$ to find the potential inflection points, and then make a second derivative chart to determine the behavior of the function in the intervals determined by the potential inflection points. Finally, by considering where the function changes from CU to CD (or vice versa), you can determine the inflection points.


\item {\bf Are there any horizontal asymptotes?} A horizontal asymptote is a horizontal line which the function approaches ``eventually''; that is, $y = L$ is a horizontal asymptote if $\displaystyle{\lim_{x\to -\infty} f(x)} = L$ or $\displaystyle{\lim_{x\to \infty} f(x)} = L$. So you need to compute both those limits to determine the behavior of the function as $x$ increases without bound.

\item Finally, stitch together all of the information you have determined about the function and sketch the graph of the function, indicating all important features.
\ee

\hrulefill

Today's Practice Example: $k(x) = \frac{x^{3} - 1}{x^{3}+1}$\\

NOTE:  \hspace{.5in} $\displaystyle{k'(x) = \frac{6 x^2}{\left(x^3+1\right)^2}}$,  \hspace{.5in} $\displaystyle{k''(x) = \frac{12 x-24 x^4}{\left(x^3+1\right)^3}.}$\\

A complete solution with all the details is posted on the course webpage under the In-Class Materials for today (Thurs 31 October.)
\end{document}
