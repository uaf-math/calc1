
\documentclass[12pt]{article}

% Layout.
\usepackage[top=1in, bottom=0.75in, left=1in, right=1in, headheight=1in, headsep=6pt]{geometry}

% Fonts.
\usepackage{mathptmx}
\usepackage[scaled=0.86]{helvet}
\renewcommand{\emph}[1]{\textsf{\textbf{#1}}}

% TiKZ.
\usepackage{tikz, pgfplots}
\usetikzlibrary{calc}
\pgfplotsset{compat = newest}
 
\pgfplotsset{my style/.append style={axis x line=middle, axis y line=
middle, xlabel={$x$}, ylabel={$y$}, axis equal
}}

% Misc packages.
\usepackage{amsmath,amssymb,latexsym}
\usepackage{graphicx}
\usepackage{array}
\usepackage{xcolor}
\usepackage{multicol}

% Commands to set various header/footer components.
\makeatletter
\def\doctitle#1{\gdef\@doctitle{#1}}
\doctitle{Use {\tt\textbackslash doctitle\{MY LABEL\}}.}
\def\docdate#1{\gdef\@docdate{#1}}
\docdate{Use {\tt\textbackslash docdate\{MY DATE\}}.}
\def\doccourse#1{\gdef\@doccourse{#1}}
\let\@doccourse\@empty
\def\docscoring#1{\gdef\@docscoring{#1}}
\let\@docscoring\@empty
\def\docversion#1{\gdef\@docversion{#1}}
\let\@docversion\@empty
\makeatother

% Headers and footers layout.
\makeatletter
\usepackage{fancyhdr}
\pagestyle{fancy}
\fancyhf{} % Clears all headers/footers.
\lhead{\baselineskip 30pt
%\emph{\@doctitle\hfill\@docdate}
\emph{\@docdate\hfill\@doctitle}
\ifnum \value{page} > 1\relax\else\\
\emph{Name: \rule{3.5in}{1pt}\hfill \@docscoring}\fi}
\rfoot{\emph{\@docversion}}
\lfoot{\emph{\@doccourse}}
\cfoot{\emph{\thepage}}
\renewcommand{\headrulewidth}{0pt}%
\makeatother

% Paragraph spacing
\parindent 0pt
\parskip 6pt plus 1pt

% A problem is a section-like command. Use \problem{5} to
% start a problem worth 5 points.
\newcounter{probcount}
\newcounter{subprobcount}
\setcounter{probcount}{0}
\newcommand{\problem}[1]{%
\par
\addvspace{4pt}%
\setcounter{subprobcount}{0}%
\stepcounter{probcount}%
\makebox[0pt][r]{\emph{\arabic{probcount}.}\hskip1ex}\emph{[#1 points]}\hskip1ex}
\newcommand{\thesubproblem}{\emph{\alph{subprobcount}.}}

% Subproblems are an enumerate-like environment with a consistent
% numbering scheme. 
% Use \begin{subproblems}\item...\item...\end{subproblems}
\newenvironment{subproblems}{%
\begin{enumerate}%
\setcounter{enumi}{\value{subprobcount}}%
\renewcommand{\theenumi}{\emph{\alph{enumi}}}}%
{\setcounter{subprobcount}{\value{enumi}}\end{enumerate}}

% Blanks for answers in normal and math mode.
\newcommand{\blank}[1]{\rule{#1}{0.75pt}}
\newcommand{\mblank}[1]{\underline{\hspace{#1}}}
\def\emptybox(#1,#2){\framebox{\parbox[c][#2]{#1}{\rule{0pt}{0pt}}}}

% Misc.
\renewcommand{\d}{\displaystyle}
\newcommand{\ds}{\displaystyle}
\def\bc{\begin{center}}
\def\ec{\end{center}}
\def\be{\begin{enumerate}}
\def\ee{\end{enumerate}}


\doctitle{Math F251X: Quiz 7}
\docdate{October 29, 2023}
\doccourse{UAF Calculus I}
\docversion{v-1}
\docscoring{\blank{0.8in} / 25}
\begin{document}

There are 25 points possible on this quiz. No aids (book, calculator, etc.)
are permitted.  {\bf Show all work for full credit.} \textcolor{red}{You should not be using a calculator on this (or any) quiz.}

\problem{9} %(Related Rates Problem) 
Sand is poured onto a surface at a rate of 15 $\mathrm{cm}^{3}$/sec, forming a conical pile whose base radius is exactly two times its height. 
\begin{subproblems}
\item Since you know that the base radius is twice the height, write an equation relating $r$ and $h$. Given that equation, what is the relationship between $\frac{dr}{dt}$ and $\frac{dh}{dt}$?
\vspace{1in}

\item 
How fast is the height of the pile changing when the pile is 3 cm high? Use the formula $V = \frac{1}{3}\pi r^{2}h$ for computing the volume of the cone. 


{\bf Write a complete sentence to answer the question}. Units should be included in your answer. 

\end{subproblems}
\newpage

\problem{8} Consider the function $f(x) = \sqrt{4 - x}$.

\begin{subproblems}
	
\item Find the linearization (linear approximation) $L(x)$ of the function $f(x)$ at $a = 0$.	

\vspace{1.2 in}

\item What is $x$ if $f(x) = \sqrt{3.9}$? Give your answer as a fraction. \hrulefill

\item Use linearization or differentials to {\bf estimate} $\sqrt{3.9}$. Clearly show your work.

\vfill
	
	
	\end{subproblems}

\problem{8} Let $f(x)=(4-x^2)^2.$
	\begin{subproblems}
	\item Find all critical points for $f(x).$ Show your work.
	\vfill
	\item Determine the absolute maximum and absolute minimum of $f(x)$ on the interval $[0,3]$ or state that none exist. You must show your work to receive full credit. See the answer-blank below. 
	\vfill
 \textbf{maximum value of $f(x)$} for $x$ in [0,3]: \hrulefill	
 
 \textbf{$x$-value(s)} where the maximum value of $f(x)$ occurs: \hrulefill
	
 \textbf{minimum value} of $f(x)$ for $x$ in [0,3]: \hrulefill
 
  \textbf{$x$-value(s)} where the minimum value of $f(x)$ occurs: \hrulefill

	\end{subproblems}

\end{document}

