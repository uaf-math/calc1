
\documentclass[12pt]{article}

% Layout.
\usepackage[top=1in, bottom=0.75in, left=1in, right=1in, headheight=1in, headsep=6pt]{geometry}

% Fonts.
\usepackage{mathptmx}
\usepackage[scaled=0.86]{helvet}
\renewcommand{\emph}[1]{\textsf{\textbf{#1}}}

% TiKZ.
\usepackage{tikz, pgfplots,mathrsfs}
\usetikzlibrary{calc}
\pgfplotsset{compat = newest}
 
\pgfplotsset{my style/.append style={axis x line=middle, axis y line=
middle, xlabel={$x$}, ylabel={$y$}, axis equal
}}

% Misc packages.
\usepackage{amsmath,amssymb,latexsym}
\usepackage{graphicx}
\usepackage{array}
\usepackage{xcolor}
\usepackage{multicol}

% Commands to set various header/footer components.
\makeatletter
\def\doctitle#1{\gdef\@doctitle{#1}}
\doctitle{Use {\tt\textbackslash doctitle\{MY LABEL\}}.}
\def\docdate#1{\gdef\@docdate{#1}}
\docdate{Use {\tt\textbackslash docdate\{MY DATE\}}.}
\def\doccourse#1{\gdef\@doccourse{#1}}
\let\@doccourse\@empty
\def\docscoring#1{\gdef\@docscoring{#1}}
\let\@docscoring\@empty
\def\docversion#1{\gdef\@docversion{#1}}
\let\@docversion\@empty
\makeatother

% Headers and footers layout.
\makeatletter
\usepackage{fancyhdr}
\pagestyle{fancy}
\fancyhf{} % Clears all headers/footers.
\lhead{\baselineskip 30pt
%\emph{\@doctitle\hfill\@docdate}
\emph{\@docdate\hfill\@doctitle}
\ifnum \value{page} > 1\relax\else\\
\emph{Name: \rule{3.5in}{1pt}\hfill \@docscoring}\fi}
\rfoot{\emph{\@docversion}}
\lfoot{\emph{\@doccourse}}
\cfoot{\emph{\thepage}}
\renewcommand{\headrulewidth}{0pt}%
\makeatother

% Paragraph spacing
\parindent 0pt
\parskip 6pt plus 1pt

% A problem is a section-like command. Use \problem{5} to
% start a problem worth 5 points.
\newcounter{probcount}
\newcounter{subprobcount}
\setcounter{probcount}{0}
\newcommand{\problem}[1]{%
\par
\addvspace{4pt}%
\setcounter{subprobcount}{0}%
\stepcounter{probcount}%
\makebox[0pt][r]{\emph{\arabic{probcount}.}\hskip1ex}\emph{[#1 points]}\hskip1ex}
\newcommand{\thesubproblem}{\emph{\alph{subprobcount}.}}

% Subproblems are an enumerate-like environment with a consistent
% numbering scheme. 
% Use \begin{subproblems}\item...\item...\end{subproblems}
\newenvironment{subproblems}{%
\begin{enumerate}%
\setcounter{enumi}{\value{subprobcount}}%
\renewcommand{\theenumi}{\emph{\alph{enumi}}}}%
{\setcounter{subprobcount}{\value{enumi}}\end{enumerate}}

% Blanks for answers in normal and math mode.
\newcommand{\blank}[1]{\rule{#1}{0.75pt}}
\newcommand{\mblank}[1]{\underline{\hspace{#1}}}
\def\emptybox(#1,#2){\framebox{\parbox[c][#2]{#1}{\rule{0pt}{0pt}}}}

% Misc.
\renewcommand{\d}{\displaystyle}
\newcommand{\ds}{\displaystyle}
\def\bc{\begin{center}}
\def\ec{\end{center}}
\def\be{\begin{enumerate}}
\def\ee{\end{enumerate}}


\doctitle{Math F251X: Quiz 9}
\docdate{November 12, 2023}
\doccourse{UAF Calculus I}
\docversion{v-1 async}
\docscoring{\blank{0.8in} / 25}
\begin{document}
%\textbf{Please circle your instructor's name:} \hfill Leah Berman  \hfill   Jill Faudree\\

There are 25 points possible on this quiz.  {\it You should be able to complete it without using your notes or textbook or a calculator --- this is practice for your exams!} If you needed to look something up, you should to me about questions you might have.  {\bf Show all work for full credit} and use some words or sentences to help communicate your answers.

\begin{minipage}{.6\linewidth}
\problem{8} (optimization) You are constructing an {\bf open-topped} cardboard box. You begin with a sheet of cardboard with dimensions  8 m by 5 m. You cut equal-sized squares from each corner so you can fold up the edges to create the sides of the box. What are the dimensions of the box with the largest volume? (See figure to the right. The gray squares have been cut out, and the dashed lines are the fold lines to construct the box.)
\end{minipage}
%
\hspace{.5cm}
\begin{minipage}{.4\linewidth}
\def\r{1.3}
\begin{tikzpicture}[scale=.6]
\def\xx{8}
\def\yy{5}

\draw[fill = gray!50] (0,0) rectangle (\r, \r);
\draw[fill = gray!50] (\xx,\yy) rectangle (\xx-\r, \yy-\r);
\draw[fill = gray!50] (\xx,0) rectangle (\xx-\r, \r);
\draw[fill = gray!50] (0,\yy) rectangle (\r, \yy-\r);
\draw[ultra thick] (0,0) rectangle (\xx,\yy);
\draw (\xx-\r, \r/2) node[left] {$x$};
\draw (\xx-\r/2, \r) node[above] {$x$};
\draw (\r, \r/2) node[right] {$x$};
\draw (\r/2, \r) node[above] {$x$};
\draw (\r, \yy-\r/2) node[right] {$x$};
\draw (\xx-\r/2, \yy-\r) node[below] {$x$};
\draw (\r/2, \yy-\r) node[below] {$x$};
\draw (\xx-\r, \yy-\r/2) node[left] {$x$};
%\draw[|-|] (4-\r-.2, \r) -- node[midway, fill = white, inner sep = 2] {$x$} (4-\r-.2, .02);
%\draw[|-|] (4-\r, \r+.2) -- node[midway, fill = white, inner sep = 2] {$x$} (4, \r+.2);
%\draw[|-|] (0, \r+.2) -- node[midway, fill = white, inner sep = 2] {$x$} (\r, \r+.2);
%\draw[|-|] (\r + .2, 0) -- node[midway, fill = white, inner sep = 2] {$x$} (\r+.2, \r);
%\draw[|-|] (0 , 2-\r- .2) -- node[midway, fill = white, inner sep = 2] {$x$} (\r, 2-\r-.2);
%\draw[|-|] (4-\r , 2-\r- .2) -- node[midway, fill = white, inner sep = 2] {$x$} (4, 2-\r-.2);
%\draw[thick,|-|] (4-\r-.2 , 2-\r) -- node[midway, fill = white, inner sep = 2] {$x$} (4-\r-.2, 2);
%\draw[|-|] (\r+.2 , 2-\r) -- node[midway, fill = white, inner sep = 2] {$x$} (\r+.2, 2);
\draw[|-|](0,-.3) -- node[midway, fill = white, inner sep = 2] {$\xx$} (\xx, -.3);
\draw[|-|](\xx+.3,0) -- node[midway, fill = white, inner sep = 2] {$\yy$} (\xx+.3, \yy);
\draw[dashed] (\r,\r) rectangle (\xx-\r, \yy-\r);
\end{tikzpicture}
\end{minipage}
\begin{subproblems}
\item Identify the quantity to be maximized or minimized. %(What do you {\bf want}?)
\vspace{0.25in}
%\item What is your constraint? (That is, what do you {\bf know}?)
\item Write the the quantity to be maximized or minimized as a function of $x$.
\vspace{.55in}
\item What is the domain of your function? Use words to explain why that is the domain.
\vspace{.45in}
\item Answer the question and use Calculus to demonstrate that you answer is correct. (That is, you need to show that you have found a minimum or maximum.)
\vfill
{\bf Dimensions} of the largest box are: length = \blank{1.5cm}, width = \blank{1.5cm}, height = \blank{1.5cm}%\quad \textbf{base}$=$ \underline{\hspace{1in}}\quad  \textbf{height}$=$ \underline{\hspace{1in}}
\end{subproblems}

\newpage
\problem{9} Evaluate the following limits. You must show/explain your work to earn full credit. If you apply L'H\^opital's Rule, you must indicate this by writing {\sf L'H} or {\sf H} above the equals sign where you use the rule.
\begin{subproblems}

\item $\displaystyle{\lim_{x\to 0} \frac{4e^x-3x-12}{4x^3}}$
\vfill
\item $\displaystyle{\lim_{x\to 0^+} x \ln(x^4)}$
\vfill
\item $\displaystyle{\lim_{x\to 0} \frac{3x^2-2x}{\sin(x)}}$
\vfill


\end{subproblems}

\problem{8} Evaluate the following indefinite integrals (and give the most general answer). You must show your work to earn full credit. 

\begin{subproblems}
\item $\displaystyle{\int \left(x^{1/3}+\cos(x) + 5e^x\right) \: dx}$
\vfill
\item $\displaystyle{\int \left(\sec(x)\tan(x)+\frac{x+6}{x} \right)\: dx}$
\vfill

\end{subproblems}
\end{document}