
\documentclass[12pt]{article}

% Layout.
\usepackage[top=1in, bottom=0.75in, left=1in, right=1in, headheight=1in, headsep=6pt]{geometry}

% Fonts.
\usepackage{mathptmx}
\usepackage[scaled=0.86]{helvet}
\renewcommand{\emph}[1]{\textsf{\textbf{#1}}}

% TiKZ.
\usepackage{tikz, pgfplots}
\usetikzlibrary{calc}
\pgfplotsset{compat = newest}
 
\pgfplotsset{my style/.append style={axis x line=middle, axis y line=
middle, xlabel={$x$}, ylabel={$y$}, axis equal
}}

% Misc packages.
\usepackage{amsmath,amssymb,latexsym}
\usepackage{graphicx}
\usepackage{array}
\usepackage{xcolor}
\usepackage{multicol}

% Commands to set various header/footer components.
\makeatletter
\def\doctitle#1{\gdef\@doctitle{#1}}
\doctitle{Use {\tt\textbackslash doctitle\{MY LABEL\}}.}
\def\docdate#1{\gdef\@docdate{#1}}
\docdate{Use {\tt\textbackslash docdate\{MY DATE\}}.}
\def\doccourse#1{\gdef\@doccourse{#1}}
\let\@doccourse\@empty
\def\docscoring#1{\gdef\@docscoring{#1}}
\let\@docscoring\@empty
\def\docversion#1{\gdef\@docversion{#1}}
\let\@docversion\@empty
\makeatother

% Headers and footers layout.
\makeatletter
\usepackage{fancyhdr}
\pagestyle{fancy}
\fancyhf{} % Clears all headers/footers.
\lhead{\baselineskip 30pt
%\emph{\@doctitle\hfill\@docdate}
\emph{\@docdate\hfill\@doctitle}
\ifnum \value{page} > 1\relax\else\\
\emph{Name: \rule{3.5in}{1pt}\hfill \@docscoring}\fi}
\rfoot{\emph{\@docversion}}
\lfoot{\emph{\@doccourse}}
\cfoot{\emph{\thepage}}
\renewcommand{\headrulewidth}{0pt}%
\makeatother

% Paragraph spacing
\parindent 0pt
\parskip 6pt plus 1pt

% A problem is a section-like command. Use \problem{5} to
% start a problem worth 5 points.
\newcounter{probcount}
\newcounter{subprobcount}
\setcounter{probcount}{0}
\newcommand{\problem}[1]{%
\par
\addvspace{4pt}%
\setcounter{subprobcount}{0}%
\stepcounter{probcount}%
\makebox[0pt][r]{\emph{\arabic{probcount}.}\hskip1ex}\emph{[#1 points]}\hskip1ex}
\newcommand{\thesubproblem}{\emph{\alph{subprobcount}.}}

% Subproblems are an enumerate-like environment with a consistent
% numbering scheme. 
% Use \begin{subproblems}\item...\item...\end{subproblems}
\newenvironment{subproblems}{%
\begin{enumerate}%
\setcounter{enumi}{\value{subprobcount}}%
\renewcommand{\theenumi}{\emph{\alph{enumi}}}}%
{\setcounter{subprobcount}{\value{enumi}}\end{enumerate}}

% Blanks for answers in normal and math mode.
\newcommand{\blank}[1]{\rule{#1}{0.75pt}}
\newcommand{\mblank}[1]{\underline{\hspace{#1}}}
\def\emptybox(#1,#2){\framebox{\parbox[c][#2]{#1}{\rule{0pt}{0pt}}}}

% Misc.
\renewcommand{\d}{\displaystyle}
\newcommand{\ds}{\displaystyle}
\def\bc{\begin{center}}
\def\ec{\end{center}}
\def\be{\begin{enumerate}}
\def\ee{\end{enumerate}}


\doctitle{Math 251: Quiz 5}
\docdate{October 8, 2023}
\doccourse{UAF Calculus I}
\docversion{v-1 async}
\docscoring{\blank{0.8in} / 25}
\begin{document}
%\textbf{Please circle your instructor's name:} \hfill Leah Berman  \hfill   Jill Faudree\\

There are 25 points possible on this quiz.  {\it You should be able to complete it without using your notes or textbook or a calculator --- this is practice for your exams!} If you needed to look something up, you should to me about questions you might have.  {\bf Show all work for full credit} and use some words or sentences to help communicate your answers.

\problem{15} Find the derivative for each function below. \fbox{You do not need to simplify.} You do need to use parentheses correctly.
\begin{subproblems}
	\item $f(x)=\csc(x) + \tan\left(\frac{\pi}{6}\right)$
	\vfill
	\item $y=\sec(6x^{3})$
	\vfill
	\item $g(\theta)=\theta^5\big(\cot(t)\big)^{4}$
	\vfill
	\item $h(t)=\big(\sin\left(\frac{\pi}{2} t\right)\big)^{5}$
	\vfill
	\item $y=\sqrt[3]{\tan\left(\frac{x}{5}\right) - 4x}$
	\vfill
\end{subproblems}
\newpage
\problem{5} Find $f''(x)$ for the function $f(x)=\sin(5x^{1/3}).$ You do not need to simplify your final answer.
\vfill

\vfill

\problem{5} Let $g(x)=(x^2 - 6 x)^3.$
\begin{subproblems}
	\item Find $g'(x).$
\vspace{2cm}
	\item Find all $x$-values where the graph of $g(x)$ has a horizontal tangent. Show your work, and make it clear what you are calculating.
	\vfill
	
		
	\vfill
	The function $g(x)$ has a horizontal tangent at $x = $ \hrulefill.
\end{subproblems}

\end{document}