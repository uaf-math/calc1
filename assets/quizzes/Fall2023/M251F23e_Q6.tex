
\documentclass[12pt]{article}

% Layout.
\usepackage[top=1in, bottom=0.75in, left=1in, right=1in, headheight=1in, headsep=6pt]{geometry}

% Fonts.
\usepackage{mathptmx}
\usepackage[scaled=0.86]{helvet}
\renewcommand{\emph}[1]{\textsf{\textbf{#1}}}

% TiKZ.
\usepackage{tikz, pgfplots}
\usetikzlibrary{calc}
\pgfplotsset{compat = newest}
 
\pgfplotsset{my style/.append style={axis x line=middle, axis y line=
middle, xlabel={$x$}, ylabel={$y$}, axis equal
}}

% Misc packages.
\usepackage{amsmath,amssymb,latexsym}
\usepackage{graphicx}
\usepackage{array}
\usepackage{xcolor}
\usepackage{multicol}

% Commands to set various header/footer components.
\makeatletter
\def\doctitle#1{\gdef\@doctitle{#1}}
\doctitle{Use {\tt\textbackslash doctitle\{MY LABEL\}}.}
\def\docdate#1{\gdef\@docdate{#1}}
\docdate{Use {\tt\textbackslash docdate\{MY DATE\}}.}
\def\doccourse#1{\gdef\@doccourse{#1}}
\let\@doccourse\@empty
\def\docscoring#1{\gdef\@docscoring{#1}}
\let\@docscoring\@empty
\def\docversion#1{\gdef\@docversion{#1}}
\let\@docversion\@empty
\makeatother

% Headers and footers layout.
\makeatletter
\usepackage{fancyhdr}
\pagestyle{fancy}
\fancyhf{} % Clears all headers/footers.
\lhead{\baselineskip 30pt
%\emph{\@doctitle\hfill\@docdate}
\emph{\@docdate\hfill\@doctitle}
\ifnum \value{page} > 1\relax\else\\
\emph{Name: \rule{3.5in}{1pt}\hfill \@docscoring}\fi}
\rfoot{\emph{\@docversion}}
\lfoot{\emph{\@doccourse}}
\cfoot{\emph{\thepage}}
\renewcommand{\headrulewidth}{0pt}%
\makeatother

% Paragraph spacing
\parindent 0pt
\parskip 6pt plus 1pt

% A problem is a section-like command. Use \problem{5} to
% start a problem worth 5 points.
\newcounter{probcount}
\newcounter{subprobcount}
\setcounter{probcount}{0}
\newcommand{\problem}[1]{%
\par
\addvspace{4pt}%
\setcounter{subprobcount}{0}%
\stepcounter{probcount}%
\makebox[0pt][r]{\emph{\arabic{probcount}.}\hskip1ex}\emph{[#1 points]}\hskip1ex}
\newcommand{\thesubproblem}{\emph{\alph{subprobcount}.}}

% Subproblems are an enumerate-like environment with a consistent
% numbering scheme. 
% Use \begin{subproblems}\item...\item...\end{subproblems}
\newenvironment{subproblems}{%
\begin{enumerate}%
\setcounter{enumi}{\value{subprobcount}}%
\renewcommand{\theenumi}{\emph{\alph{enumi}}}}%
{\setcounter{subprobcount}{\value{enumi}}\end{enumerate}}

% Blanks for answers in normal and math mode.
\newcommand{\blank}[1]{\rule{#1}{0.75pt}}
\newcommand{\mblank}[1]{\underline{\hspace{#1}}}
\def\emptybox(#1,#2){\framebox{\parbox[c][#2]{#1}{\rule{0pt}{0pt}}}}

% Misc.
\renewcommand{\d}{\displaystyle}
\newcommand{\ds}{\displaystyle}
\def\bc{\begin{center}}
\def\ec{\end{center}}
\def\be{\begin{enumerate}}
\def\ee{\end{enumerate}}


\doctitle{Math 251X: Quiz 6}
\docdate{October 15, 2023}
\doccourse{UAF Calculus I}
\docversion{v-1 async}
\docscoring{\blank{0.8in} / 25}
\begin{document}
%\textbf{Please circle your instructor's name:} \hfill Leah Berman  \hfill   Jill Faudree\\

There are 25 points possible on this quiz.{\it You should be able to complete it without using your notes or textbook -- this is practice for your exams!} If you needed to look something up, you should talk to me about questions you might have.  {\bf Show all work for full credit} and use some words or sentences to help communicate your answers. {\bf Do not use a calculator.} No aids (book, calculator, etc.)
are permitted. 

\problem{8} Find the equation of the tangent line to the implicitly defined function \[ x^2 - (y + 4)^3 = x y \]
at the point $P = (-4,-6)$ and sketch the tangent line on the graph. Clearly show your work.

\includegraphics[width = 2in]{impPlot2.pdf}

\vspace{2cm}

Equation of tangent line: \hrulefill
\problem{5} Use {\bf logarithmic differentiation} to find the derivative of \[\ds{f(x) = (x^{2} - 4x)^{3x} }.\] 

\vfill

\newpage

\problem{12} Find the derivative for each function below. Use whatever technique you like. \fbox{Do not simplify}. You do need to use parentheses correctly.

\begin{subproblems}

\item $h(x)=\frac{1}{x}+\ln(x) $
\vfill
\item $f(x)=\arcsin(7x)\left(\frac{1}{x^{3}}\right)$
\vfill
\item $y=\left(2^x + \arctan(x)\right)^5$
\vfill
	
\item $g(x)=\ds{\frac{x^5-2}{e^{7x+6}}} $
\vfill


\end{subproblems}



%\problem{15} Find the derivative for each function below. \fbox{Do not simplify}. You do need to use parentheses correctly.
%\begin{subproblems}
%	\item $h(x)=\frac{1}{x}+\log_2(x)$
%	\vfill
%	\item $f(x)=\sin^{-1}(\sqrt[3]{x})$
%	\vfill
%	\item $y=\left(2^x + \tan^{-1}(x)\right)^3$
%	\vfill
%	
%	\item $g(x)=\frac{x^2+7}{e^{5x+3}} $
%	\vfill
%	
%	\item $y=5x^{4/3}+\ln(5x^{4/3})$
%	\vfill
%\end{subproblems}
%\newpage
%\problem{5} Use implicit differentiation to find $\frac{dy}{dx}$ for $x^2+y^2=\sin(xy)+8.$ 
%\vfill
%\problem{5} Use logarithmic differentiation to find $\frac{dy}{dx}$ for $y=(\cos(3x))^{x}.$
%\vfill

\end{document}