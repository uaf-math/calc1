
\documentclass[12pt]{article}

% Layout.
\usepackage[top=1in, bottom=0.75in, left=1in, right=1in, headheight=1in, headsep=6pt]{geometry}

% Fonts.
\usepackage{mathptmx}
\usepackage[scaled=0.86]{helvet}
\renewcommand{\emph}[1]{\textsf{\textbf{#1}}}

% TiKZ.
\usepackage{tikz, pgfplots}
\usetikzlibrary{calc}
\pgfplotsset{compat = newest}
 
\pgfplotsset{my style/.append style={axis x line=middle, axis y line=
middle, xlabel={$x$}, ylabel={$y$}, axis equal
}}

% Misc packages.
\usepackage{amsmath,amssymb,latexsym}
\usepackage{graphicx}
\usepackage{array}
\usepackage{xcolor}
\usepackage{multicol}

% Commands to set various header/footer components.
\makeatletter
\def\doctitle#1{\gdef\@doctitle{#1}}
\doctitle{Use {\tt\textbackslash doctitle\{MY LABEL\}}.}
\def\docdate#1{\gdef\@docdate{#1}}
\docdate{Use {\tt\textbackslash docdate\{MY DATE\}}.}
\def\doccourse#1{\gdef\@doccourse{#1}}
\let\@doccourse\@empty
\def\docscoring#1{\gdef\@docscoring{#1}}
\let\@docscoring\@empty
\def\docversion#1{\gdef\@docversion{#1}}
\let\@docversion\@empty
\makeatother

% Headers and footers layout.
\makeatletter
\usepackage{fancyhdr}
\pagestyle{fancy}
\fancyhf{} % Clears all headers/footers.
\lhead{\baselineskip 30pt
%\emph{\@doctitle\hfill\@docdate}
\emph{\@docdate\hfill\@doctitle}
\ifnum \value{page} > 1\relax\else\\
\emph{Name: \rule{3.5in}{1pt}\hfill \@docscoring}\fi}
\rfoot{\emph{\@docversion}}
\lfoot{\emph{\@doccourse}}
\cfoot{\emph{\thepage}}
\renewcommand{\headrulewidth}{0pt}%
\makeatother

% Paragraph spacing
\parindent 0pt
\parskip 6pt plus 1pt

% A problem is a section-like command. Use \problem{5} to
% start a problem worth 5 points.
\newcounter{probcount}
\newcounter{subprobcount}
\setcounter{probcount}{0}
\newcommand{\problem}[1]{%
\par
\addvspace{4pt}%
\setcounter{subprobcount}{0}%
\stepcounter{probcount}%
\makebox[0pt][r]{\emph{\arabic{probcount}.}\hskip1ex}\emph{[#1 points]}\hskip1ex}
\newcommand{\thesubproblem}{\emph{\alph{subprobcount}.}}

% Subproblems are an enumerate-like environment with a consistent
% numbering scheme. 
% Use \begin{subproblems}\item...\item...\end{subproblems}
\newenvironment{subproblems}{%
\begin{enumerate}%
\setcounter{enumi}{\value{subprobcount}}%
\renewcommand{\theenumi}{\emph{\alph{enumi}}}}%
{\setcounter{subprobcount}{\value{enumi}}\end{enumerate}}

% Blanks for answers in normal and math mode.
\newcommand{\blank}[1]{\rule{#1}{0.75pt}}
\newcommand{\mblank}[1]{\underline{\hspace{#1}}}
\def\emptybox(#1,#2){\framebox{\parbox[c][#2]{#1}{\rule{0pt}{0pt}}}}

% Misc.
\renewcommand{\d}{\displaystyle}
\newcommand{\ds}{\displaystyle}
\def\bc{\begin{center}}
\def\ec{\end{center}}
\def\be{\begin{enumerate}}
\def\ee{\end{enumerate}}


\doctitle{Math 251: Quiz 7}
\docdate{Mar 21, 2024}
\doccourse{UAF Calculus I}
\docversion{v-1}
\docscoring{\blank{0.8in} / 25}
\begin{document}

There are 25 points possible on this quiz. No aids (book, calculator, etc.)
are permitted.  {\bf Show all work for full credit.}

\problem{10} (Related Rate Problem) A spherical snowball is melting so that its volume is decreasing at a constant rate of $4\pi \: cm^3/min$. Use this information to answer the following questions.
\begin{subproblems}
\item How fast is the {\bf radius} of the snowball decreasing when the radius is $8 \: cm$? Include units in your answer. (Use the fact that the volume of a sphere is given by $\ds V=\frac{4}{3}\pi r^3$.)
\vfill
\item How fast is the {\bf surface area} of the snowball decreasing when the radius is $8 \: cm$? Include units in your answer. (Use your answer in part (a) and that the surface area of a sphere is given by $\ds S=4\pi r^2$.)
\vfill

\end{subproblems}


\newpage

\problem{7} (Linear Approximation and Differentials) Let $f(x)= x^3-\ln(x).$
	\begin{subproblems}
	\item Find the linear approximation $L(x)=f(a)+f'(a)(x-a)$ to $y=f(x)$ at $a=1$.
	\vfill
	\item Use your linear approximation to estimate $\ds f\left(\frac{3}{2}\right)$.
	\vfill
	\end{subproblems}

\problem{8} Let $h(x)= 4x^3-3x^4+6$.
	\begin{subproblems}
	\item Find all critical points for $h(x).$
	\vfill
	\item Determine the absolute maximum and absolute minimum of $h(x)$ on the interval $[-1,2]$ or state that none exist. You must show your work to receive full credit. See the answer-blank below. 
	\vfill
	\hfill \textbf{maximum value of $h(x)$:} \underline{\hspace{1in}}\\
	
	\hfill \textbf{minimum value of $h(x)$:} \underline{\hspace{1in}}\\
	\end{subproblems}

\end{document}

