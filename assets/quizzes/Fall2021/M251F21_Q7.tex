
\documentclass[12pt]{article}

% Layout.
\usepackage[top=1in, bottom=0.75in, left=1in, right=1in, headheight=1in, headsep=6pt]{geometry}

% Fonts.
\usepackage{mathptmx}
\usepackage[scaled=0.86]{helvet}
\renewcommand{\emph}[1]{\textsf{\textbf{#1}}}

% TiKZ.
\usepackage{tikz, pgfplots}
\usetikzlibrary{calc}
\pgfplotsset{compat = newest}
 
\pgfplotsset{my style/.append style={axis x line=middle, axis y line=
middle, xlabel={$x$}, ylabel={$y$}, axis equal }}

% Misc packages.
\usepackage{amsmath,amssymb,latexsym}
\usepackage{graphicx}
\usepackage{array}
\usepackage{xcolor}
\usepackage{multicol}

% Commands to set various header/footer components.
\makeatletter
\def\doctitle#1{\gdef\@doctitle{#1}}
\doctitle{Use {\tt\textbackslash doctitle\{MY LABEL\}}.}
\def\docdate#1{\gdef\@docdate{#1}}
\docdate{Use {\tt\textbackslash docdate\{MY DATE\}}.}
\def\doccourse#1{\gdef\@doccourse{#1}}
\let\@doccourse\@empty
\def\docscoring#1{\gdef\@docscoring{#1}}
\let\@docscoring\@empty
\def\docversion#1{\gdef\@docversion{#1}}
\let\@docversion\@empty
\makeatother

% Headers and footers layout.
\makeatletter
\usepackage{fancyhdr}
\pagestyle{fancy}
\fancyhf{} % Clears all headers/footers.
\lhead{\baselineskip 30pt
%\emph{\@doctitle\hfill\@docdate}
\emph{\@docdate\hfill\@doctitle}
\ifnum \value{page} > 1\relax\else\\
\emph{Name: \rule{3.5in}{1pt}\hfill \@docscoring}\fi}
\rfoot{\emph{\@docversion}}
\lfoot{\emph{\@doccourse}}
\cfoot{\emph{\thepage}}
\renewcommand{\headrulewidth}{0pt}%
\makeatother

% Paragraph spacing
\parindent 0pt
\parskip 6pt plus 1pt

% A problem is a section-like command. Use \problem{5} to
% start a problem worth 5 points.
\newcounter{probcount}
\newcounter{subprobcount}
\setcounter{probcount}{0}
\newcommand{\problem}[1]{%
\par
\addvspace{4pt}%
\setcounter{subprobcount}{0}%
\stepcounter{probcount}%
\makebox[0pt][r]{\emph{\arabic{probcount}.}\hskip1ex}\emph{[#1 points]}\hskip1ex}
\newcommand{\thesubproblem}{\emph{\alph{subprobcount}.}}

% Subproblems are an enumerate-like environment with a consistent
% numbering scheme. 
% Use \begin{subproblems}\item...\item...\end{subproblems}
\newenvironment{subproblems}{%
\begin{enumerate}%
\setcounter{enumi}{\value{subprobcount}}%
\renewcommand{\theenumi}{\emph{\alph{enumi}}}}%
{\setcounter{subprobcount}{\value{enumi}}\end{enumerate}}

% Blanks for answers in normal and math mode.
\newcommand{\blank}[1]{\rule{#1}{0.75pt}}
\newcommand{\mblank}[1]{\underline{\hspace{#1}}}
\def\emptybox(#1,#2){\framebox{\parbox[c][#2]{#1}{\rule{0pt}{0pt}}}}

% Misc.
\renewcommand{\d}{\displaystyle}
\newcommand{\ds}{\displaystyle}
\def\bc{\begin{center}}
\def\ec{\end{center}}
\def\be{\begin{enumerate}}
\def\ee{\end{enumerate}}


\doctitle{Math 251: Quiz 7}
\docdate{October 19, 2021}
\doccourse{UAF Calculus I}
\docversion{v-1}
\docscoring{\blank{0.8in} / 25}
\begin{document}

There are 25 points possible on this quiz. No aids (book, calculator, etc.)
are permitted.  {\emph{Show all work for full credit.}}

\problem{8} Follow the steps below to solve a related rates problem.  \begin{subproblems}
	\item Assume the base, $b$, of a triangle is growing at a rate of $2$ feet per minute and the height, $h,$ of the triangle is shrinking at a rate of 4 feet per minute when the base is 10 feet long and height is 15 feet long. Using this information, identify values for $h$, $b$, $dh/dt$ and $db/dt$.
	\vfill
	\item The area of a triangle is given by the formula $\large{A=\frac{1}{2}bh}$ where $b$ is the length of the base of the triangle and $h$ is its height. Take the derivative of the above equation implicitly with respect to time.
	\vfill
		\item Use the above information to determine the rate of change of the area of the triangle. Include units.
		\vfill
		\item Is the area increasing or decreasing at this instant? Justify your answer.
		\vspace{1in}
\end{subproblems}
\newpage
\problem{8} Let $f(x)=x^4.$
\begin{subproblems}
	\item Find the linear approximation, $L(x),$ of $f(x)$ at $x=2.$
	\vfill
	\item Use the linear approximation to estimate $(1.8)^4.$ Your answer here must be in the form of a simplified fraction.
	\vfill
\end{subproblems}
\problem{9}  Let $g(x)=3x^4-4x^3.$
\begin{subproblems}
	\item Find all critical points of $g(x).$
	\vfill
	\item Determine the \emph{absolute} minimum and \emph{absolute} maximum of $g(x)$ on the interval $[-1,2].$ Make sure to show your work.
	\vfill
	%\item Does $g(x)$ have an \emph{absolute} maximum on the interval $(-\infty, \infty)$? Justify your answer with a short sentence.
	\vspace{0.5in}
\end{subproblems}
\end{document}