
\documentclass[12pt]{article}

% Layout.
\usepackage[top=1in, bottom=0.75in, left=1in, right=1in, headheight=1in, headsep=6pt]{geometry}

% Fonts.
\usepackage{mathptmx}
\usepackage[scaled=0.86]{helvet}
\renewcommand{\emph}[1]{\textsf{\textbf{#1}}}

% TiKZ.
\usepackage{tikz, pgfplots}
\usetikzlibrary{calc}
\pgfplotsset{compat = newest}
 
\pgfplotsset{my style/.append style={axis x line=middle, axis y line=
middle, xlabel={$x$}, ylabel={$y$}, axis equal }}

% Misc packages.
\usepackage{amsmath,amssymb,latexsym}
\usepackage{graphicx}
\usepackage{array}
\usepackage{xcolor}
\usepackage{multicol}

% Commands to set various header/footer components.
\makeatletter
\def\doctitle#1{\gdef\@doctitle{#1}}
\doctitle{Use {\tt\textbackslash doctitle\{MY LABEL\}}.}
\def\docdate#1{\gdef\@docdate{#1}}
\docdate{Use {\tt\textbackslash docdate\{MY DATE\}}.}
\def\doccourse#1{\gdef\@doccourse{#1}}
\let\@doccourse\@empty
\def\docscoring#1{\gdef\@docscoring{#1}}
\let\@docscoring\@empty
\def\docversion#1{\gdef\@docversion{#1}}
\let\@docversion\@empty
\makeatother

% Headers and footers layout.
\makeatletter
\usepackage{fancyhdr}
\pagestyle{fancy}
\fancyhf{} % Clears all headers/footers.
\lhead{\baselineskip 30pt
%\emph{\@doctitle\hfill\@docdate}
\emph{\@docdate\hfill\@doctitle}
\ifnum \value{page} > 1\relax\else\\
\emph{Name: \rule{3.5in}{1pt}\hfill \@docscoring}\fi}
\rfoot{\emph{\@docversion}}
\lfoot{\emph{\@doccourse}}
\cfoot{\emph{\thepage}}
\renewcommand{\headrulewidth}{0pt}%
\makeatother

% Paragraph spacing
\parindent 0pt
\parskip 6pt plus 1pt

% A problem is a section-like command. Use \problem{5} to
% start a problem worth 5 points.
\newcounter{probcount}
\newcounter{subprobcount}
\setcounter{probcount}{0}
\newcommand{\problem}[1]{%
\par
\addvspace{4pt}%
\setcounter{subprobcount}{0}%
\stepcounter{probcount}%
\makebox[0pt][r]{\emph{\arabic{probcount}.}\hskip1ex}\emph{[#1 points]}\hskip1ex}
\newcommand{\thesubproblem}{\emph{\alph{subprobcount}.}}

% Subproblems are an enumerate-like environment with a consistent
% numbering scheme. 
% Use \begin{subproblems}\item...\item...\end{subproblems}
\newenvironment{subproblems}{%
\begin{enumerate}%
\setcounter{enumi}{\value{subprobcount}}%
\renewcommand{\theenumi}{\emph{\alph{enumi}}}}%
{\setcounter{subprobcount}{\value{enumi}}\end{enumerate}}

% Blanks for answers in normal and math mode.
\newcommand{\blank}[1]{\rule{#1}{0.75pt}}
\newcommand{\mblank}[1]{\underline{\hspace{#1}}}
\def\emptybox(#1,#2){\framebox{\parbox[c][#2]{#1}{\rule{0pt}{0pt}}}}

% Misc.
\renewcommand{\d}{\displaystyle}
\newcommand{\ds}{\displaystyle}
\def\bc{\begin{center}}
\def\ec{\end{center}}
\def\be{\begin{enumerate}}
\def\ee{\end{enumerate}}


\doctitle{Math 251: Quiz 2}
\docdate{Jan 26, 2023}
\doccourse{UAF Calculus I}
\docversion{v-1}
\docscoring{\blank{0.8in} / 25}
\begin{document}
%\textbf{Please circle your instructor's name:} \hfill Leah Berman  \hfill   Jill Faudree\\

There are 25 points possible on this quiz. No aids (book, calculator, etc.)
are permitted.  {\bf Show all work for full credit.}

\problem{6} Evaluate the trigonometric functions below. Assume all angles are in radians.
\begin{subproblems}    
 \item  {\large{$\sin(\pi/2)=$}} 
 \vfill
 \item  {\large{$\cos(2\pi/3)=$}} 
 \vfill
 \item {\large{$\tan(3 \pi / 4)=$}} 
 \vfill
 \end{subproblems}

\problem{4}  Solve the equation $\: \large{\sin(x) = 0} \:.$ Give the most complete solution.
\vspace{3in}

\newpage
\problem{15} The height of an object in meters is given by the equation $h(t)=10t-t^2$ where $t$ is measured in seconds. The graph of $h(t)$ is provided below.\\

\begin{tikzpicture}[yscale=0.2]
  \draw[->] (-1, 0) -- (11, 0) node[right] {$t$};
  \draw[->] (0, -1) -- (0, 26.5) node[above] {$h(t)$};
  \draw[<->, domain=-0.2:10.2, smooth, variable=\x, ultra thick] plot ({\x}, {10*\x-\x*\x});
  \foreach \i in {1,2,...,10}{
  	\draw (\i,-.5)--(\i,0.5);
	\node at (\i,-1.3){\i};
	}
\draw (-.1,25)--(0.1,25);
\node at (-0.5,25){25};
\end{tikzpicture}

\begin{subproblems}
\item Find $h(2)$ and explain (using a complete sentence) what this number represents in the context of the problem. Include units. 
\vfill

\item Find the average velocity of the object over the time interval from $t=2$ to $t=6.$ Include units with your answer.

\vfill

\item On the graph above, draw and label the secant line between the points $P(2,h(2))$ and $Q(6,h(6)).$  (By \emph{label}, we mean label with the word \textbf{secant}.) 

\item On the graph above, draw and label the tangent line at the point $P(2,h(2)).$ (By \emph{label}, we mean label with the word \textbf{tangent}.) 

\item \emph{Based on the graph and the lines you drew in parts c and d,} do you expect the slope of the tangent line to $h(x)$ at $P$ to be larger than, equal to, or smaller than the slope of the secant line between points $P$ and $Q$? Explain your reasoning using complete sentences.\\
\vfill
\end{subproblems}
	
\end{document}