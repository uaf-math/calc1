\documentclass[12pt]{article}

% Layout.
\usepackage[top=1.2in, bottom=0.75in, left=1in, right=1in, headheight=1.0in, headsep=0pt]{geometry}

% Fonts.
\usepackage{mathptmx}
\usepackage[scaled=0.86]{helvet}
\renewcommand{\emph}[1]{\textsf{\textbf{#1}}}

% TiKZ.
\usepackage{tikz, pgfplots}
\usetikzlibrary{calc}
\pgfplotsset{my style/.append style={axis x line=middle, axis y line=middle, xlabel={$x$}, ylabel={$y$}}}

% Misc packages.
\usepackage{amsmath,amssymb,latexsym}
\usepackage{graphicx}
\usepackage{array}
\usepackage{xcolor}
\usepackage{multicol}

% Commands to set various header/footer components.
\makeatletter
\def\doctitle#1{\gdef\@doctitle{#1}}
\doctitle{Use {\tt\textbackslash doctitle\{MY LABEL\}}.}
\def\docdate#1{\gdef\@docdate{#1}}
\docdate{Use {\tt\textbackslash docdate\{MY DATE\}}.}
\def\doccourse#1{\gdef\@doccourse{#1}}
\let\@doccourse\@empty
\def\docscoring#1{\gdef\@docscoring{#1}}
\let\@docscoring\@empty
\def\docversion#1{\gdef\@docversion{#1}}
\let\@docversion\@empty
\makeatother

% Headers and footers layout.
\makeatletter
\usepackage{fancyhdr}
\pagestyle{fancy}
\fancyhf{} % Clears all headers/footers.
\lhead{\emph{\@doctitle\hfill\@docdate}
\ifnum \value{page} > 1\relax\else\\
\emph{Name: \rule{3.5in}{1pt}\hfill \@docscoring}
\\
\emph{Circle one: \quad Faudree (F01) \hskip 1ex\rule{1pt}{9pt}\hskip 1ex Bueler (F02) \hskip 1ex\rule{1pt}{9pt}\hskip 1ex VanSpronsen (UX1)}
\fi}

\rfoot{\emph{\@docversion}}
\lfoot{\emph{\@doccourse}}
\cfoot{\emph{\thepage}}
\renewcommand{\headrulewidth}{0pt}%
\makeatother

% Paragraph spacing
\parindent 0pt
\parskip 6pt plus 1pt

% A problem is a section-like command. Use \problem{5} for a problem worth 5 points.
\newcounter{probcount}
\newcounter{subprobcount}
\setcounter{probcount}{0}
\newcommand{\problem}[1]{%
\par
\addvspace{4pt}%
\setcounter{subprobcount}{0}%
\stepcounter{probcount}%
\makebox[0pt][r]{\emph{\arabic{probcount}.}\hskip1ex}\emph{[#1 points]}\hskip1ex}
\newcommand{\thesubproblem}{\emph{\alph{subprobcount}.}}

% Subproblems are an enumerate-like environment with a consistent
% numbering scheme. 
% Use \begin{subproblems}\item...\item...\end{subproblems}
\newenvironment{subproblems}{%
\begin{enumerate}%
\setcounter{enumi}{\value{subprobcount}}%
\renewcommand{\theenumi}{\emph{\alph{enumi}}}}%
{\setcounter{subprobcount}{\value{enumi}}\end{enumerate}}

% Blanks for answers in normal and math mode.
\newcommand{\blank}[1]{\rule{#1}{0.75pt}}
\newcommand{\mblank}[1]{\underline{\hspace{#1}}}
\def\emptybox(#1,#2){\framebox{\parbox[c][#2]{#1}{\rule{0pt}{0pt}}}}

% Misc.
\renewcommand{\d}{\displaystyle}
\newcommand{\ds}{\displaystyle}


\doctitle{Math 251: Quiz 7}
\docdate{31 March, 2020}
\doccourse{UAF Calculus I}
\docversion{v-1}
\docscoring{{\LARGE \strut}\blank{0.8in} / 10}

\begin{document}
10 points possible.  \textcolor{red}{\textbf{No aids (internet, other students, book, calculator, etc.) are permitted.} } You do not need to simplify final answers, but \textcolor{red}{\textbf{answers without supporting work will lose points for completeness and effort.}}

% 4.1 question
\problem{3}  Find the absolute maximum and absolute minimum values of $f$ on the given interval.  State the answer as points; give both the $x$- and $y$-coordinates of the extrema.
    $$f(x)=1+12x-x^3, \qquad [0,3] \hspace{4.0in}$$
\vspace{3.0in}

% critical numbers from 4.1, infl. pts. from 4.3
\problem{2}  Consider the function \, $g(t) = t e^{-t^2}$.
\begin{subproblems}
% 3pts
\item Find all of the critical numbers. 
\vfill

% 3pts
\item Find the $x$-coordinate all of the inflection points.
\vfill
\end{subproblems}

\newpage
% based on 4.3 #29, but simplified
\problem{3}  Sketch a graph that satisfies all of the conditions:
\begin{align*}
  &\text{domain } f = (-\infty,\infty), \quad f(0)=0,  \hspace{4.0in}\\
  &f'(3) = 0, \quad f'(x)<0 \text{ when } x<3, \\
  &f'(x)>0 \text{ when } x > 3, \\
  &f''(1)=0, \quad f''(5)=0, \\
  &f''(x)<0 \text{ when $x<1$ or $x>5$}, \\
  &f''(x)>0 \text{ for } 1<x<5
\end{align*}

\vspace{-40mm}

\hfill \begin{tikzpicture}
\begin{axis}[scale = 1.4, axis x line=middle, axis y line=
middle, xlabel={$x$}, ylabel={$y$}, xtick={-2,-1,0,1,2,3,4,5,6,7}, ytick={-3,-2,-1,0,1,2,3,4}, xmin=-3, xmax=8, ymin=-4, ymax=5]
\end{axis}
\end{tikzpicture}

\bigskip

% based on 4.3 #16, but simplified
\problem{2}  Consider the function $f(x) = x \ln x$.
\begin{subproblems}
% 3pts
\item What is the domain of $f$?
\vspace{1.2in}

% 3pts
\item Find the intervals of increase and decrease
\vfill
\end{subproblems}
\end{document}