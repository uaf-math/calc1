\documentclass[12pt]{article}

% Layout.
\usepackage[top=1.2in, bottom=0.75in, left=1in, right=1in, headheight=1.0in, headsep=0pt]{geometry}

% Fonts.
\usepackage{mathptmx}
\usepackage[scaled=0.86]{helvet}
\renewcommand{\emph}[1]{\textsf{\textbf{#1}}}

% TiKZ.
\usepackage{tikz, pgfplots}
\usetikzlibrary{calc}
\pgfplotsset{my style/.append style={axis x line=middle, axis y line=middle, xlabel={$x$}, ylabel={$y$}}}

% Misc packages.
\usepackage{amsmath,amssymb,latexsym}
\usepackage{graphicx}
\usepackage{array}
\usepackage{xcolor}
\usepackage{multicol}

% Commands to set various header/footer components.
\makeatletter
\def\doctitle#1{\gdef\@doctitle{#1}}
\doctitle{Use {\tt\textbackslash doctitle\{MY LABEL\}}.}
\def\docdate#1{\gdef\@docdate{#1}}
\docdate{Use {\tt\textbackslash docdate\{MY DATE\}}.}
\def\doccourse#1{\gdef\@doccourse{#1}}
\let\@doccourse\@empty
\def\docscoring#1{\gdef\@docscoring{#1}}
\let\@docscoring\@empty
\def\docversion#1{\gdef\@docversion{#1}}
\let\@docversion\@empty
\makeatother

% Headers and footers layout.
\makeatletter
\usepackage{fancyhdr}
\pagestyle{fancy}
\fancyhf{} % Clears all headers/footers.
\lhead{\emph{\@doctitle\hfill\@docdate}
\ifnum \value{page} > 1\relax\else\\
\emph{Name: \rule{3.5in}{1pt}\hfill \@docscoring}
\\
\emph{Circle one: \quad Faudree (F01) \hskip 1ex\rule{1pt}{9pt}\hskip 1ex Bueler (F02) \hskip 1ex\rule{1pt}{9pt}\hskip 1ex VanSpronsen (UX1)}
\fi}

\rfoot{\emph{\@docversion}}
\lfoot{\emph{\@doccourse}}
\cfoot{\emph{\thepage}}
\renewcommand{\headrulewidth}{0pt}%
\makeatother

% Paragraph spacing
\parindent 0pt
\parskip 6pt plus 1pt

% A problem is a section-like command. Use \problem{5} for a problem worth 5 points.
\newcounter{probcount}
\newcounter{subprobcount}
\setcounter{probcount}{0}
\newcommand{\problem}[1]{%
\par
\addvspace{4pt}%
\setcounter{subprobcount}{0}%
\stepcounter{probcount}%
\makebox[0pt][r]{\emph{\arabic{probcount}.}\hskip1ex}\emph{[#1 points]}\hskip1ex}
\newcommand{\thesubproblem}{\emph{\alph{subprobcount}.}}

% Subproblems are an enumerate-like environment with a consistent
% numbering scheme. 
% Use \begin{subproblems}\item...\item...\end{subproblems}
\newenvironment{subproblems}{%
\begin{enumerate}%
\setcounter{enumi}{\value{subprobcount}}%
\renewcommand{\theenumi}{\emph{\alph{enumi}}}}%
{\setcounter{subprobcount}{\value{enumi}}\end{enumerate}}

% Blanks for answers in normal and math mode.
\newcommand{\blank}[1]{\rule{#1}{0.75pt}}
\newcommand{\mblank}[1]{\underline{\hspace{#1}}}
\def\emptybox(#1,#2){\framebox{\parbox[c][#2]{#1}{\rule{0pt}{0pt}}}}

% Misc.
\renewcommand{\d}{\displaystyle}
\newcommand{\ds}{\displaystyle}


\doctitle{Math 251: Quiz 8 \hspace{1in} \textcolor{blue}{OPTIONAL}}
\docdate{DUE: 6 April, 2020 by 5PM}
\doccourse{UAF Calculus I}
\docversion{v-1}
\docscoring{{\LARGE \strut}\blank{0.8in} / 10}

\begin{document}

This OPTIONAL Quiz is worth 10 points.  The purpose is to get some additional practice and additional feedback prior to Midterm 2. \textcolor{red}{\textbf{No aids (internet, other students, book, calculator, etc.) are permitted.} } You do not need to simplify final answers, but \textcolor{red}{\textbf{answers without supporting work will lose points for completeness and effort.}}

% 4.5 question
%like 19-54
\problem{4}  Answer the questions below for the function $f(x)=e^{2x}-4e^x+1.$
\begin{subproblems}
\item Evaluate $\displaystyle{\lim_{x \to -\infty}f(x)}$ and explain what this implies about the graph of $f(x).$
\vspace{0.5in}
\item Determine the intervals of increase or decrease and identify the $x$-values of any local extrema. State whether they are maxima or minima.)
\vfill
\item Determine the concavity of the graph of $f$ and find the $x$-values of any points of inflection.
\vfill
\item Use the information about to sketch the graph of $f.$  Your graph should give the coordinates of at least two points.
\vfill
\end{subproblems}
\newpage
% 4.4 question
%like #39 and #60
\problem{2}  Find the limit. Use l'Hospital's Rule where appropriate. 
\begin{subproblems}
% 3pts
\item $\displaystyle{\lim_{x \to 1} \frac{x^p-1}{2x-2} }$, where $p$ is a fixed constant.
\vspace{1in}
\item $\displaystyle{\lim_{x \to \infty}\, x\, \ln\!\left(1+\frac{5}{x}\right)}$
\vspace{1in}
\end{subproblems}

% based on 4.7 #17
\problem{4}  A rectangular box with square base must have volume 20 cubic meters. Material for the base and sides costs \$ 2 per square meter. Material for the top costs \$6 per cubic meter.  Find the dimensions of the least expensive box.
\vfill

\end{document}