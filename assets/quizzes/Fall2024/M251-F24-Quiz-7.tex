
\documentclass[12pt]{article}

% Layout.
\usepackage[top=1in, bottom=0.75in, left=1in, right=1in, headheight=1in, headsep=6pt]{geometry}

% Fonts.
\usepackage{mathptmx}
\usepackage[scaled=0.86]{helvet}
\renewcommand{\emph}[1]{\textsf{\textbf{#1}}}

% TiKZ.
\usepackage{tikz, pgfplots,tabularx}
\usetikzlibrary{calc}
\pgfplotsset{compat = newest}
 
\pgfplotsset{my style/.append style={axis x line=middle, axis y line=
middle, xlabel={$x$}, ylabel={$y$}, axis equal }}

% Misc packages.
\usepackage{amsmath,amssymb,latexsym}
\usepackage{graphicx}
\usepackage{array}
\usepackage{xcolor}
\usepackage{multicol}
\usepackage{enumerate,tabularx}

% Commands to set various header/footer components.
\makeatletter
\def\doctitle#1{\gdef\@doctitle{#1}}
\doctitle{Use {\tt\textbackslash doctitle\{MY LABEL\}}.}
\def\docdate#1{\gdef\@docdate{#1}}
\docdate{Use {\tt\textbackslash docdate\{MY DATE\}}.}
\def\doccourse#1{\gdef\@doccourse{#1}}
\let\@doccourse\@empty
\def\docscoring#1{\gdef\@docscoring{#1}}
\let\@docscoring\@empty
\def\docversion#1{\gdef\@docversion{#1}}
\let\@docversion\@empty
\makeatother

% Headers and footers layout.
\makeatletter
\usepackage{fancyhdr}
\pagestyle{fancy}
\fancyhf{} % Clears all headers/footers.
\lhead{\baselineskip 30pt
%\emph{\@doctitle\hfill\@docdate}
\emph{\@docdate\hfill\@doctitle}
\ifnum \value{page} > 1\relax\else\\
\emph{Name: \rule{3.5in}{1pt}\hfill \@docscoring}\fi}
\rfoot{\emph{\@docversion}}
\lfoot{\emph{\@doccourse}}
\cfoot{\emph{\thepage}}
\renewcommand{\headrulewidth}{0pt}%
\makeatother

% Paragraph spacing
\parindent 0pt
\parskip 6pt plus 1pt

% A problem is a section-like command. Use \problem{5} to
% start a problem worth 5 points.
\newcounter{probcount}
\newcounter{subprobcount}
\setcounter{probcount}{0}
\newcommand{\problem}[1]{%
\par
\addvspace{4pt}%
\setcounter{subprobcount}{0}%
\stepcounter{probcount}%
\makebox[0pt][r]{\emph{\arabic{probcount}.}\hskip1ex}\emph{[#1 points]}\hskip1ex}
\newcommand{\thesubproblem}{\emph{\alph{subprobcount}.}}

% Subproblems are an enumerate-like environment with a consistent
% numbering scheme. 
% Use \begin{subproblems}\item...\item...\end{subproblems}
\newenvironment{subproblems}{%
\begin{enumerate}%
\setcounter{enumi}{\value{subprobcount}}%
\renewcommand{\theenumi}{\emph{\alph{enumi}}}}%
{\setcounter{subprobcount}{\value{enumi}}\end{enumerate}}

% Blanks for answers in normal and math mode.
\newcommand{\blank}[1]{\rule{#1}{0.5pt}}
\newcommand{\mblank}[1]{\underline{\hspace{#1}}}
\def\emptybox(#1,#2){\framebox{\parbox[c][#2]{#1}{\rule{0pt}{0pt}}}}

% Misc.
\renewcommand{\d}{\displaystyle}
\newcommand{\ds}{\displaystyle}
\def\bc{\begin{center}}
\def\ec{\end{center}}
\def\be{\begin{enumerate}}
\def\ee{\end{enumerate}}


\doctitle{Math F251X: Quiz 7}
\docdate{October 24, 2024}
\doccourse{UAF Calculus I}
\docversion{v-1}
\docscoring{\blank{0.8in} / 25}
\begin{document}
\emph{Please circle your instructor's name:} \hfill Leah Berman  \hfill   Jill Faudree \hfill James Gossell \\

There are 25 points possible on this quiz. Any outside materials (textbook, course notes, calculator) are not allowed.  \emph{For full credit, show all work in a way someone else can follow it.} 

\begin{enumerate}
\item (10 points) Water is draining from the bottom of a cone-shaped funnel at a rate of $0.1$ cubic feet per second. The height of the funnel is $2$ feet and the radius at the top of the funnel is $1$ foot.

\begin{multicols}{2}

\includegraphics[scale=0.8]{cone.jpg}
\columnbreak

Note that the formula for the volume of water in the cone is given by $\displaystyle V=\frac{1}{3}\pi r^2h$.\\

\vspace{.1in}

Note that you can use  similar triangles to find a relationship between $r$ and $h$.\\

\end{multicols}

\begin{enumerate}[a)]

\item \textbf{Find the rate at which the height of the water in the funnel is changing} when the height of the water ($h$) is 1 foot.\\

\vfill
\vfill

\item Using complete sentaces, \textbf{explain what your answer in part (a) means} in the context of this problem. Include units in your explaination.

\vspace{1in}

\end{enumerate}

\newpage

\item (7 points) Complete the following steps to approximate $\sqrt[3]{30}$ without a calculator:
	\begin{enumerate}
	\item Find the linear approximation $L(x)$ to $f(x)=\sqrt[3]{x}$ at $a=27$.
	\vfill
	\item Use $L(x)$ to approximate $\sqrt[3]{30}$. Write your answer as a fraction.
	\vfill
	\end{enumerate}

\item (8 points) Find the absolute maxima and minima for the function $f(x)=x(x-4)^3$ over the interval $[0,5]$. Show your work, including relevant computations.

\vfill
\vfill


	\hfill \textbf{maximum value of $f(x)$:} \underline{\hspace{1in}}\\
	
	\hfill \textbf{minimum value of $f(x)$:} \underline{\hspace{1in}}\\


\end{enumerate}
\end{document}



