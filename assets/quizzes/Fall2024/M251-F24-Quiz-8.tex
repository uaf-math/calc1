
\documentclass[12pt]{article}

% Layout.
\usepackage[top=1in, bottom=0.75in, left=1in, right=1in, headheight=1in, headsep=6pt]{geometry}

% Fonts.
\usepackage{mathptmx}
\usepackage[scaled=0.86]{helvet}
\renewcommand{\emph}[1]{\textsf{\textbf{#1}}}

% TiKZ.
\usepackage{tikz, pgfplots,tabularx}
\usetikzlibrary{calc}
\pgfplotsset{compat = newest}
 
\pgfplotsset{my style/.append style={axis x line=middle, axis y line=
middle, xlabel={$x$}, ylabel={$y$}, axis equal }}

% Misc packages.
\usepackage{amsmath,amssymb,latexsym}
\usepackage{graphicx}
\usepackage{array}
\usepackage{xcolor}
\usepackage{multicol}
\usepackage{enumerate,tabularx}

%%%%%%%%%%%%%%Happy Halloween%%%%%%%%%%%%%%%
\usepackage{halloweenmath}
%%%%%%%%%%%%%%Happy Halloween%%%%%%%%%%%%%%%

% Commands to set various header/footer components.
\makeatletter
\def\doctitle#1{\gdef\@doctitle{#1}}
\doctitle{Use {\tt\textbackslash doctitle\{MY LABEL\}}.}
\def\docdate#1{\gdef\@docdate{#1}}
\docdate{Use {\tt\textbackslash docdate\{MY DATE\}}.}
\def\doccourse#1{\gdef\@doccourse{#1}}
\let\@doccourse\@empty
\def\docscoring#1{\gdef\@docscoring{#1}}
\let\@docscoring\@empty
\def\docversion#1{\gdef\@docversion{#1}}
\let\@docversion\@empty
\makeatother

% Headers and footers layout.
\makeatletter
\usepackage{fancyhdr}
\pagestyle{fancy}
\fancyhf{} % Clears all headers/footers.
\lhead{\baselineskip 30pt
%\emph{\@doctitle\hfill\@docdate}
\emph{\@docdate\hfill\@doctitle}
\ifnum \value{page} > 1\relax\else\\
\emph{Name: \rule{3.5in}{1pt}\hfill \@docscoring}\fi}
\rfoot{\emph{\@docversion}}
\lfoot{\emph{\@doccourse}}
\cfoot{\emph{\thepage}}
\renewcommand{\headrulewidth}{0pt}%
\makeatother

% Paragraph spacing
\parindent 0pt
\parskip 6pt plus 1pt

% A problem is a section-like command. Use \problem{5} to
% start a problem worth 5 points.
\newcounter{probcount}
\newcounter{subprobcount}
\setcounter{probcount}{0}
\newcommand{\problem}[1]{%
\par
\addvspace{4pt}%
\setcounter{subprobcount}{0}%
\stepcounter{probcount}%
\makebox[0pt][r]{\emph{\arabic{probcount}.}\hskip1ex}\emph{[#1 points]}\hskip1ex}
\newcommand{\thesubproblem}{\emph{\alph{subprobcount}.}}

% Subproblems are an enumerate-like environment with a consistent
% numbering scheme. 
% Use \begin{subproblems}\item...\item...\end{subproblems}
\newenvironment{subproblems}{%
\begin{enumerate}%
\setcounter{enumi}{\value{subprobcount}}%
\renewcommand{\theenumi}{\emph{\alph{enumi}}}}%
{\setcounter{subprobcount}{\value{enumi}}\end{enumerate}}

% Blanks for answers in normal and math mode.
\newcommand{\blank}[1]{\rule{#1}{0.5pt}}
\newcommand{\mblank}[1]{\underline{\hspace{#1}}}
\def\emptybox(#1,#2){\framebox{\parbox[c][#2]{#1}{\rule{0pt}{0pt}}}}

% Misc.
\renewcommand{\d}{\displaystyle}
\newcommand{\ds}{\displaystyle}
\def\bc{\begin{center}}
\def\ec{\end{center}}
\def\be{\begin{enumerate}}
\def\ee{\end{enumerate}}


\doctitle{$\mathrightbat$ Math F251X: Quiz 8 }
\docdate{October 31, 2024 $\pumpkin$}
\doccourse{ $ \mathwitch$ UAF Calculus I}
\docversion{v-1 $\mathghost$}
\docscoring{\blank{0.8in} / 25}
\begin{document}
\emph{Please circle your instructor's name:} \hfill Leah Berman  \hfill   Jill Faudree \hfill James Gossell \\

There are 25 points possible on this quiz. Any outside materials (textbook, course notes, calculator) are not allowed.  \emph{For full credit, show all work in a way someone else can follow it.} 

\begin{enumerate}

%\problem{12} 
\item (12 points) Answer the questions below about the function $\ds f(x)= x^{3}(x+2)$. After simplification, \[ \quad f'(x)= 2 x^2 (2 x+3), \quad \text{ and }  f''(x)= 12 x (x+1).\]
You must show your work and justify your conclusion with a few words or a computation. Make sure someone else can follow your work.
%\begin{subproblems}

\begin{enumerate}[(a)]
	\item Determine the intervals where $f$ is \emph{increasing} and where $f$ is \emph{decreasing}. Show your work. 
	\vfill
	
	\vfill
	
	Increasing: \hrulefill Decreasing:\hrulefill	
	
	{\it (Use interval notation. If none write ``none''.)}
	
	\item Fill in the blanks: $f(x)$ has a local maximum at $x = $ \blank{1in}  and a local minimum at $x = \blank{1in} $. (If none, write ``none''.)
	

	\item Find all intervals where $f$ is \emph{concave up} and where $f$ is \emph{concave down}. Show your work.
	\vfill
	\vfill

	Concave up: \hrulefill Concave down:\hrulefill	
	
	{\it (Use interval notation. If none write ``none''.)}
	
	\item Fill in the blanks: $f(x)$ has (an) inflection point(s) at $x =$ \blank{1in}. (If none, write ``none''.)
	
%	\item \textcolor{red}{ Do we want to ask anything about an asymptote at 0?}

%\end{subproblems}
\end{enumerate}
\newpage

%\problem{6} 
%\begin{subproblems}
%\vspace{-.7cm}
\item (6 points)
\begin{enumerate}[(a)]
\item Determine $\ds \lim_{x\to\infty} \frac{\sqrt{3x^{2}+1}}{4x-2}$. Show some work.
\vspace{1in}



\item Fill in the empty boxes to make a true sentence.

The function $\ds g(x)= \frac{\sqrt{3x^{2}+1}}{4x-2}$ has a horizontal asymptote whose equation is \fbox{\rule{0pt}{2em}\hspace{2.5cm}} because
\[ \lim_{\fbox{\strut\hspace{1.5cm}}} g(x) = \fbox{\strut\hspace{1.5cm}}.\]

%\end{subproblems}
\end{enumerate}

%\problem{7} 
\item (7 points) Sketch a graph of a function $h(x)$ with the following properties:

\begin{multicols}{2}
\begin{itemize}
\item The domain of $h(x)$ is $(-\infty, 3) \cup (3, \infty)$.
\item $h(0)=1$
\item $h(1) = 2$
%\columnbreak
\item $\ds \lim_{x \to -\infty} h(x) = 0$
\item $\ds \lim_{x \to \infty} h(x) = -2$
\item $\ds \lim_{x \to 3^{-}} h(x) = -\infty$
\columnbreak
\item $\ds \lim_{x \to 3^{+}} h(x) = \infty$
\item $h'(x) > 0$ when $x<1$.
\item $h'(x) < 0$ when $1<x<3$ or $x>3$.
\item $h''(x) > 0$ when $x < 0$ or $x > 3$.
\item $h''(x) < 0$ when $0 < x < 3$

\end{itemize}
\end{multicols}

%After drawing the graph:
\begin{itemize}
\item \emph{Label} on the graph the following things, if they exist, by drawing a point on the graph and labeling: any local maximums by writing  \textsf{LOCAL MAX}, local minimums by writing \textsf{LOCAL MIN}, inflection points by writing \textsf{IP} 
\item \emph{Draw} any horizontal and vertical asymptotes with dashed lines and \emph{label} them with their equation.
\item \emph{Mark} any important $x$-values and $y$-values on the $x$- and $y$-axes.
\end{itemize}
\centering{
\begin{tikzpicture}
\draw[<->] (-5,0) --  (5,0);
\draw[<->] (0,3) -- (0,-3);
\end{tikzpicture}
}
 \end{enumerate}
\end{document}



