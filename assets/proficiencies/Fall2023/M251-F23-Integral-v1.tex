
\documentclass[12pt]{article}

% Layout.
\usepackage[top=1in, bottom=0.75in, left=1in, right=1in, headheight=1in, headsep=6pt]{geometry}

% Fonts.
\usepackage{mathptmx}
\usepackage[scaled=0.86]{helvet}
\renewcommand{\emph}[1]{\textsf{\textbf{#1}}}

% TiKZ.
\usepackage{tikz, pgfplots}
\usetikzlibrary{calc}
\pgfplotsset{my style/.append style={axis x line=middle, axis y line=
middle, xlabel={$x$}, ylabel={$y$}, axis equal }}

% Misc packages.
\usepackage{amsmath,amssymb,latexsym}
\usepackage{graphicx}
\usepackage{array}
\usepackage{xcolor}
\usepackage{multicol}
\usepackage{nicefrac}

% Commands to set various header/footer components.
\makeatletter
\def\doctitle#1{\gdef\@doctitle{#1}}
\doctitle{Use {\tt\textbackslash doctitle\{MY LABEL\}}.}
\def\docdate#1{\gdef\@docdate{#1}}
\docdate{Use {\tt\textbackslash docdate\{MY DATE\}}.}
\def\doccourse#1{\gdef\@doccourse{#1}}
\let\@doccourse\@empty
\def\docscoring#1{\gdef\@docscoring{#1}}
\let\@docscoring\@empty
\def\docversion#1{\gdef\@docversion{#1}}
\let\@docversion\@empty
\makeatother

% Headers and footers layout.
\makeatletter
\usepackage{fancyhdr}
\pagestyle{fancy}
\fancyhf{} % Clears all headers/footers.
\lhead{
\ifnum \value{page} > 1\relax\else \vskip 25pt\fi
\baselineskip 25pt
\emph{\@doctitle\hfill\@docdate}
\ifnum \value{page} > 1\relax\else\\
\emph{Name: \rule{2.5in}{1pt}\ \hfill
	%Instructor (circle):\quad Maxwell\quad Jurkowski\quad Sus%\@docscoring
}
\\
\vskip -10pt
\fi}
\rfoot{\emph{\@docversion}}
\lfoot{\emph{\@doccourse}}
\cfoot{\emph{\thepage}}
\renewcommand{\headrulewidth}{0pt}%
\makeatother

% Paragraph spacing
\parindent 0pt
\parskip 6pt plus 1pt

% A problem is a section-like command. Use \problem{5} to
% start a problem worth 5 points.
\newcounter{probcount}
\newcounter{subprobcount}
\setcounter{probcount}{0}
\newcommand{\problem}[1]{%
\par
\addvspace{4pt}%
\setcounter{subprobcount}{0}%
\stepcounter{probcount}%
\makebox[0pt][r]{\emph{\arabic{probcount}.}\hskip1ex}\emph{[#1 points]}\hskip1ex}
\newcommand{\thesubproblem}{\emph{\alph{subprobcount}.}}

% Subproblems are an enumerate-like environment with a consistent
% numbering scheme. 
% Use \begin{subproblems}\item...\item...\end{subproblems}
\newenvironment{subproblems}{%
\begin{enumerate}%
\setcounter{enumi}{\value{subprobcount}}%
\renewcommand{\theenumi}{\emph{\alph{enumi}}}}%
{\setcounter{subprobcount}{\value{enumi}}\end{enumerate}}

% Blanks for answers in normal and math mode.
\newcommand{\blank}[1]{\rule{#1}{0.75pt}}
\newcommand{\mblank}[1]{\underline{\hspace{#1}}}
\def\emptybox(#1,#2){\framebox{\parbox[c][#2]{#1}{\rule{0pt}{0pt}}}}

% Misc.
\renewcommand{\d}{\displaystyle}
\newcommand{\ds}{\displaystyle}
\def\bc{\begin{center}}
\def\ec{\end{center}}


\doctitle{Math F251X: Integral Proficiency}
\docdate{December 6, 2023}
\doccourse{UAF Calculus I}
\docversion{v-1}
\docscoring{\blank{0.8in} / 12}
\begin{document}
\begin{itemize}
\item 
There are 12 points possible on this proficiency: one point per problem with
no partial credit. 

\item You have 30 minutes to complete this proficiency.

\item No aids (book, calculator, etc.) are permitted.  

\item You do \textbf{not} need to simplify your expressions.

\item For at least one problem you must indicate correct
use of a constant of integration.

\item Circle or box your final answer.

\item You must use parentheses correctly. A mis-parenthesized answer is incorrect. Do not write $8x \cdot -x^{2}$ to indicate $8x(-x^{2})$, and definitely do not write $8x\cdot -x^{2}+2$ if you mean $8x(-x^{2}+2)$.
\end{itemize}

\problem{12}  Compute the following definite/indefinite integrals.
\begin{subproblems}

\item $\ds \int (-2x^5+\sin(x)) \; dx$
\vfill
%easy u-sub 
%inside sine

\item $\ds \int \cos(6x)\; dx$
\vfill

%Def integral + substitution
%v. similar to second one of Fall 2021
\item $\ds \int_1^{2} xe^{x^2}\; dx$
\vfill


\newpage
%basic rules: power, ln, contants
\item $\ds \int \left( \frac{x}{2}+\frac{4}{x} +\frac{6}{5}\right)\; dx$
\vfill
%easy u-sub for ln
\item $\ds \int \frac{1-2\sin(2x)}{x+\cos(2x)}\; dx$
\vfill
%thoughtful u-sub, power rule, 
%exact homework problems
\item $\ds \int \frac{7}{3x(\ln x)^2}\; dx$
\vfill

\newpage
%arc sine
\item $\ds \int  \frac{1}{\sqrt{1-x^2}}\; dx$
\vfill

%process of elimination u-sub 
%exact homreowrk problems
\item $\ds \int \frac{\arctan(x)}{1+x^2}\; dx$ $\quad$ (recall $ \arctan(x)=\tan^{-1}(x) $)
\vfill

%no sub + u-sub
%e and trig
\item $\ds \int (e^{-2x}+\sec(x)\tan(x)) \; dx$
\vfill

\newpage
%nonproduct rule
\item $\ds \int_{-2}^1 x(3-x)\; dx$
\vfill
%sqrt in denominator, u-sub
\item $\ds \int \frac{x^4}{\sqrt{6-x^5}} \; dx$
\vfill
%sophisticated substitution \# 283
\item $\ds \int \frac{x}{x+2} \: dx$
\vfill

\end{subproblems}
\end{document}