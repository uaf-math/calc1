\documentclass[12pt]{article}

% Layout.
\usepackage[top=1in, bottom=0.75in, left=1in, right=1in, headheight=1in, headsep=6pt]{geometry}

% Fonts.
\usepackage{mathptmx}
\usepackage[scaled=0.86]{helvet}
\renewcommand{\emph}[1]{\textsf{\textbf{#1}}}

% TiKZ.
\usepackage{tikz, pgfplots}
\usetikzlibrary{calc}
\pgfplotsset{my style/.append style={axis x line=middle, axis y line=
middle, xlabel={$x$}, ylabel={$y$}, axis equal }}

% Misc packages.
\usepackage{amsmath,amssymb,latexsym}
\usepackage{graphicx}
\usepackage{array}
\usepackage{xcolor}
\usepackage{multicol,tabularx}

% Commands to set various header/footer components.
\makeatletter
\def\doctitle#1{\gdef\@doctitle{#1}}
\doctitle{Use {\tt\textbackslash doctitle\{MY LABEL\}}.}
\def\docdate#1{\gdef\@docdate{#1}}
\docdate{Use {\tt\textbackslash docdate\{MY DATE\}}.}
\def\doccourse#1{\gdef\@doccourse{#1}}
\let\@doccourse\@empty
\def\docscoring#1{\gdef\@docscoring{#1}}
\let\@docscoring\@empty
\def\docversion#1{\gdef\@docversion{#1}}
\let\@docversion\@empty
\makeatother

% Headers and footers layout.
\makeatletter
\usepackage{fancyhdr}
\pagestyle{fancy}
\fancyhf{} % Clears all headers/footers.
\lhead{\baselineskip 30pt
\emph{\@doctitle\hfill\@docdate}
\ifnum \value{page} > 1\relax\else\\
\emph{Name: \rule{3.5in}{1pt}\ \hfill
%Class (circle): \ \  Sync. \hfill Online%\@docscoring
}
\fi}
\rfoot{\emph{\@docversion}}
\lfoot{\emph{\@doccourse}}
\cfoot{\emph{\thepage}}
\renewcommand{\headrulewidth}{0pt}%
\makeatother

% Paragraph spacing
\parindent 0pt
\parskip 6pt plus 1pt

% A problem is a section-like command. Use \problem{5} to
% start a problem worth 5 points.
\newcounter{probcount}
\newcounter{subprobcount}
\setcounter{probcount}{0}
\newcommand{\problem}[1]{%
\par
\addvspace{4pt}%
\setcounter{subprobcount}{0}%
\stepcounter{probcount}%
\makebox[0pt][r]{\emph{\arabic{probcount}.}\hskip1ex}\emph{[#1 points]}\hskip1ex}
\newcommand{\thesubproblem}{\emph{\alph{subprobcount}.}}

% Subproblems are an enumerate-like environment with a consistent
% numbering scheme. 
% Use \begin{subproblems}\item...\item...\end{subproblems}
\newenvironment{subproblems}{%
\begin{enumerate}%
\setcounter{enumi}{\value{subprobcount}}%
\renewcommand{\theenumi}{\emph{\alph{enumi}}}}%
{\setcounter{subprobcount}{\value{enumi}}\end{enumerate}}

% Blanks for answers in normal and math mode.
\newcommand{\blank}[1]{\rule{#1}{0.75pt}}
\newcommand{\mblank}[1]{\underline{\hspace{#1}}}
\def\emptybox(#1,#2){\framebox{\parbox[c][#2]{#1}{\rule{0pt}{0pt}}}}

% Misc.
\renewcommand{\d}{\displaystyle}
\newcommand{\ds}{\displaystyle}
\def\bc{\begin{center}}
\def\ec{\end{center}}


\doctitle{Math F251X: Derivative Proficiency}
\docdate{Fall 2024}
\doccourse{UAF Calculus I}
\docversion{v-1 try 2}
\docscoring{\blank{0.8in} / 12}

\begin{document}
%%Page 1
%%Directions
\begin{tabular}{m{12cm} m{5cm}}
{\large{Rules:}} & {\large{Circle your instructor:}}\\
&\\

$\circ$ One point per problem, 12 points total. & Leah Berman\\

$\circ$  {\bf No partial credit}. & \\

$\circ$ Time to complete: 1 hour. & Jill Faudree\\

$\circ$ No aids (book, calculator, etc.) permitted.  &\\

$\circ$ You do \textbf{not} need to simplify your expressions.&James Gossell\\

$\circ$ Show sufficient work to justify your final expression. &\\

$\circ$ Final answers \textbf{must start with} $f'(x)=$, $\frac{dy}{dx}=$, or similar.&\\

\end{tabular}

\vspace{.2in}

%Beginning of problems
Compute the derivatives of the following functions. Each problem is worth 1 point for a total of 12 points.
\begin{enumerate}


\item $\ds y=e^{x/2}\sin(1-4x)$

%\textcolor{red}{Looks just like 3rd problem on Try 1 thus rewarding students who learn how to do that particular problem. Satisfies (i) product rule, (ii) product rule with chain rule (iii) chain rule with $e^u$, (iv) chain rule w ith $u$ in a trig fcn}

\vfill

\item $\ds f(x)= \frac{x-\ln(2)}{5}-\frac{1}{6x} $\\

%\textcolor{red}{From Spr 2019. Satisfies (i) $\ln(c)$ (ii) intelligent use of constants (iii) negative exponents} 

\vfill

\item   $\ds L(t)=\ln(t^2+\cos^2(t))$\\

%\textcolor{red}{Looks just like problem (e) on Try 1  thus rewarding students who learn how to do that particular problem. Satisfies (i) chain rule in $\ln$ (ii) double-chain rule problem (iii) derivative of cosine (iv) derivative of chain rule with $u^p$.} 

\vfill
\newpage

%%%Page 2

\item   $\ds y(x)=\frac{\pi \sec(x)}{1+\ln(x)} $\\

%\textcolor{red}{Spring 2019: (i) quotient rule, (ii) secant (iii) straight $\ln(x)$ (iv) $\pi$ as constant}\\
\vfill

\item   $\ds j(\theta) = \tan(\theta-\sqrt[3]{\theta^2+1})$\\

%\textcolor{red}{From Spring 2019. (i) tangent, (ii) the second double chain rule (iii) understand root notation}

\vfill

\item  $\ds y=4 \log_{10} (x^2) + (\sin(x))^{-5}$

%\textcolor{red}{(i) logarithm with standard base, (ii) algebra makes it easier, (iii) forces the negative exponent check, (iv) sine} 

\vfill



\newpage

%%%page 3
\item   $\ds x(\theta)=\arcsin(2\theta ) $ (Note: $\arcsin(2 \theta)$ is the same as $\sin^{-1}(2 \theta)$)\\

%\textcolor{red}{ Spring 2019. (i) arctrig function, }\\

\vfill

\item   $\ds u(x)=(e^2+e^x)(\sqrt{6}-x^2) $

%\textcolor{red}{ Fall 2019. (i) product rule (second time, (ii) $\sqrt{c}$, (ii) plain $e^x$}\\

\vfill

\item   $\ds f(x)=\frac{1}{{x^2+1}}+\frac{1}{\tan(x)}$\\

%\textcolor{red}{Fall 2019. (i) cube root (ii) negative exponent (iii) choice to use $\cot(x)$ or quotient or chain}\\

\vfill

\newpage
%%%% page 4

\item   $\ds y = \sqrt{\frac{2^{x}}{x^{3}}} $\\

%\textcolor{red}{Left over from Try 1. I changed the $x^2$ in the denominator to $x^3$ to avoid the derivative of $|x|$. Satisfies (i) easier if simplified and (ii) exponential with non standard base.} 

\vfill

\item   $\ds f(x) = x^k+e^{-kx}+2k$, where $k$ is a fixed constant\\

%\textcolor{red}{From Spring 2019. (i) use of parameters, (ii) chain rule with $e^u$ ( a second time for this but OK for me due to its frequency of use in science)}
\vfill

\item  Find $\ds \frac{dy}{dx}$ for $\ds 
x^2y^2+2x=2 +\ln(y)
.$ {\it [You must solve for $\ds \frac{dy}{dx}$.]}\\

%\textcolor{red}{Fall 2019. $dy/dx$ on both sides} 

\vfill

\end{enumerate}
\end{document}