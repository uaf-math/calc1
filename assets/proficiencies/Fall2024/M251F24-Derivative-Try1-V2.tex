\documentclass[12pt]{article}

% Layout.
\usepackage[top=1in, bottom=0.75in, left=1in, right=1in, headheight=1in, headsep=6pt]{geometry}

% Fonts.
\usepackage{mathptmx}
\usepackage[scaled=0.86]{helvet}
\renewcommand{\emph}[1]{\textsf{\textbf{#1}}}

% TiKZ.
\usepackage{tikz, pgfplots}
\usetikzlibrary{calc}
\pgfplotsset{my style/.append style={axis x line=middle, axis y line=
middle, xlabel={$x$}, ylabel={$y$}, axis equal }}

% Misc packages.
\usepackage{amsmath,amssymb,latexsym}
\usepackage{graphicx}
\usepackage{array}
\usepackage{xcolor}
\usepackage{multicol}

% Commands to set various header/footer components.
\makeatletter
\def\doctitle#1{\gdef\@doctitle{#1}}
\doctitle{Use {\tt\textbackslash doctitle\{MY LABEL\}}.}
\def\docdate#1{\gdef\@docdate{#1}}
\docdate{Use {\tt\textbackslash docdate\{MY DATE\}}.}
\def\doccourse#1{\gdef\@doccourse{#1}}
\let\@doccourse\@empty
\def\docscoring#1{\gdef\@docscoring{#1}}
\let\@docscoring\@empty
\def\docversion#1{\gdef\@docversion{#1}}
\let\@docversion\@empty
\makeatother

% Headers and footers layout.
\makeatletter
\usepackage{fancyhdr}
\pagestyle{fancy}
\fancyhf{} % Clears all headers/footers.
\lhead{\baselineskip 30pt
\emph{\@doctitle\hfill\@docdate}
\ifnum \value{page} > 1\relax\else\\
\emph{Name: \rule{3.5in}{1pt}\ \hfill
%Class (circle): \ \  Sync. \hfill Online%\@docscoring
}
\fi}
\rfoot{\emph{\@docversion}}
\lfoot{\emph{\@doccourse}}
\cfoot{\emph{\thepage}}
\renewcommand{\headrulewidth}{0pt}%
\makeatother

% Paragraph spacing
\parindent 0pt
\parskip 6pt plus 1pt

% A problem is a section-like command. Use \problem{5} to
% start a problem worth 5 points.
\newcounter{probcount}
\newcounter{subprobcount}
\setcounter{probcount}{0}
\newcommand{\problem}[1]{%
\par
\addvspace{4pt}%
\setcounter{subprobcount}{0}%
\stepcounter{probcount}%
\makebox[0pt][r]{\emph{\arabic{probcount}.}\hskip1ex}\emph{[#1 points]}\hskip1ex}
\newcommand{\thesubproblem}{\emph{\alph{subprobcount}.}}

% Subproblems are an enumerate-like environment with a consistent
% numbering scheme. 
% Use \begin{subproblems}\item...\item...\end{subproblems}
\newenvironment{subproblems}{%
\begin{enumerate}%
\setcounter{enumi}{\value{subprobcount}}%
\renewcommand{\theenumi}{\emph{\alph{enumi}}}}%
{\setcounter{subprobcount}{\value{enumi}}\end{enumerate}}

% Blanks for answers in normal and math mode.
\newcommand{\blank}[1]{\rule{#1}{0.75pt}}
\newcommand{\mblank}[1]{\underline{\hspace{#1}}}
\def\emptybox(#1,#2){\framebox{\parbox[c][#2]{#1}{\rule{0pt}{0pt}}}}

% Misc.
\renewcommand{\d}{\displaystyle}
\newcommand{\ds}{\displaystyle}
\def\bc{\begin{center}}
\def\ec{\end{center}}


\doctitle{Math F251X: Derivative Proficiency}
\docdate{October 17, 2024}
\doccourse{UAF Calculus I}
\docversion{v-2 try 1}
\docscoring{\blank{0.8in} / 12}

\begin{document}

\begin{itemize}
\addtolength\itemsep{-1mm}
\item There are 12 points possible on this proficiency, one point per problem. {\bf No partial credit will be given.}

\item You have one hour to complete this proficiency.

\item No aids (book, calculator, etc.) are permitted.  

\item You do \textbf{not} need to simplify your expressions.

\item You must show sufficient work to justify your final expression. A correct answer for a nontrivial computation with no supporting work will be marked as incorrect.


\item Your final answers \textbf{must start with} $f'(x)=$, $\frac{dy}{dx}=$, or similar.

\item {\bf \fbox{Draw a box around your final answer.}}
\end{itemize}




\problem{12}  Compute the derivatives of the following functions.
\begin{subproblems}

%%% page 2

\item $\ds f(t) = 7t^{8} + \frac{9}{t} + \sqrt{\frac{3}{11}}$

\vfill

\item $\ds g(x) =  \ln( 6x^{2}) + \tan(x)$\\  \\

%\textcolor{red}{BEFORE:} $\ds g(x) =  \log_{5}( 7x^{2}) + \sin(x)$\\
%\textcolor{blue}{I changed the base from 5 to e. Make the problems on page 1 easier. Include non-standard logarithm in part (i)}
\vfill

\item   $\ds y = e^{3x^{2}-4}\sin(12x-3)$
\vfill

%\item   $\ds f(\theta)=\ln(\tan \theta + \csc \theta)$
%\vfill

\newpage

\item   $\ds h(x) = \frac{7\sec(3x)}{9e^{x} +  \sqrt{3}}$
\vfill



\item   $\ds j(\theta) = \ln( \cot(\theta) + \cos(5\theta) )$\\



\vfill


\item   $\ds f(x) = 5^{x}(Ax+B)^{-1/2}, $ where $A$ and $B$ are fixed constants \\


%\textcolor{red}{BEFORE:} $\ds f(x) = 3^{x}\cos(5x^{2} - \ln(2)) $\\
%\textcolor{blue}{I changed the second function from cosine to a function that includes parameters and uses the chain rule where the outside function is a power. I moved the $\ln(2)$ constant to the next problem.}

\vfill



\newpage

%%%page 3
\item   $\ds y =  {\pi}\;{\csc(x)}+\ln(3)$\\


%\textcolor{red}{BEFORE:} $\ds y =  \frac{\pi}{\sec(x)} $\\
%\textcolor{blue}{I moved secant to the numerator so students have to know its derivative and added the $\ln(2)$ constant to the previous problem.}

\vfill

\item   $\ds k(t) = \frac{t^{2} - 4t + 5}{t^{3/2}} $
\vfill

\item   $\ds f(h) = \frac{h+\log_{3}(h^2)}{7}$\\

%\textcolor{red}{BEFORE:} $\ds f(h) = \ln(\sqrt[3]{4x^{5}+3}) $\\
%\textcolor{blue}{I moved the nonstandard logarithm here and added a modest intelligent use of constants aspect. It is also simplified so that the chain rule is not needed if log rules are applied.}

\vfill



\newpage
%%%% page 4


\item   $\ds y = \sqrt[3]{e^2+e^{\cos(x)}}$\\

%\textcolor{red}{BEFORE:} $\ds y = \sqrt{\frac{2^{x}}{x^{2}}} $\\
%\textcolor{blue}{I just changed this problem. It allowed me to add a second double-chain rule problem. Also,  I was thinking that non-standard bases should appear exactly twice: once as an exponential and once as a logarithm. Since it appeared earlier, I ditched it here. }

\vfill
\vfill

\item  $\ds f(x)= \arctan{(5x)}$ $\quad$ (this is the same as writing $\ds f(x)=\tan^{-1}{(5x)}$)
\vfill

\item  Find $\ds \frac{dy}{dx}$ for $\ds 
y^{3} + \cos(x + y^{2}) = x^{4} - 12
.$ {\it [You must solve for $\ds \frac{dy}{dx}$.]}\\

%\textcolor{red}{BEFORE:} $\ds \frac{dy}{dx}$ for $\ds 
%y^{4} + \cot(x + y^{2}) = x^{3} - 7x
%.$\\
%\textcolor{blue}{I made it easier by changing the cotangent to cosine and I changed the 7x to 7.}



\vfill
\vfill



\end{subproblems}
\end{document}