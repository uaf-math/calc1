
\documentclass[12pt]{article}

% Layout.
\usepackage[top=1in, bottom=0.75in, left=1in, right=1in, headheight=1in, headsep=6pt]{geometry}

% Fonts.
\usepackage{mathptmx}
\usepackage[scaled=0.86]{helvet}
\renewcommand{\emph}[1]{\textsf{\textbf{#1}}}

% TiKZ.
\usepackage{tikz, pgfplots}
\usetikzlibrary{calc}
\pgfplotsset{my style/.append style={axis x line=middle, axis y line=
middle, xlabel={$x$}, ylabel={$y$}, axis equal }}

% Misc packages.
\usepackage{amsmath,amssymb,latexsym}
\usepackage{graphicx}
\usepackage{array}
\usepackage{xcolor}
\usepackage{multicol}
\usepackage{nicefrac}

% Commands to set various header/footer components.
\makeatletter
\def\doctitle#1{\gdef\@doctitle{#1}}
\doctitle{Use {\tt\textbackslash doctitle\{MY LABEL\}}.}
\def\docdate#1{\gdef\@docdate{#1}}
\docdate{Use {\tt\textbackslash docdate\{MY DATE\}}.}
\def\doccourse#1{\gdef\@doccourse{#1}}
\let\@doccourse\@empty
\def\docscoring#1{\gdef\@docscoring{#1}}
\let\@docscoring\@empty
\def\docversion#1{\gdef\@docversion{#1}}
\let\@docversion\@empty
\makeatother

% Headers and footers layout.
\makeatletter
\usepackage{fancyhdr}
\pagestyle{fancy}
\fancyhf{} % Clears all headers/footers.
\lhead{
\ifnum \value{page} > 1\relax\else \vskip 25pt\fi
\baselineskip 25pt
\emph{\@doctitle\hfill\@docdate}
\ifnum \value{page} > 1\relax\else\\
\emph{Name: \rule{2.5in}{1pt}\ \hfill
	%Instructor (circle):\quad Maxwell\quad Jurkowski\quad Sus%\@docscoring
}
\\
\vskip -10pt
\fi}
\rfoot{\emph{\@docversion}}
\lfoot{\emph{\@doccourse}}
\cfoot{\emph{\thepage}}
\renewcommand{\headrulewidth}{0pt}%
\makeatother

% Paragraph spacing
\parindent 0pt
\parskip 6pt plus 1pt

% A problem is a section-like command. Use \problem{5} to
% start a problem worth 5 points.
\newcounter{probcount}
\newcounter{subprobcount}
\setcounter{probcount}{0}
\newcommand{\problem}[1]{%
\par
\addvspace{4pt}%
\setcounter{subprobcount}{0}%
\stepcounter{probcount}%
\makebox[0pt][r]{\emph{\arabic{probcount}.}\hskip1ex}\emph{[#1 points]}\hskip1ex}
\newcommand{\thesubproblem}{\emph{\alph{subprobcount}.}}

% Subproblems are an enumerate-like environment with a consistent
% numbering scheme. 
% Use \begin{subproblems}\item...\item...\end{subproblems}
\newenvironment{subproblems}{%
\begin{enumerate}%
\setcounter{enumi}{\value{subprobcount}}%
\renewcommand{\theenumi}{\emph{\alph{enumi}}}}%
{\setcounter{subprobcount}{\value{enumi}}\end{enumerate}}

% Blanks for answers in normal and math mode.
\newcommand{\blank}[1]{\rule{#1}{0.75pt}}
\newcommand{\mblank}[1]{\underline{\hspace{#1}}}
\def\emptybox(#1,#2){\framebox{\parbox[c][#2]{#1}{\rule{0pt}{0pt}}}}

% Misc.
\renewcommand{\d}{\displaystyle}
\newcommand{\ds}{\displaystyle}
\def\bc{\begin{center}}
\def\ec{\end{center}}


\doctitle{Math 251: Integral Proficiency}
\docdate{December 3, 2021}
\doccourse{UAF Calculus I}
\docversion{v-3}
\docscoring{\blank{0.8in} / 12}
\begin{document}
\begin{itemize}
\item 
There are 12 points possible on this proficiency: one point per problem with
no partial credit. 

\item You have 60 minutes to complete this proficiency.

\item No aids (book, calculator, etc.) are permitted.  

\item You do \textbf{not} need to simplify your expressions.

\item For at least one problem you must indicate correct
use of a constant of integration.

\item Circle your final answer.
\end{itemize}

\problem{12}  Compute the following definite/indefinite integrals.
\begin{subproblems}

\item $\ds \int_1^2 \frac{2+x^3}{x^2} \; dx$
\vfill

\item $\ds \int_0^{\pi} (6x+\sin\left(\frac{x}{2}\right))\; dx$
\vfill

\item $\ds \int 10x^2(x-5)\; dx$
\vfill


\newpage
\item $\ds \int e^x\cos(1+e^x)\; dx$
\vfill

\item $\ds \int \frac{1}{x^5}+\frac{\sqrt{x}}{5}\; dx$
\vfill





\item $\ds \int \frac{e^{3x}}{\sqrt{5+e^{3x}}}\; dx$
\vfill

\newpage

\item $\ds \int \frac{1}{x} + \sec(x)\tan(x) \; dx$
\vfill

\item $\ds \int \left(\frac{1}{\sqrt{1-x^2}}+\frac{1-x^2}{3}\right) \; dx$
\vfill

\item $\ds \int \frac{3x}{x^2+1}\; dx$
\vfill


\newpage
\item $\ds \int x\sqrt{2-x} \; dx$
\vfill

\item $\ds \int \tan(x)\sec^2(x)\; dx$
\vfill

\item $\ds \int  \frac{x+e^{-x}}{8}\; dx$
\vfill
\end{subproblems}
\end{document}