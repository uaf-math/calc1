\documentclass[12pt]{article}

% Layout.
\usepackage[top=1in, bottom=0.75in, left=1in, right=1in, headheight=1in, headsep=6pt]{geometry}

% Fonts.
\usepackage{mathptmx}
\usepackage[scaled=0.86]{helvet}
\renewcommand{\emph}[1]{\textsf{\textbf{#1}}}

% TiKZ.
\usepackage{tikz, pgfplots}
\usetikzlibrary{calc}
\pgfplotsset{my style/.append style={axis x line=middle, axis y line=
middle, xlabel={$x$}, ylabel={$y$}, axis equal }}

% Misc packages.
\usepackage{amsmath,amssymb,latexsym}
\usepackage{graphicx}
\usepackage{array}
\usepackage{xcolor}
\usepackage{multicol}

% Commands to set various header/footer components.
\makeatletter
\def\doctitle#1{\gdef\@doctitle{#1}}
\doctitle{Use {\tt\textbackslash doctitle\{MY LABEL\}}.}
\def\docdate#1{\gdef\@docdate{#1}}
\docdate{Use {\tt\textbackslash docdate\{MY DATE\}}.}
\def\doccourse#1{\gdef\@doccourse{#1}}
\let\@doccourse\@empty
\def\docscoring#1{\gdef\@docscoring{#1}}
\let\@docscoring\@empty
\def\docversion#1{\gdef\@docversion{#1}}
\let\@docversion\@empty
\makeatother

% Headers and footers layout.
\makeatletter
\usepackage{fancyhdr}
\pagestyle{fancy}
\fancyhf{} % Clears all headers/footers.
\lhead{\baselineskip 30pt
\emph{\@doctitle\hfill\@docdate}
\ifnum \value{page} > 1\relax\else\\
\emph{Name: \rule{3.5in}{1pt}\ \hfill
%Class (circle): \ \  Sync. \hfill Online%\@docscoring
}
\fi}
\rfoot{\emph{\@docversion}}
\lfoot{\emph{\@doccourse}}
\cfoot{\emph{\thepage}}
\renewcommand{\headrulewidth}{0pt}%
\makeatother

% Paragraph spacing
\parindent 0pt
\parskip 6pt plus 1pt

% A problem is a section-like command. Use \problem{5} to
% start a problem worth 5 points.
\newcounter{probcount}
\newcounter{subprobcount}
\setcounter{probcount}{0}
\newcommand{\problem}[1]{%
\par
\addvspace{4pt}%
\setcounter{subprobcount}{0}%
\stepcounter{probcount}%
\makebox[0pt][r]{\emph{\arabic{probcount}.}\hskip1ex}\emph{[#1 points]}\hskip1ex}
\newcommand{\thesubproblem}{\emph{\alph{subprobcount}.}}

% Subproblems are an enumerate-like environment with a consistent
% numbering scheme. 
% Use \begin{subproblems}\item...\item...\end{subproblems}
\newenvironment{subproblems}{%
\begin{enumerate}%
\setcounter{enumi}{\value{subprobcount}}%
\renewcommand{\theenumi}{\emph{\alph{enumi}}}}%
{\setcounter{subprobcount}{\value{enumi}}\end{enumerate}}

% Blanks for answers in normal and math mode.
\newcommand{\blank}[1]{\rule{#1}{0.75pt}}
\newcommand{\mblank}[1]{\underline{\hspace{#1}}}
\def\emptybox(#1,#2){\framebox{\parbox[c][#2]{#1}{\rule{0pt}{0pt}}}}

% Misc.
\renewcommand{\d}{\displaystyle}
\newcommand{\ds}{\displaystyle}
\def\bc{\begin{center}}
\def\ec{\end{center}}


\doctitle{Math 251: Derivative Proficiency}
\docdate{3 November 2022}
\doccourse{UAF Calculus I}
\docversion{v-4}
\docscoring{\blank{0.8in} / 12}

\begin{document}
\begin{itemize}
\addtolength\itemsep{-1mm}
\item There are 12 points possible on this proficiency, one point per problem. {\bf No partial credit will be given.}

\item You have 1 hour to complete this proficiency.

\item No aids (book, calculator, etc.) are permitted.  

\item You do \textbf{not} need to simplify your expressions.

\item Correct parenthesization is required.

%\item You must show sufficient work to justify your final expression; a correct answer for a nontrivial computation with no supporting work will be marked as incorrect.


\item Your final answers \textbf{must start with} $f'(x)=$, $dy/dx=$, or similar.

\item {\bf Circle or box your final answer.}
\end{itemize}

\problem{12}  Compute the derivatives of the following functions.
\begin{subproblems}
%page 1 problems
\item $\ds f(x) =(\sqrt{3})x + \frac{1}{\sqrt{7x}}-\sqrt{\frac{2}{3}}$
%\textcolor{red}{type: intelligent use of constants}
\vfill 
\item $\ds g(x)=e^x\cos(x)$
%\textcolor{red}{type: straight product rule, e^x, cos x}

\vfill 
\item $\ds h(\theta)=\sec\left(\frac{\theta}{9}\right)$
%\textcolor{red}{type: straight chain rule}
\vfill 
\newpage
%page 2 problems
\item $\ds y=(x+\ln(x^2-4))^3$
%\textcolor{red}{type: double chain rule that requires care}

\vfill 
\item $\ds k(x)=\frac{1}{x}+x\arcsin(x)$
%\textcolor{red}{type: arc trig function with product rule}

\vfill 
\item $\ds r(x)=\frac{\cos(\pi x)}{e^{2x}+1} $
%\textcolor{red}{type: an obvious quotient}

\vfill 

\newpage
%page 3 problems
\item $\ds f(x)=\left(x^2 + \ln(x)\right)^a$ where $a$ is a fixed constant
%\textcolor{red}{type: simple chain with parameter}
\vfill 
\item $\ds y=\cot(x)$
%\textcolor{red}{type: Chain with quotient}

\vfill 
\item $\ds y=\sin^6(x^2) $
%\textcolor{red}{type: double chain, chain with e}

\vfill 



\newpage
%page 4 problems
\item $\ds f(x)=\tan \left( \frac{2-x}{3}\right)$
%\textcolor{red}{type: double chain rule, cosine}

\vfill 
\item $\ds y=(\pi -1)x^\pi$
%\textcolor{red}{type: easy quotient rule or choice to rewrite, $e^x,$ tempting bad algebra}

\vfill 

\item Find $\ds \frac{dy}{dx}$ for $\ds \ln(y) +x=10+xy^2 $
%\textcolor{red}{type: implicit differentiation with nontrivial algebra, sine}

\vfill 
\end{subproblems}
\end{document}
