\documentclass[12pt]{article}

% Layout.
\usepackage[top=1in, bottom=0.75in, left=1in, right=1in, headheight=1in, headsep=6pt]{geometry}

% Fonts.
\usepackage{mathptmx}
\usepackage[scaled=0.86]{helvet}
\renewcommand{\emph}[1]{\textsf{\textbf{#1}}}

% TiKZ.
\usepackage{tikz, pgfplots}
\usetikzlibrary{calc}
\pgfplotsset{my style/.append style={axis x line=middle, axis y line=
middle, xlabel={$x$}, ylabel={$y$}, axis equal }}

% Misc packages.
\usepackage{amsmath,amssymb,latexsym}
\usepackage{graphicx}
\usepackage{array}
\usepackage{xcolor}
\usepackage{multicol}

% Commands to set various header/footer components.
\makeatletter
\def\doctitle#1{\gdef\@doctitle{#1}}
\doctitle{Use {\tt\textbackslash doctitle\{MY LABEL\}}.}
\def\docdate#1{\gdef\@docdate{#1}}
\docdate{Use {\tt\textbackslash docdate\{MY DATE\}}.}
\def\doccourse#1{\gdef\@doccourse{#1}}
\let\@doccourse\@empty
\def\docscoring#1{\gdef\@docscoring{#1}}
\let\@docscoring\@empty
\def\docversion#1{\gdef\@docversion{#1}}
\let\@docversion\@empty
\makeatother

% Headers and footers layout.
\makeatletter
\usepackage{fancyhdr}
\pagestyle{fancy}
\fancyhf{} % Clears all headers/footers.
\lhead{\baselineskip 30pt
\emph{\@doctitle\hfill\@docdate}
\ifnum \value{page} > 1\relax\else\\
\emph{Name: \rule{3.5in}{1pt}\ \hfill
%Class (circle): \ \  Sync. \hfill Online%\@docscoring
}
\fi}
\rfoot{\emph{\@docversion}}
\lfoot{\emph{\@doccourse}}
\cfoot{\emph{\thepage}}
\renewcommand{\headrulewidth}{0pt}%
\makeatother

% Paragraph spacing
\parindent 0pt
\parskip 6pt plus 1pt

% A problem is a section-like command. Use \problem{5} to
% start a problem worth 5 points.
\newcounter{probcount}
\newcounter{subprobcount}
\setcounter{probcount}{0}
\newcommand{\problem}[1]{%
\par
\addvspace{4pt}%
\setcounter{subprobcount}{0}%
\stepcounter{probcount}%
\makebox[0pt][r]{\emph{\arabic{probcount}.}\hskip1ex}\emph{[#1 points]}\hskip1ex}
\newcommand{\thesubproblem}{\emph{\alph{subprobcount}.}}

% Subproblems are an enumerate-like environment with a consistent
% numbering scheme. 
% Use \begin{subproblems}\item...\item...\end{subproblems}
\newenvironment{subproblems}{%
\begin{enumerate}%
\setcounter{enumi}{\value{subprobcount}}%
\renewcommand{\theenumi}{\emph{\alph{enumi}}}}%
{\setcounter{subprobcount}{\value{enumi}}\end{enumerate}}

% Blanks for answers in normal and math mode.
\newcommand{\blank}[1]{\rule{#1}{0.75pt}}
\newcommand{\mblank}[1]{\underline{\hspace{#1}}}
\def\emptybox(#1,#2){\framebox{\parbox[c][#2]{#1}{\rule{0pt}{0pt}}}}

% Misc.
\renewcommand{\d}{\displaystyle}
\newcommand{\ds}{\displaystyle}
\def\bc{\begin{center}}
\def\ec{\end{center}}


\doctitle{Math 251: Derivative Proficiency}
\docdate{March 7, 2024}
\doccourse{UAF Calculus I}
\docversion{v-2}
\docscoring{\blank{0.8in} / 12}

\begin{document}

\begin{itemize}
\addtolength\itemsep{-1mm}
\item There are 12 points possible on this proficiency, one point per problem. {\bf No partial credit will be given.}

\item You have one hour to complete this proficiency.

\item No aids (book, calculator, etc.) are permitted.  

\item You do \textbf{not} need to simplify your expressions.

\item Your final answers {\bf must start with} $f'(x)=$, $\frac{dy}{dx}=$, or similar.

\item {\bf \fbox{Draw a box around your final answer.}}
\end{itemize}



\problem{12}  Compute the derivatives of the following functions.
\begin{subproblems}
\item
$\ds f(x)=4\sin(x)\cos(x)$
\vfill

\item   $\ds f(x)=\frac{\sqrt{3}}{4}+\frac{\sqrt{x}}{5}-\frac{5}{\sqrt{x}}$
\vfill


\item  $\ds f(x)=\frac{\ln (x)}{\tan(x)}$
\vfill


\newpage
\item   $\ds y=3\csc(e^x)$
\vfill

\item   $\ds y=5^x-\log_5(x)$
\vfill

\item   $\ds f(x)=\left(x^4+\frac{1}{x}+e^5\right)^3$
\vfill





\newpage
\item   $\ds y=\left( x^{0.2}+\sec(x)\right)^{-2/3}$
\vfill

\item   $\ds f(x) = \frac{\cos(\pi/x)}{x^2}$
\vfill

\item   $\ds f(x)=3 \sin^{-1}(3x^3)$
\vfill



\newpage
\item  $\ds f(x) = \ln\left(\frac{x^2e^x}{14x} \right)$
\vfill

\item   $\ds f(x) = \frac{\sin(6)}{\sqrt[3]{\sin(x)}}$
\vfill




\item  Find $\ds \frac{dy}{dx}$ for the equation $\ds e^x+e^y = 2\sin(xy)$. You must solve for $\ds \frac{dy}{dx}$.
\vfill



\end{subproblems}
\end{document}