% !TEX TS-program = pdflatexmk
\documentclass[12pt]{article}

% Layout.
\usepackage[top=1in, bottom=0.75in, left=1in, right=1in, headheight=1in, headsep=6pt]{geometry}

% Fonts.
\usepackage{mathptmx}
\usepackage[scaled=0.86]{helvet}
\renewcommand{\emph}[1]{\textsf{\textbf{#1}}}

% TiKZ.
\usepackage{tikz, pgfplots}
\usetikzlibrary{calc}
\pgfplotsset{my style/.append style={axis x line=middle, axis y line=
middle, xlabel={$x$}, ylabel={$y$}, axis equal }}

% Misc packages.
\usepackage{amsmath,amssymb,latexsym}
\usepackage{graphicx}
\usepackage{array}
\usepackage{xcolor}
\usepackage{multicol}

% Commands to set various header/footer components.
\makeatletter
\def\doctitle#1{\gdef\@doctitle{#1}}
\doctitle{Use {\tt\textbackslash doctitle\{MY LABEL\}}.}
\def\docdate#1{\gdef\@docdate{#1}}
\docdate{Use {\tt\textbackslash docdate\{MY DATE\}}.}
\def\doccourse#1{\gdef\@doccourse{#1}}
\let\@doccourse\@empty
\def\docscoring#1{\gdef\@docscoring{#1}}
\let\@docscoring\@empty
\def\docversion#1{\gdef\@docversion{#1}}
\let\@docversion\@empty
\makeatother

% Headers and footers layout.
\makeatletter
\usepackage{fancyhdr}
\pagestyle{fancy}
\fancyhf{} % Clears all headers/footers.
\lhead{\baselineskip 30pt
\emph{\@doctitle\hfill\@docdate}
\ifnum \value{page} > 1\relax\else}
%\emph{Name: \rule{3.5in}{1pt}\hfill \@docscoring}\fi}
\rfoot{\emph{\@docversion}}
\lfoot{\emph{\@doccourse}}
\cfoot{\emph{\thepage}}
\renewcommand{\headrulewidth}{0pt}%
\makeatother

% Paragraph spacing
\parindent 0pt
\parskip 6pt plus 1pt

% A problem is a section-like command. Use \problem{5} to
% start a problem worth 5 points.
\newcounter{probcount}
\newcounter{subprobcount}
\setcounter{probcount}{0}
\newcommand{\problem}[1]{%
\par
\addvspace{4pt}%
\setcounter{subprobcount}{0}%
\stepcounter{probcount}%
\makebox[0pt][r]{\emph{\arabic{probcount}.}\hskip1ex}\emph{[#1 points]}\hskip1ex}
\newcommand{\thesubproblem}{\emph{\alph{subprobcount}.}}

% Subproblems are an enumerate-like environment with a consistent
% numbering scheme. 
% Use \begin{subproblems}\item...\item...\end{subproblems}
\newenvironment{subproblems}{%
\begin{enumerate}%
\setcounter{enumi}{\value{subprobcount}}%
\renewcommand{\theenumi}{\emph{\alph{enumi}}}}%
{\setcounter{subprobcount}{\value{enumi}}\end{enumerate}}

% Blanks for answers in normal and math mode.
\newcommand{\blank}[1]{\rule{#1}{0.75pt}}
\newcommand{\mblank}[1]{\underline{\hspace{#1}}}
\def\emptybox(#1,#2){\framebox{\parbox[c][#2]{#1}{\rule{0pt}{0pt}}}}

% Misc.
\renewcommand{\d}{\displaystyle}
\newcommand{\ds}{\displaystyle}
\def\bc{\begin{center}}
\def\ec{\end{center}}


\doctitle{Math 251: Criteria for Derivative Proficiencies}
\docdate{Spring 2023}
\doccourse{UAF Calculus I}
%\docversion{v-2}
%\docscoring{\blank{0.8in} / 12}

\begin{document}
\phantom{foo}
\vspace{0.5in}

\begin{tabular}{r|l|l|l|l|l|l|l|}
\textsl{required element}   & sample & v-1 \phantom{xx} & v-2 \phantom{xx} & v-3 \phantom{xx} & v-4 \phantom{xx} & v-5 \phantom{xx} & v-6 \phantom{xx} \\ \hline
$e^x$                       & & & & & & & \\ \hline
$\ln x$                     & & & & & & & \\ \hline
$\sin(x)$                   & & & & & & & \\ \hline
$\cos(x)$                   & & & & & & & \\ \hline
$\tan(x)$                   & & & & & & & \\ \hline
$\sec(x)$                   & & & & & & & \\ \hline
kooky trig & & & & & & & \\ \hline

$\operatorname{arc}\left\{\begin{matrix} \sin \\ \cos \\ \tan \end{matrix}\right\} \,\leftarrow$ (one of)
                            & & & & & & & \\ \hline
implicit differentiation    & & & & & & & \\ \hline
has parameter               & & & & & & & \\ \hline
has constant: $\pi$ or $e$  & & & & & & & \\ \hline
has constant: $\sqrt{c}$    & & & & & & & \\ \hline
has constant: $\ln c$       & & & & & & & \\ \hline
$\ge 5$ obligatory chain w/ $u$:   & & & & & & & \\ 
  $e^u$ & & & & & & & \\
  $\ln(u)$ & & & & & & & \\
  $(u)^p$  & & & & & & & \\
  inside trig fcn& & & & & & & \\ \hline
$\ge 2$ double chain rule           & & & & & & & \\ \hline
$\ge 2$ product rule        & & & & & & & \\ \hline
$\ge 1$ quotient rule       & & & & & & & \\ \hline
power rule  & & & & & & & \\
w/ fractional exponent      & & & & & & & \\ \hline
power rule  & & & & & & & \\
w/ negative exponent        & & & & & & & \\ \hline
$\ge 1$ quotient rule       & & & & & & & \\ \hline
manage simple constants       & & & & & & & \\ \hline
easy simplification       & & & & & & & \\ \hline

\end{tabular}

\end{document}