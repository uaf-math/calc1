\documentclass[11pt]{report}

\usepackage{geometry,amsmath,amssymb,amsthm,xcolor}

\geometry{margin=1.in}

\theoremstyle{plain}
\newtheorem{thm}{Theorem}
\newtheorem{lem}[thm]{Lemma}
\newtheorem{prop}[thm]{Proposition}
\newtheorem{cor}[thm]{Corollary}

\newcommand{\opt}[1]{\textcolor{red}{#1}}

\begin{document}
\hfill Math 251 Calculus I
\begin{center}
\Large{\textbf{Written Homework Problems \S 4.5}} \\
15 problems for 30 points\\
\end{center}

Problems in \textcolor{red}{red} are optional extra practice.\\

\begin{description}
\item{\S 4.5} \# 202, \opt{203}, 204, 207, 208, \opt{210}, 213, \opt{214}, 215, 216, 217, 220, 225, \opt{227}, 234, 235, \opt{237}, 241, \opt{243}\\

\textbf{Problem A:} Let $f(x)=x^4-4x^3.$\\
(i) Use the First Derivative Test to identify any local maximums or minimums. (Note: $f'(x)=4x^3-12x^2=4x(x-3).$)\\
(ii) Use the Second Derivative Test to identify any local maximums or minimums. (Note: $f''(x)=12x^2-24x=12x(x-2)$.)\\
(iii) Describe the advantages and disadvantages of the two tests.\\
(iv) Use $f'(x)$ and $f''(x)$ to determine where $f(x)$ is increasing or decreasing, concave up or concave down. Use this information to sketch the graph.\\


\textbf{Problem B:} Let $\displaystyle f(x)=x^{2/3}(6-x)^{1/3},$ $\displaystyle f'(x)= \frac{4-x}{x^{1/3}(6-x)^{2/3}}$, $\displaystyle  f''(x)=\frac{-8}{x^{4/3}(6-x)^{5/3}}.$\\
(i) Determine intervals of increase and decrease.\\
(ii) Identify any local extrema. (ie maxs and mins)\\
(iii) Determine intervals of concavity.\\
(iv) Determine any inflection points.\\
(v) Sketch $f(x).$\\

\end{description}


\end{document}