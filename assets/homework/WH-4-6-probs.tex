\documentclass[11pt]{report}

\usepackage{geometry,amsmath,amssymb,amsthm,xcolor}

\geometry{margin=1.in}

\theoremstyle{plain}
\newtheorem{thm}{Theorem}
\newtheorem{lem}[thm]{Lemma}
\newtheorem{prop}[thm]{Proposition}
\newtheorem{cor}[thm]{Corollary}

\newcommand{\opt}[1]{\textcolor{red}{#1}}

\begin{document}
\hfill Math 251 Calculus I
\begin{center}
\Large{\textbf{Written Homework Problems \S 4.6}} \\
15 problems for 30 points\\
\end{center}

Problems in \textcolor{red}{red} are optional extra practice.\\


\begin{description}
\item{\S 4.6} \#{253}, 256, 259, 261, 263, 265, 267,\opt{268}, 272*, \opt{273}*, 274*, \opt{277}*, 285\\

* You must \textbf{justify} your answer.\\

\textbf{Graphing Problems:} For each function below, draw a sophisticated graph without the aid of technology. (When you are done, you should check your answer with technology.) Your analysis should include all important features of the graph including:\\
(a) intervals of increase and decrease\\
(b) local maxima and minima, if they exist\\
(c) intervals of concavity and any inflection points\\
(d) any vertical or horizontal asymptotes\\

All your work should be justified. Note that derivatives for each function have been provided for you.

\textbf{A:} $\displaystyle f(x)=\frac{2x^2-8}{x^2-16}$, \quad ($\displaystyle f'(x)=\frac{-48x}{(x^2-16)^2}$,  \quad $\displaystyle f''(x)=\frac{48(16+3x^2)}{(x^2-16)^3}$)  \\
\textbf{B:} $\displaystyle f(x)=(x-4)^{2/3}$,  \quad  ($\displaystyle f'(x)=\frac{2}{3(x-4)^{1/3}}$,  \quad $\displaystyle f''(x)=\frac{-2}{9(x-4)^{4/3}}$)\\
\textbf{C:} $\displaystyle f(x)=e^{-x^2/2}=\frac{1}{e^{x^2/2}}$,  \quad  ($\displaystyle f'(x)=\frac{-x}{e^{x^2/2}}$,  \quad $\displaystyle f''(x)=\frac{x^2-1}{e^{x^2/2}}$)\\
\textbf{D:} $\displaystyle f(x)=\sqrt{x^2-1}$,  \quad ($\displaystyle f'(x)=\frac{x}{\sqrt{x^2-1}}$, \quad  $\displaystyle f''(x)=\frac{-1}{(x^2-1)^{3/2}}$) \\


\textbf{Problem E:} Let $f(x)=Ax+e^{-kx},$ where $A>0$ and $k>0.$ Find $f'(x)$ and $f''(x).$ Determine intervals of increase or decrease and the locations of any local extrema. Determine intervals of concavity and inflection points.




\end{description}


\end{document}