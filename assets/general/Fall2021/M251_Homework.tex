
\documentclass[11pt,fleqn]{article} 
\usepackage[margin=0.8in, head=0.8in]{geometry} 
\usepackage{amsmath, amssymb, amsthm,cmbright}
\usepackage[OT1]{fontenc}
\usepackage{fancyhdr} 
\usepackage{palatino, url, multicol, hyperref}
\usepackage{graphicx, pgfplots, tabularx} 
\usepackage[all]{xy}
\usepackage{polynom} 
%\usepackage{pdfsync} %% I don't know why this messes up tabular column widths
\usepackage{enumerate}
\usepackage{framed}
\usepackage{setspace}
\usepackage{array,tikz, xcolor,colortbl}
\usepackage{ stmaryrd }

\pgfplotsset{compat=1.6}

\pgfplotsset{soldot/.style={color=blue,only marks,mark=*}} \pgfplotsset{holdot/.style={color=blue,fill=white,only marks,mark=*}}

\hypersetup{
    colorlinks=true,
    linkcolor=blue,
    filecolor=magenta,      
    urlcolor=cyan,
    pdftitle={Overleaf Example},
    pdfpagemode=FullScreen,
    }

\renewcommand{\headrulewidth}{0pt}
\newcommand{\blank}[1]{\rule{#1}{0.75pt}}
\newcommand{\bc}{\begin{center}}
\newcommand{\ec}{\end{center}}
\newcommand{\be}{\begin{enumerate}}
\newcommand{\ee}{\end{enumerate}}

\newcommand{\llb}{\llbracket}
\newcommand{\rrb}{\rrbracket}
\renewcommand{\d}{\displaystyle}

\definecolor{ddgreen}{cmyk}{1, 0, .5, 0.2}
\definecolor{dcyan}{cmyk}{1, 0, 0, 0.1}

\pagestyle{fancy} 
\lfoot{MATH 251 Calculus I}
\rfoot{Fall 2021 Homework Problems}

\begin{document}

\vspace*{-0.7in}

%%%%%%%%%intro page
\begin{center}
  {\large{
  \sc{MATH 251 Calculus I \hfill List of Homework Problems \hfill Fall 2021}}}\\
\end{center}
\quad\\
\noindent \textsc{The Basics}\\
\begin{itemize}
\item \textbf{Homework} is due by 11:59PM on the due date. 
\item Turn in your homework via Gradescope. \\
Gradescope is accessed via Canvas. (See menu on left.)\\
Technology help page: 
\url{https://uaf-math251.github.io/techHelp.html}
\item All problems come from our online textbook:\\ \url{https://openstax.org/details/books/calculus-volume-1}
\item Answers to odd problems are linked from the online text and can be found in the back of the paper textbook.
\item Complete solutions to odd problems can be found in the student solutions manual (found at the textbook link).
\item Complete solutions to all problems will be posted in our Canvas site after the homework is due.
\item Homework will be graded on completion and effort. \\
\end{itemize}

\noindent \textsc{How to Get Full Credit}
\begin{itemize}
\item Since you have answers and complete solutions to most of the problems and you are graded based on completion and you have all the problems in advance, \textbf{you should get 100\% on your homework}.
\item Number all problems and all parts of a problem.
\item Write the problems in order.
\item Write legibly and space things so they are easy to find.
\item Follow directions. \textbf{When you are asked for an explanation, make sure to give one!}
\item Turn your homework in on time or in advance. \\
\end{itemize}

\noindent \textsc{Make the Homework into Quiz/Test Prep}
\begin{itemize}
\item Attempt all problems in a particular section \textit{before} looking at the answers.
\item After working the problems, check your answers and re-attempt any that are incorrect.
\item Do not look at the compete worked solutions until you have attempted the problem twice on your own. 
\item Attempt all problems at least two days before they are due so that you can ask questions if you don't understand.
\item Star problems you did correctly on the first try. Circle problems where you got help (using solutions, videos, a tutor, etc). When preparing for a quiz or test, start by going over the circled problems and working a similar (but different) problem of that type in order to know you are ready!
\item Get help when you have questions.
\end{itemize}

\newpage

\begin{tabularx}{\textwidth}{|c|| X | X | }
\hline
Section&Problems& Notes \\
\hline \hline
\S 2.1 & 7,8,9,13,14,15,22,23\\ 
(8 problems)&&\\ \hline

\S 2.2 &32,33,34*,&* Hint: It's a well known constant!\\
(14 problems) &40, (46,47,48,49)**,&** These are all T/F. You are supposed to explain why something is false.\\
&(71,72,73,74,75)***,& ***These are SUPER short problems! Don't despair.\\
&77****&****There are MANY correct answers here. Your answer may look different from the back of the book and still be correct. Ask if you have questions. \\ \hline

\S 2.3 &91,97,98,99,102,103,105,109,111,113,&\\
(12 problems)&119,125&\\ \hline

\S 2.4& 133,135,137,149,153*&*You can use a calculator to help with computations but your explanation must appeal to the IVT.\\
(7 problems)&154,157**&**There are many correct solutions. \\ \hline \hline


\S 3.1  &9,15,21,23,25,39,45,47,51&\\ 
(9 problems)&&\\ \hline

\S 3.2  &56,59,62,65,67,69,75,79,91,95,97,99&\\ 
(12 problems)&& \\ \hline

\S 3.3 (1) &109,110,113,114,115,117,120,123,125,&\\
(13 problems)&127,129,131,136& \\ \hline

\S3.3(2) &141,142,144,146,147&\\
(5 problems) &&\\  \hline

\S 3.4 &151,153,155,159,160,165,167*,168*&*For part (a), you can find a regression tool on the web (say Desmos), but it's OK if you just use the solutions for part (a).\\
(8 problems)&&\\  \hline

\S 3.5 &175,177,180,181,182,187,189,191,&\\
(13 problems)&195,197,199,201,211&\\ \hline

\S 3.6 &217,219,223,225,227,229,233,235&\\
(15 Problems)&237,239,243,247,249,251,253& \\ \hline

\S 3.7 &261,265,273,275,279,283,285,291&\\
(11 problems) &(A) $y=x^{2/3}-6x^{-2/3}+\pi^{4/3}$& For problems A,B,C, find $\frac{dy}{dx}$.\\ 
&(B) $y=(x+\sin(5x))^{8/3}$& \\
&(C) $y=x\cos^{-1}(\frac{\pi x }{2})$ & \\ \hline

\S 3.8 &303,305,307,313,317,321,323&\\ 
(7 problems) && \\ \hline

\S 3.9 &331,332,333,337,339,340,341,343,347&\\ 
(13 problems) &348,355,359,361& \\ \hline
\end{tabularx}
\vfill

\begin{tabularx}{\textwidth}{|c|| X | X | }
\hline
Section&Problems& Notes \\
\hline \hline
4.1&5,7,9,11,12,19,31,35& \\
(8 problems)&& \\ \hline

4.2&51,55,57,59,65,67,73,77,81,84& \\
(10 problems)&& \\ \hline

4.3&91,93,95,103,105,109,111,119,121,126& \\
(10 problems)&& \\ \hline

4.5&203,207,215,217,227,231,235,237,& \\
(10 problems)&241,243& \\ \hline

4.6&253,259,261,263,265,267,273,274,275,& \\
(16 problems) &277,285,& \\
&(295,298,299,301,304)*&*You are encouraged to use the derivatives from the solutions instead of finding those derivatives yourself by hand.\\ \hline

4.7&315,317,319,326,330,331,353,355& \\
(8 problems)&& \\ \hline

4.8&369,371,377,381,383,387,393& \\
(7 problems) && \\ \hline

4.10&467,471,473,476,477,478,481,485,487, 489,490,493,495,497,498,499,505,511,& \\
(19 problems)&512& \\ \hline \hline

5.1&22,39,41,43& \\ 
(4 problems)&& \\ \hline

5.2&75,77,80,89,91,93,111*&*Hint: Use the graph. \\ 
(7 problems)&& \\ \hline

5.3&149,155,159,161,171,177,179,183,185,& \\ 
(13 problems)&187,189,191,199*&*Set up integrals that answer each part, then use a computational tool to actually evaluate the integral.\\ \hline

5.4&207,209,211,213,219,223,227,247,249,& \\ 
(11 problems)&250,251& \\ \hline

5.5&259,273,275,277,279,281,283,284,285,& \\ 
(13 problems)&292,293,297,305& \\ \hline

5.6&321,323,324,325,327,329,331,333,335,& \\ 
(14 problems)&337,339,347,361&\\
&(A) Suppose the rate of growth of bacteria in a Petri dish is given by $p(t)=\frac{e^{0.2t}}{5}$ where $t$ is given in hours and $p(t)$ is given in hundreds of bacteria per hour. If a culture starts with $1000$ bacteria, find a function $P(t)$ that gives the number of bacteria in the Petri dish at any time $t.$ How many bacterial are in the dish after 10  hours. & \\ \hline

5.7&391,393,395,397,399,401,411,413,423,& \\ 
(10 problems) &425& \\ \hline


\end{tabularx}
\vfill
\end{document}
