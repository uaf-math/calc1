\documentclass[12pt]{article}

% Layout.
\usepackage[top=1in, bottom=0.75in, left=1in, right=1in, headheight=1in, headsep=6pt]{geometry}

% Fonts.
\usepackage{mathptmx}
\usepackage[scaled=0.86]{helvet}
\renewcommand{\emph}[1]{\textsf{\textbf{#1}}}

% TiKZ.
\usepackage{tikz, pgfplots}
\usetikzlibrary{calc}
\pgfplotsset{my style/.append style={axis x line=middle, axis y line=
middle, xlabel={$x$}, ylabel={$y$}, axis equal }}

% Misc packages.
\usepackage{amsmath,amssymb,latexsym}
\usepackage{graphicx}
\usepackage{array}
\usepackage{xcolor}
\usepackage{multicol}

% Commands to set various header/footer components.
\makeatletter
\def\doctitle#1{\gdef\@doctitle{#1}}
\doctitle{Use {\tt\textbackslash doctitle\{MY LABEL\}}.}
\def\docdate#1{\gdef\@docdate{#1}}
\docdate{Use {\tt\textbackslash docdate\{MY DATE\}}.}
\def\doccourse#1{\gdef\@doccourse{#1}}
\let\@doccourse\@empty
\def\docscoring#1{\gdef\@docscoring{#1}}
\let\@docscoring\@empty
\def\docversion#1{\gdef\@docversion{#1}}
\let\@docversion\@empty
\makeatother

% Headers and footers layout.
\makeatletter
\usepackage{fancyhdr}
\pagestyle{fancy}
\fancyhf{} % Clears all headers/footers.
\lhead{\baselineskip 30pt
\emph{\@doctitle\hfill\@docdate}
\ifnum \value{page} > 1\relax\else\\
\emph{Name: \rule{3.5in}{1pt}\ \hfill
%Class (circle): \ \  Sync. \hfill Online%\@docscoring
}
\fi}
\rfoot{\emph{\@docversion}}
\lfoot{\emph{\@doccourse}}
\cfoot{\emph{\thepage}}
\renewcommand{\headrulewidth}{0pt}%
\makeatother

% Paragraph spacing
\parindent 0pt
\parskip 6pt plus 1pt

% A problem is a section-like command. Use \problem{5} to
% start a problem worth 5 points.
\newcounter{probcount}
\newcounter{subprobcount}
\setcounter{probcount}{0}
\newcommand{\problem}[1]{%
\par
\addvspace{4pt}%
\setcounter{subprobcount}{0}%
\stepcounter{probcount}%
\makebox[0pt][r]{\emph{\arabic{probcount}.}\hskip1ex}\emph{[#1 points]}\hskip1ex}
\newcommand{\thesubproblem}{\emph{\alph{subprobcount}.}}

% Subproblems are an enumerate-like environment with a consistent
% numbering scheme. 
% Use \begin{subproblems}\item...\item...\end{subproblems}
\newenvironment{subproblems}{%
\begin{enumerate}%
\setcounter{enumi}{\value{subprobcount}}%
\renewcommand{\theenumi}{\emph{\alph{enumi}}}}%
{\setcounter{subprobcount}{\value{enumi}}\end{enumerate}}

% Blanks for answers in normal and math mode.
\newcommand{\blank}[1]{\rule{#1}{0.75pt}}
\newcommand{\mblank}[1]{\underline{\hspace{#1}}}
\def\emptybox(#1,#2){\framebox{\parbox[c][#2]{#1}{\rule{0pt}{0pt}}}}

% Misc.
\renewcommand{\d}{\displaystyle}
\newcommand{\ds}{\displaystyle}
\def\bc{\begin{center}}
\def\ec{\end{center}}


\doctitle{Math 251: Derivative Proficiency Post-Try 1 Extra Practice}
\docdate{Recitation Week 9}
\doccourse{UAF Calculus I}
\docversion{v-practice}
%\docscoring{\blank{0.8in} / 12}

\begin{document}
\addtolength\itemsep{-1mm}

\problem{12}  Compute the derivatives of the following functions.
\begin{subproblems}
% e and chain
\item  $\ds f(x)=e^{(\sin(x))}$
\vfill

%simple prod / quotient 
%sine
\item   $\ds f(x)=\frac{x^2-x}{\cos(x)}$
\vfill

%log and chain
%tan
%sec
\item   $\ds f(x)=\ln(x^2-e^x)$; $\ds f(x)=(\sec(x) +x)^2$; $\ds f(x)=\tan (x^3)$; 
\vfill


%fractions
%fractional powers
%understanding of constants
\item   $\ds f(x)=\frac{x^{1/2}}{2} + \frac{2}{\sqrt[3]{x}} + \frac{1}{\sqrt{5}}$
\vfill

%base other than e
%product rule inside of chain rule
\item   $\ds f(x)=\log_5(x^b\cos x)$ (where $b > 1$);
\vfill

%double chain
%cosine
\item   $\ds f(x) = \left(e^{x/7} + \cos(x)\right)^{3/4}$
\vfill

\newpage
%No triple product rule problems were assigned.
%chain rule
%understanding of constants 
\item   $\ds y=8\left( \frac{\pi-x}{2} \right)^8$
%$\ds f(x) = x^2\sin x \ln x$
\vfill
%arc tric
%chain rule
\item   $\ds f(x) = \arctan(3x)$; $\ds f(x) = \arcsin(3x)$
\vfill
%understanding of constants
%thoughtful algebra
%quotient/product
\item   $\ds f(x) = \frac{4^x}{x\sin(4)}$
\vfill

%chain inside of product
%decimal exponent
\item   $\ds f(x) = (\ln(4+x+x^2))^3$
\vfill

\item  $\ds f(x) = e^{-3x} + e^2 + x^{\pi}$
\vfill

\item  Find $\ds \frac{dy}{dx}$ for $\ds x^3+e^y=25+y\sin(x)$. You must solve for $\ds \frac{dy}{dx}$.
\vfill



\end{subproblems}
\end{document}