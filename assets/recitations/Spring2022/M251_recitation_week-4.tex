
\documentclass[11pt,fleqn]{article} 
\usepackage[margin=0.8in, head=0.8in]{geometry} 
\usepackage{amsmath, amssymb, amsthm}
\usepackage{fancyhdr} 
\usepackage{palatino, url, multicol}
\usepackage{graphicx, pgfplots,xfrac} 
\usepackage[all]{xy}
\usepackage{polynom} 
%\usepackage{pdfsync} %% I don't know why this messes up tabular column widths
\usepackage{enumerate}
\usepackage{framed}
\usepackage{setspace}
\usepackage{array,tikz}

\pgfplotsset{compat=1.6}

\pgfplotsset{soldot/.style={color=black,only marks,mark=*}} \pgfplotsset{holdot/.style={color=black,fill=white,only marks,mark=*}}


\pagestyle{fancy} 
\lfoot{}
\rfoot{ R4: 3-3 \& 3-4 prep}

\begin{document}
\renewcommand{\headrulewidth}{0pt}
\newcommand{\blank}[1]{\rule{#1}{0.75pt}}
\newcommand{\bc}{\begin{center}}
\newcommand{\ec}{\end{center}}
\renewcommand{\d}{\displaystyle}

\vspace*{-0.7in}

%%%%%%%%%intro page
\begin{center}
  \large
  \sc{Recitation: Week 4}\\ \vfill
\end{center}
\vspace*{-1.5in}
\begin{enumerate}
\item Review Topic
	\begin{enumerate}
	\item Write an equation of the line tangent to the graph of $f(x)=\frac{\sin(x)}{2}$ at $x=2 \pi /3.$ 
	\vspace{1in}
	\item On the same set of axes, make a sketch of $f(x)$ and the tangent line from part (a). 
	\vfill
	\item Does your answer in part (a) seem plausible given your sketch in part (b)? Explain.
	\vspace{0.5in}
	\end{enumerate}

\item Weekly Exponential / Logarithm Practice\\
	\begin{enumerate}
	\item Determine whether the equalities below are true or false. Explain how you can remember which is which.\\
	
	$ e^x e^{2x}=e^{3x} \hspace{1.5in} e^xe^{2x}=e^{2x^2}\hspace{1.2in} e^x+e^{2x}=e^{x+2x}$
	
	\vfill
	$(e^{x+1})^2=e^{2(x+1)} \hspace{1in} (e^{x+1})^2=e^{((x+1)^2)} \hspace{1in} (e^{x+1})^2=e^{2x}e^2$ 
	\vfill
	
\newpage
	\item Logarithmic expressions can be rewritten as an exponential expressions. (See example below.)\\
	
	\begin{center}\fbox{\Large{$\log_{10}A=B$}} is equivalent to \fbox{\Large{$10^B=A$}} \end{center}
	\begin{enumerate}
	\item Rewrite each logarithmic expressions as an exponential expression.\\ 
	\fbox{\Large{$\log_{2}A=B$}} is equivalent to \\
	\fbox{\Large{$\ln A=B$}} is equivalent to 
	\item In each expression below, find $y$ by rewriting the logarithmic expression  as an exponential expression.\\
	$\log_2 \frac{1}{32}=y$ \hfill $\ln e^{4.76}=y$ \hfill $\log_{10} y = 1$ 
		
	\end{enumerate}
\end{enumerate}
	\vspace{1in}
In the following problems, you will practice the simplification skills needed to efficiently find derivatives in Section 3.3 and 3.4.

\item Write each expression using a single exponent with $x$ in the numerator, if possible.
	\begin{enumerate}
	\begin{multicols}{2}
	\item $x^ax^b$\\ \vfill
	\item $x^a+x^b$\\ \vfill 
	\item $(x^a)^b$\\ \vfill
	\item $\frac{x^a}{x^b}$\\ \vfill
	\item ($a$ and $b$ are integers) $\sqrt[a]{x^b}$\\ \vfill
	\item ($a$ and $b$ are integers) $(\sqrt[a]{x})^b$\\ \vfill
	\item $\frac{1}{x^a}$\\ \vfill
	\end{multicols}
	\end{enumerate}
\vspace{1in}

\item Write each expression below such that $x$ has a single exponent and so that $x$ is in the numerator.	
\begin{enumerate}
\begin{multicols}{2}
	\item $x^{5}x^{-1/2}$\\ \vfill
	
	\item $(x^{5})^{-1/2}$\\ \vfill
	\item $\frac{3x^{-5}}{x^{4}}$\\ \vfill
	\item $\frac{x^{-5}}{3x^{-3}}$\\ \vfill
	\item $\frac{2}{\sqrt[3]{x^2}}$\\ \vfill
	\item $\frac{x}{\sqrt{x}}$\\
\end{multicols}
\end{enumerate}
\newpage
	\item (3.3\#110)Write this expression so that fractions are not necessary.\\ 
	
	$x^3-\frac{2}{\sqrt{x}}+\frac{1}{x}$\\ \vfill
	\item (3.3\#113)Expand and simplify this expression so that no $x$'s are in the denominator.\\ 
	
	$x^3\left(\frac{3}{x}-\frac{1}{5x^3}+\frac{2}{x^4}\right)$\\ \vfill
	
	\item (3.3\#115) Write this expression so that fractions are not necessary.\\ 
	
	$\frac{3-x^2+x^3}{x^2}$\\ \vfill
	
	\item (Write this expression so that fractions are not necessary) \\$(\frac{x^2}{x^4+1})^4$\\ 
	\vfill
	\item (Write this expression with a single exponent on $x$, if possible.) \\
	
	$\sqrt{16x^3}$\\ 
	\vfill
	\item (Write this expression with a single exponent on $x$, if possible.) \\
	
	$\frac{3x}{\sqrt[3]{8x}}$\\ 
	\vfill
	\item (Write this expression with a single exponent on $x$, if possible.) \\
	
	$\frac{3x}{\sqrt[3]{8x+27}}$\\ 
	\vfill
	
\newpage
\item Identify which of the following equalities are true and which are false. For the ones that are false, give some reason. For the ones that are true, do the algebraic steps that demonstrate this.
	\begin{enumerate}
	\item $\frac{1}{\sqrt{3x}}=\frac{x^{-1/2}}{\sqrt{3}}$\\ \vfill
	\item $\sqrt{4+x}=2+\sqrt{x}$\\ \vfill
	\item $\sqrt{27x^5}=3\sqrt{3}x^{5/2}$\\ \vfill
	\item $\sqrt{x^2-25}=x-5$ \\ \vfill
	\item $\frac{1}{2x}=2x^{-1}$\\ \vfill
	\end{enumerate}
	
Manipulating Fractions
\item Identify which of the following equalities are true and which are false. For the ones that are false, give some reason. 
	\begin{enumerate}
	\item $\frac{a+b}{c+d}=\frac{a}{c+d}+\frac{b}{c+d}$\\ \vfill
	\item $\frac{a+b}{c+d}=\frac{a+b}{c}+\frac{a+b}{d}$\\ \vfill
	\item $\frac{a+b}{c+d}=(a+b)(c+d)^{-1}$\\ \vfill
	\item (3.3 \#115) $\frac{x+x^2}{2x}=\frac{1+x^2}{2}$ \\ \vfill
	\item (3.3 \#114) $\frac{t^2-2t-\pi}{8}=\frac{1}{8}t^2-\frac{1}{4}t-\frac{\pi}{8}$ \\ \vfill
	\item (3.3 \#115) $\frac{1+3x-2x^3}{2x^3}=\frac{1}{2}x^{-3}+\frac{3x^{-2}}{2}-1$ \\ \vfill

	\item $\frac{x}{x-1}=x(x^{-1}-1)=1-x$\\ \vfill
	\end{enumerate}
\end{enumerate}
\end{document}
Bonus Questions
\item Are the expressions in  1.e. and 1.f. exactly the same?
\vfill
\item Think of everyday examples of functions with the properties below. You must state explicitly what the $x$-value represents, what the $f(x)$ value represents, and what the units are of both. The more interesting the better. See one example below. Then state the units of the derivative and what it would measure.
	\begin{enumerate}
	\item a function that is always increasing on its domain
	\vfill
	\item a function that is always decreasing on its domain
	\vfill
	\item a function that is increasing and decreasing on its domain\\
	
	SAMPLE EXAMPLE: A kid is jumping on a trampoline. $f(x)$ measures the height of the top of the kid's head over time. So $x$ is time measured in seconds, $f(x)$ is height of above ground-level measured in feet. Then $f'(x)$ would be the velocity of the kid's head in feet per second.
	\vfill
	\item a function that is discontinuous at some points in its domain.
	\vfill
	\end{enumerate}
\end{enumerate}
\end{document}

