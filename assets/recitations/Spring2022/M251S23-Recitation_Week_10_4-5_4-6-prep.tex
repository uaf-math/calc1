\documentclass[12pt]{article}

% Layout.
\usepackage[top=1in, bottom=0.75in, left=1in, right=1in, headheight=1in, headsep=6pt]{geometry}

% Fonts.
\usepackage{mathptmx}
\usepackage[scaled=0.86]{helvet}
\renewcommand{\emph}[1]{\textsf{\textbf{#1}}}

% TiKZ.
\usepackage{tikz, pgfplots}
\usetikzlibrary{calc}
\pgfplotsset{my style/.append style={axis x line=middle, axis y line=
middle, xlabel={$x$}, ylabel={$y$}, axis equal }}

% Misc packages.
\usepackage{amsmath,amssymb,latexsym}
\usepackage{graphicx}
\usepackage{array}
\usepackage{xcolor}
\usepackage{multicol}

% Commands to set various header/footer components.
\makeatletter
\def\doctitle#1{\gdef\@doctitle{#1}}
\doctitle{Use {\tt\textbackslash doctitle\{MY LABEL\}}.}
\def\docdate#1{\gdef\@docdate{#1}}
\docdate{Use {\tt\textbackslash docdate\{MY DATE\}}.}
\def\doccourse#1{\gdef\@doccourse{#1}}
\let\@doccourse\@empty
\def\docscoring#1{\gdef\@docscoring{#1}}
\let\@docscoring\@empty
\def\docversion#1{\gdef\@docversion{#1}}
\let\@docversion\@empty
\makeatother

% Headers and footers layout.
\makeatletter
\usepackage{fancyhdr}
\pagestyle{fancy}
\fancyhf{} % Clears all headers/footers.
\lhead{\baselineskip 30pt
\emph{\@doctitle\hfill\@docdate}
\ifnum \value{page} > 1\relax\else\\
\emph{Name: \rule{3.5in}{1pt}\ \hfill
%Class (circle): \ \  Sync. \hfill Online%\@docscoring
}
\fi}
\rfoot{\emph{\@docversion}}
\lfoot{\emph{\@doccourse}}
\cfoot{\emph{\thepage}}
\renewcommand{\headrulewidth}{0pt}%
\makeatother

% Paragraph spacing
\parindent 0pt
\parskip 6pt plus 1pt

% A problem is a section-like command. Use \problem{5} to
% start a problem worth 5 points.
\newcounter{probcount}
\newcounter{subprobcount}
\setcounter{probcount}{0}
\newcommand{\problem}[1]{%
\par
\addvspace{4pt}%
\setcounter{subprobcount}{0}%
\stepcounter{probcount}%
\makebox[0pt][r]{\emph{\arabic{probcount}.}\hskip1ex}\emph{[#1 points]}\hskip1ex}
\newcommand{\thesubproblem}{\emph{\alph{subprobcount}.}}

% Subproblems are an enumerate-like environment with a consistent
% numbering scheme. 
% Use \begin{subproblems}\item...\item...\end{subproblems}
\newenvironment{subproblems}{%
\begin{enumerate}%
\setcounter{enumi}{\value{subprobcount}}%
\renewcommand{\theenumi}{\emph{\alph{enumi}}}}%
{\setcounter{subprobcount}{\value{enumi}}\end{enumerate}}

% Blanks for answers in normal and math mode.
\newcommand{\blank}[1]{\rule{#1}{0.75pt}}
\newcommand{\mblank}[1]{\underline{\hspace{#1}}}
\def\emptybox(#1,#2){\framebox{\parbox[c][#2]{#1}{\rule{0pt}{0pt}}}}

% Misc.
\renewcommand{\d}{\displaystyle}
\newcommand{\ds}{\displaystyle}
\def\bc{\begin{center}}
\def\ec{\end{center}}


\doctitle{Math 251: Section 4.5 \& 4.6 Homework Help}
\docdate{Recitation Week 9}
\doccourse{UAF Calculus I}
%\docversion{v-practice}
%\docscoring{\blank{0.8in} / 12}

\begin{document}
\addtolength\itemsep{-1mm}

Solve the following equations for $x$ or state that none exist.\\
\begin{multicols}{2}
\begin{enumerate}
\item $5e^x-2=0$\\

\item $5e^x+4=0$\\

\item $5 \ln(x)-6=0$\\

\item $5 \ln(x)+7=0$\\
\end{enumerate}
\end{multicols}

This page contains information and techniques you will need for Sections 4.5 and 4.6.
\begin{enumerate}
\item Write in your own words how to find the critical numbers of a function $f(x)$ and why they are important.
\vspace{1in}
\item Draw a graph of a function $f(x)$ with domain $(-\infty, \infty)$  such that \\
(i) $f'(a)=f'(b)=0$ and $f'(c)$ is undefined,\\
and\\
(i) $f$ has a local minimum at $x=a$, a local maximum at $x=c$ and neither at $x=b.$
\vfill
\item Draw a graph of a function $f(x)$with domain $(-\infty, \infty)$ such that \\
\begin{multicols}{2}
\begin{enumerate}
\item $f(x)<0$ and $f'(x) >0.$
\item $f'(x)<0$ and $f''(x) >0.$
\end{enumerate}
\end{multicols}
\vfill
\newpage
\item For each function below, find (a) its domain and (b) all its critical points.
	\begin{enumerate}
	\item $f(x)=x^3-2x^2$
	\vfill
	\item $f(x)=x^{1/5}$
	\vfill
	\item $f(x)=\arctan(x)$
	\vfill
	\item $f(x)=\frac{x^2}{x^2-4}$ (Note: $f'(x)=\frac{-8x}{(x^2-4)^2}.$)
	\vfill
	\item $f(x)=e^{(1-x)^2}$
	\vfill
	\item $f(x)=\sqrt{x^2-4}$
	\vfill
	\end{enumerate}
\newpage
\item For each derivative below, determine the intervals for which that derivative is positive and negative.
\begin{enumerate}
\item $f'(x)=x^{-4/5}$
\vfill
\item $y''=\frac{8(3x^2+4)}{(x^2-4)^3}$
\vfill
\item $g'(x)=3x^2e^{2x}+2x^3e^{2x}$
\vfill
\end{enumerate}
\newpage
\item Write a formula for a function $f(x)$ such that $f(x)$ has asymptotes $x=1$, $x=4$ and $y=0.$
\vspace{1in}

\item Give an example of a graph with two different horizontal asymptotes.
\vfill
\item Evaluate each limit below. \\
\begin{multicols}{2}

	\begin{enumerate}
	\item $\displaystyle \lim_{x \to 2^+} \frac{5}{x-2}$\\ 
	
	\vfill
	\item $\displaystyle \lim_{x \to 2^-} \frac{5}{x-2}$\\ 
	
	\vfill
	\item $\displaystyle \lim_{x \to 2} \frac{5}{x-2}$\\ 
	
	\vfill
	
	\columnbreak
	
	\item $\displaystyle \lim_{x \to \infty} \frac{5}{x-2}$\\ 
	
	\vfill
	\item $\displaystyle \lim_{x \to -\infty} \frac{5}{x-2}$\\ 
	
	\vfill
	\item $\displaystyle \lim_{x \to \infty} \left(8+\frac{5}{x-2}\right)$\\ 
	
	\vfill
	\item $\displaystyle \lim_{x \to \infty} \left(x+\frac{5}{x-2}\right)$\\
	
	 \vfill
	\end{enumerate}
\end{multicols}	
\end{enumerate}
\end{document}

