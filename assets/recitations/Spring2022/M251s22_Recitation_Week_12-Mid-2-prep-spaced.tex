\documentclass[12pt]{article}

% Layout.
\usepackage[top=1in, bottom=0.75in, left=1in, right=1in, headheight=1in, headsep=6pt]{geometry}

% Fonts.
\usepackage{mathptmx}
\usepackage[scaled=0.86]{helvet}
\renewcommand{\emph}[1]{\textsf{\textbf{#1}}}

% TiKZ.
\usepackage{tikz, pgfplots}
\usetikzlibrary{calc}
\pgfplotsset{my style/.append style={axis x line=middle, axis y line=
middle, xlabel={$x$}, ylabel={$y$}, axis equal }}

% Misc packages.
\usepackage{amsmath,amssymb,latexsym}
\usepackage{graphicx}
\usepackage{array}
\usepackage{xcolor}
\usepackage{multicol,wrapfig,txfonts}

% Commands to set various header/footer components.
\makeatletter
\def\doctitle#1{\gdef\@doctitle{#1}}
\doctitle{Use {\tt\textbackslash doctitle\{MY LABEL\}}.}
\def\docdate#1{\gdef\@docdate{#1}}
\docdate{Use {\tt\textbackslash docdate\{MY DATE\}}.}
\def\doccourse#1{\gdef\@doccourse{#1}}
\let\@doccourse\@empty
\def\docscoring#1{\gdef\@docscoring{#1}}
\let\@docscoring\@empty
\def\docversion#1{\gdef\@docversion{#1}}
\let\@docversion\@empty
\makeatother

% Headers and footers layout.
\makeatletter
\usepackage{fancyhdr}
\pagestyle{fancy}
\fancyhf{} % Clears all headers/footers.
\lhead{\baselineskip 30pt
\emph{\@doctitle\hfill\@docdate}}
%\ifnum \value{page} > 1\relax\else\\
%\emph{\quad}{1pt}\ \hfill
%\fi}
%Class (circle): \ \  Sync. \hfill Online%\@docscoring

\rfoot{\emph{\@docversion}}
\lfoot{\emph{\@doccourse}}
\cfoot{\emph{\thepage}}
\renewcommand{\headrulewidth}{0pt}%
\makeatother

% Paragraph spacing
\parindent 0pt
\parskip 6pt plus 1pt

% A problem is a section-like command. Use \problem{5} to
% start a problem worth 5 points.
\newcounter{probcount}
\newcounter{subprobcount}
\setcounter{probcount}{0}
\newcommand{\problem}[1]{%
\par
\addvspace{4pt}%
\setcounter{subprobcount}{0}%
\stepcounter{probcount}%
\makebox[0pt][r]{\emph{\arabic{probcount}.}\hskip1ex}\emph{[#1 points]}\hskip1ex}
\newcommand{\thesubproblem}{\emph{\alph{subprobcount}.}}

\pgfplotsset{my style/.append style={axis x line=middle, axis y line=middle, 
xlabel=$x$,ylabel=$y$,
every axis x label/.style={
    at={(ticklabel* cs:1)},
    anchor=west,
},
every axis y label/.style={
    at={(ticklabel* cs:1)},
    anchor=south,
},}}


% Subproblems are an enumerate-like environment with a consistent
% numbering scheme. 
% Use \begin{subproblems}\item...\item...\end{subproblems}
\newenvironment{subproblems}{%
\begin{enumerate}%
\setcounter{enumi}{\value{subprobcount}}%
\renewcommand{\theenumi}{\emph{\alph{enumi}}}}%
{\setcounter{subprobcount}{\value{enumi}}\end{enumerate}}

% Blanks for answers in normal and math mode.
\newcommand{\blank}[1]{\rule{#1}{0.75pt}}
\newcommand{\mblank}[1]{\underline{\hspace{#1}}}
\def\emptybox(#1,#2){\framebox{\parbox[c][#2]{#1}{\rule{0pt}{0pt}}}}

% Misc.
\renewcommand{\d}{\displaystyle}
\newcommand{\ds}{\displaystyle}
\def\bc{\begin{center}}
\def\ec{\end{center}}


\doctitle{Math 251: Mid 2 Prep}
\docdate{Recitation Week 12}
\doccourse{UAF Calculus I}
%\docversion{v-practice}
%\docscoring{\blank{0.8in} / 12}

\begin{document}
\begin{enumerate}
\item Sketch a graph that satisfies all of the conditions:
\begin{align*}
  &\text{domain } f = (-\infty,\infty), \hspace{5.0in}\\
  &f(3)=-1, \quad f'(3)=0  \\
  & f'(x)<0 \text{ when } x<3,f'(x)>0 \text{ when } x > 3,  \\
  &f''(x)<0 \text{ when } x<0, \quad f''(x)>0 \text{ when } x>0 \\
  &\displaystyle{\lim_{x \to -\infty} f(x)=4}\\
\end{align*}
\newpage
\item Evaluate the following limits.
\begin{enumerate}
\item $\lim_{x \to 0} \frac{\sin(x^2)}{x^2}$
\vfill
\item $\lim_{x \to 0^+} \sqrt{x} \ln(x)$
\vfill
\item $\lim_{x \to 0^+} \left(1+\sin(x) \right)^{\frac{1}{x}}$
\vfill
\end{enumerate}
\newpage

\item A function and its first and second derivatives are given below.
$$f(x)=x^{5/3}-5x^{2/3}, \quad \quad f'(x)=\frac{5x-10}{3x^{1/3}}, \quad \quad f''(x)=\frac{10x+10}{9x^{4/3}}$$
\begin{enumerate}
\item Identify any critical points of $f(x).$
\vfill
\item Find the intervals of increase and decrease, and identify any local maximum or minimum values. Your answer should have the form: ``$f(x)$ has a maximum of \underline{\hspace{.3in}}  at \underline{\hspace{.3in}}" or ``$f(x)$ has no maxima."
\vfill
\item Find the intervals of concavity and any inflection points.
\vfill
\end{enumerate}
\newpage
\item The graph of the function $\displaystyle f(x) = \sqrt{\frac{x}{2}+1}$ is shown.

\begin{tikzpicture}[scale=1.0]
\begin{axis}[scale=1.0, axis equal, my style,
xtick={-2,-1,...,3}, ytick={0,1,...,3},
xmin=-2.5, xmax=3.5, ymin=0, ymax=2.2,
minor y tick num=1, minor x tick num=1,
samples=200]
%
\addplot[domain=-2:4,-,thick] {sqrt(0.5*x+1)};
\end{axis}
\end{tikzpicture}

\begin{enumerate}
\item  Let $G(x)$ be the square of the distance from the origin to a point on the graph of $y=f(x)$.  Write an expression for $G(x)$.
\vfill

\item Use the expression for $G(x)$ to find the closest point on the graph $y=f(x)$ to the origin.
\vfill

\item  Show your result by adding a point, with coordinates, to the graph.
\end{enumerate}
\newpage
\item A ship passes a lighthouse at 3:30pm, sailing to the east at 5 mph, while another ship sailing due south at 6 mph passes the same point half an hour later.  How fast will the distance between the ships be increasing at 5:30pm?
\vfill
\item Use differentials to estimate the amount of paint needed to apply a coat of paint 0.1 cm thick to a hemispherical dome with radius 10 m. Give your final answer with proper units. (Note the volume of a sphere is $V=\frac{4}{3} \pi r^3.$)
\vfill
\newpage
\item Find the linearization of $f(x)=e^x$ at $a=0$ and use it to estimate $e^{0.1}.$ Express your answer as simplified fraction or decimal.
\vfill
\item Solve the initial value problem. If the velocity of an object is given by $v(t)=e^{t} +t,$ find the position of the object assuming that the initial position of the object is $0.$ (That is, $s(0)=0.$)
\vfill
\item Evaluate the indefinite integral below. Give the most complete answer.
$\int (5\sec^2(x) + \frac{1}{x^5}) \: dx$.
\vfill
\newpage
\item Estimate the area under the curve $f(x)=x^3$ and above the $x$-axis on the interval $[0,2]$ using 4 rectangles and right-hand endpoints. (i.e. Find $R_4.$) Draw a picture to illustrate your computation.
\vfill
\item Determine the absolute maximum and absolute minimum of $f(t)=\frac{{t}}{2+2t^2}$ on the interval $[0,2].$
\vfill
\end{enumerate}
\end{document}