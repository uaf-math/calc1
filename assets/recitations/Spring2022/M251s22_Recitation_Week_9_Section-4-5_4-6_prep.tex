\documentclass[12pt]{article}

% Layout.
\usepackage[top=1in, bottom=0.75in, left=1in, right=1in, headheight=1in, headsep=6pt]{geometry}

% Fonts.
\usepackage{mathptmx}
\usepackage[scaled=0.86]{helvet}
\renewcommand{\emph}[1]{\textsf{\textbf{#1}}}

% TiKZ.
\usepackage{tikz, pgfplots}
\usetikzlibrary{calc}
\pgfplotsset{my style/.append style={axis x line=middle, axis y line=
middle, xlabel={$x$}, ylabel={$y$}, axis equal }}

% Misc packages.
\usepackage{amsmath,amssymb,latexsym}
\usepackage{graphicx}
\usepackage{array}
\usepackage{xcolor}
\usepackage{multicol}

% Commands to set various header/footer components.
\makeatletter
\def\doctitle#1{\gdef\@doctitle{#1}}
\doctitle{Use {\tt\textbackslash doctitle\{MY LABEL\}}.}
\def\docdate#1{\gdef\@docdate{#1}}
\docdate{Use {\tt\textbackslash docdate\{MY DATE\}}.}
\def\doccourse#1{\gdef\@doccourse{#1}}
\let\@doccourse\@empty
\def\docscoring#1{\gdef\@docscoring{#1}}
\let\@docscoring\@empty
\def\docversion#1{\gdef\@docversion{#1}}
\let\@docversion\@empty
\makeatother

% Headers and footers layout.
\makeatletter
\usepackage{fancyhdr}
\pagestyle{fancy}
\fancyhf{} % Clears all headers/footers.
\lhead{\baselineskip 30pt
\emph{\@doctitle\hfill\@docdate}
\ifnum \value{page} > 1\relax\else\\
\emph{Name: \rule{3.5in}{1pt}\ \hfill
%Class (circle): \ \  Sync. \hfill Online%\@docscoring
}
\fi}
\rfoot{\emph{\@docversion}}
\lfoot{\emph{\@doccourse}}
\cfoot{\emph{\thepage}}
\renewcommand{\headrulewidth}{0pt}%
\makeatother

% Paragraph spacing
\parindent 0pt
\parskip 6pt plus 1pt

% A problem is a section-like command. Use \problem{5} to
% start a problem worth 5 points.
\newcounter{probcount}
\newcounter{subprobcount}
\setcounter{probcount}{0}
\newcommand{\problem}[1]{%
\par
\addvspace{4pt}%
\setcounter{subprobcount}{0}%
\stepcounter{probcount}%
\makebox[0pt][r]{\emph{\arabic{probcount}.}\hskip1ex}\emph{[#1 points]}\hskip1ex}
\newcommand{\thesubproblem}{\emph{\alph{subprobcount}.}}

% Subproblems are an enumerate-like environment with a consistent
% numbering scheme. 
% Use \begin{subproblems}\item...\item...\end{subproblems}
\newenvironment{subproblems}{%
\begin{enumerate}%
\setcounter{enumi}{\value{subprobcount}}%
\renewcommand{\theenumi}{\emph{\alph{enumi}}}}%
{\setcounter{subprobcount}{\value{enumi}}\end{enumerate}}

% Blanks for answers in normal and math mode.
\newcommand{\blank}[1]{\rule{#1}{0.75pt}}
\newcommand{\mblank}[1]{\underline{\hspace{#1}}}
\def\emptybox(#1,#2){\framebox{\parbox[c][#2]{#1}{\rule{0pt}{0pt}}}}

% Misc.
\renewcommand{\d}{\displaystyle}
\newcommand{\ds}{\displaystyle}
\def\bc{\begin{center}}
\def\ec{\end{center}}


\doctitle{Math 251: Section 4.5 \& 4.6 Homework Help}
\docdate{Recitation Week 9}
\doccourse{UAF Calculus I}
%\docversion{v-practice}
%\docscoring{\blank{0.8in} / 12}

\begin{document}
\addtolength\itemsep{-1mm}

This page contains information and techniques you will need for Sections 4.5 and 4.6.\\
\begin{enumerate}
\item What is a critical point? How do you find them?
\vspace{.5in}
\item What is the importance of critical points? What are they good for?
\vspace{.5in}
\item Draw a graph of a function $f(x)$ with three critical points (at $x=a$, $x=b$ and $x=c$ such that \\
(i) $f'(a)=f'(b)=0$ and $f'(c)$ is undefined,\\
and\\
(i) $f$ has a local minimum at $x=a$, a local maximum at $x=c$ and neither at $x=b.$
\vfill
\item Draw a graph of a function $f(x)$ such that $f(x)<0$ and $f'(x) >0.$
\vfill
\newpage
\item For each function below, find (a) its domain and (b) all its critical points.
	\begin{enumerate}
	\item $f(x)=x^3-2x^2$
	\vfill
	\item $f(x)=x^{1/3}$
	\vfill
	\item $f(x)=\arctan(x)$
	\vfill
	\item $f(x)=\frac{x^2}{x^2-4}$ (Note: $f'(x)=\frac{-8x}{x^2-4}.$)
	\vfill
	\item $f(x)=e^{-x^2}$
	\vfill
	\item $f(x)=\sqrt{x^2-4}$
	\vfill
	\end{enumerate}
\newpage
\item For $f(x)=x^{1/3}$, you should have gotten $x=0$ as the one and only critical point. Explain (in whatever method works for you) \emph{HOW} you will determine the \emph{sign} of $f'(x)$? Repeat the process for $f''(x).$
\vfill
\item Assume the expression below is the second derivative of some function. How will you determine where $y''$ is positive or negative? Are some parts of the expression that are more important than other parts? What would have happened in the term in the denominator was raised to the \emph{fourth} power instead of the third power?\\
$$y''=\frac{8(3x^2+4)}{(x^2-4)^3}$$
\vfill
\item Sketch a graph with two vertical asymptotes and one horizontal asymptote.
\vfill
\item Write a formula for a function $f(x)$ such that $f(x)$ has asymptotes $x=1$, $x=4$ and $y=2.$
\vspace{1in}
\newpage
\item Can a function have more than one \emph{horizontal} asymptote? Explain.
\vfill
\item Can a function with a horizontal asymptote ever cross that asymptote? Explain.
\vfill
\item Evaluate each limit below and \emph{justify} your answer. Your justification can be in words, using a graph, and/or numerical (or all of the above). 
	\begin{enumerate}
	\item $\displaystyle \lim_{x \to 2^+} \frac{5}{x-2}$\\ \vfill
	\item $\displaystyle \lim_{x \to 2^-} \frac{5}{x-2}$\\ \vfill
	\item $\displaystyle \lim_{x \to 2} \frac{5}{x-2}$\\ \vfill
	
	\item $\displaystyle \lim_{x \to \infty} \frac{5}{x-2}$\\ \vfill
	\item $\displaystyle \lim_{x \to -\infty} \frac{5}{x-2}$\\ \vfill
	\item $\displaystyle \lim_{x \to \infty} \left(8+\frac{5}{x-2}\right)$\\ \vfill
	\item $\displaystyle \lim_{x \to \infty} \left(x+\frac{5}{x-2}\right)$\\ \vfill
	\end{enumerate}	
\end{enumerate}
\end{document}

\vfill
\item Graph $y=e^x$ and $y=e^{-x}$ on the same set of axes. Label the points associated with $x=-1,$ $x=0,$ and $x=1$ on both graphs.
\vfill
\item On separate axes, graph $y=x^2$, $y=\sqrt{x}$, $y=1/x^2.$ 
\vfill
\item For all of the graphs above, describe the \emph{end behavior} of the graphs. This means, describe what happens for really large $x$ values (think $10^{100}$ and really small $x$-values (think $-10^{100}.$ Note ``small" means toward negative infinity, not close to zero.

\end{enumerate}
\end{document}