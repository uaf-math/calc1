\documentclass[12pt]{article}

% Layout.
\usepackage[top=1in, bottom=0.75in, left=1in, right=1in, headheight=1in, headsep=6pt]{geometry}

% Fonts.
\usepackage{mathptmx}
\usepackage[scaled=0.86]{helvet}
\renewcommand{\emph}[1]{\textsf{\textbf{#1}}}

% TiKZ.
\usepackage{tikz, pgfplots}
\usetikzlibrary{calc}
\pgfplotsset{my style/.append style={axis x line=middle, axis y line=
middle, xlabel={$x$}, ylabel={$y$}, axis equal }}

% Misc packages.
\usepackage{amsmath,amssymb,latexsym}
\usepackage{graphicx}
\usepackage{array}
\usepackage{xcolor}
\usepackage{multicol}

% Commands to set various header/footer components.
\makeatletter
\def\doctitle#1{\gdef\@doctitle{#1}}
\doctitle{Use {\tt\textbackslash doctitle\{MY LABEL\}}.}
\def\docdate#1{\gdef\@docdate{#1}}
\docdate{Use {\tt\textbackslash docdate\{MY DATE\}}.}
\def\doccourse#1{\gdef\@doccourse{#1}}
\let\@doccourse\@empty
\def\docscoring#1{\gdef\@docscoring{#1}}
\let\@docscoring\@empty
\def\docversion#1{\gdef\@docversion{#1}}
\let\@docversion\@empty
\makeatother

% Headers and footers layout.
\makeatletter
\usepackage{fancyhdr}
\pagestyle{fancy}
\fancyhf{} % Clears all headers/footers.
\lhead{\baselineskip 30pt
\emph{\@doctitle\hfill\@docdate}}
%\ifnum \value{page} > 1\relax\else\\
%\emph{\quad}{1pt}\ \hfill
%\fi}
%Class (circle): \ \  Sync. \hfill Online%\@docscoring

\rfoot{\emph{\@docversion}}
\lfoot{\emph{\@doccourse}}
\cfoot{\emph{\thepage}}
\renewcommand{\headrulewidth}{0pt}%
\makeatother

% Paragraph spacing
\parindent 0pt
\parskip 6pt plus 1pt

% A problem is a section-like command. Use \problem{5} to
% start a problem worth 5 points.
\newcounter{probcount}
\newcounter{subprobcount}
\setcounter{probcount}{0}
\newcommand{\problem}[1]{%
\par
\addvspace{4pt}%
\setcounter{subprobcount}{0}%
\stepcounter{probcount}%
\makebox[0pt][r]{\emph{\arabic{probcount}.}\hskip1ex}\emph{[#1 points]}\hskip1ex}
\newcommand{\thesubproblem}{\emph{\alph{subprobcount}.}}

% Subproblems are an enumerate-like environment with a consistent
% numbering scheme. 
% Use \begin{subproblems}\item...\item...\end{subproblems}
\newenvironment{subproblems}{%
\begin{enumerate}%
\setcounter{enumi}{\value{subprobcount}}%
\renewcommand{\theenumi}{\emph{\alph{enumi}}}}%
{\setcounter{subprobcount}{\value{enumi}}\end{enumerate}}

% Blanks for answers in normal and math mode.
\newcommand{\blank}[1]{\rule{#1}{0.75pt}}
\newcommand{\mblank}[1]{\underline{\hspace{#1}}}
\def\emptybox(#1,#2){\framebox{\parbox[c][#2]{#1}{\rule{0pt}{0pt}}}}

% Misc.
\renewcommand{\d}{\displaystyle}
\newcommand{\ds}{\displaystyle}
\def\bc{\begin{center}}
\def\ec{\end{center}}


\doctitle{Math 251: 4.8, 4.10, 5.1}
\docdate{Recitation Week 11}
\doccourse{UAF Calculus I}
%\docversion{v-practice}
%\docscoring{\blank{0.8in} / 12}

\begin{document}
\begin{enumerate}
\item Give an explanation in your own words for why $x=\frac{1}{x^{-1}}.$\\
\vspace{.5in}

\item Simplify $\ds \frac{5 \left(\frac{1}{x}\right)}{x^{-3}}$\\
\vspace{.5in}

\item Write in your own words how you know when to write $\lim_{x \to \infty}$ and when to stop writing it. Then evaluate the following limits being obsessive about your use of notation. Note that you must give an \textbf{algebraic} justification for your answer, possibly with the use of L'H\^{o}pital's Rule.\\
\vspace{1in}
\begin{enumerate}
	\item $\displaystyle{\lim_{x \to \infty} \frac{\ln (x)}{\sqrt[10]{x}}}$\\
	\vfill
	\item $\displaystyle{\lim_{x \to \infty} \frac{\sqrt{3x^2-1}}{3-x}}$\\
	\vfill
\end{enumerate}
\item What do the limits above imply about the graphs $f(x) = \frac{\ln (x)}{\sqrt[10]{x}}$ and $g(x) = \frac{\sqrt{3x^2-1}}{3-x}$?
\vfill
\item Do either $f(x)$ or $g(x)$ have vertical asymptotes? Justify your answer.
\vfill
\newpage
\item Determine if the following statements are True or False. Give an explanation. Bonus points for the most succinct explanation.
	\begin{enumerate}
	\item $\displaystyle{\int h(x) j(x) \: dx = \left(\int h(x) \: dx \right) \left(\int j(x) \: dx \right)}$
	\vfill
	\item $\displaystyle{\int h(x) +j(x) \: dx = \left(\int h(x) \: dx \right) +\left(\int j(x) \: dx \right)}$
	\vfill
	\item $\displaystyle{\int \frac{h(x)}{j(x)} \: dx = \frac{\int h(x) \: dx}{ \int j(x) \: dx }}$
	\vfill
	\item $k$ is a constant, $\displaystyle{\int kh(x) \: dx = k\int h(x) \: dx}$
	\vfill
	\item $\displaystyle{\int (h(x))^2 \: dx = \frac{1}{3}(h(x))^3 +C}$
	\vfill
	\end{enumerate}
\item Evaluate $\int (x+2)^2 \: dx$
\vfill
\item Convert 60 miles per hour into feet per second. 
\vfill
\item Write the equation for the top-half of the circle of radius 4 centered at $x=10$ on the $x$-axis.
\vfill
\end{enumerate}

\end{document}