\documentclass[12pt]{article}

% Layout.
\usepackage[top=1in, bottom=0.75in, left=1in, right=1in, headheight=1in, headsep=6pt]{geometry}

% Fonts.
\usepackage{mathptmx}
\usepackage[scaled=0.86]{helvet}
\renewcommand{\emph}[1]{\textsf{\textbf{#1}}}

% TiKZ.
\usepackage{tikz, pgfplots}
\usetikzlibrary{calc}
\pgfplotsset{my style/.append style={axis x line=middle, axis y line=
middle, xlabel={$x$}, ylabel={$y$}, axis equal }}

% Misc packages.
\usepackage{amsmath,amssymb,latexsym}
\usepackage{graphicx}
\usepackage{array}
\usepackage{xcolor}
\usepackage{multicol,wrapfig,txfonts}

% Commands to set various header/footer components.
\makeatletter
\def\doctitle#1{\gdef\@doctitle{#1}}
\doctitle{Use {\tt\textbackslash doctitle\{MY LABEL\}}.}
\def\docdate#1{\gdef\@docdate{#1}}
\docdate{Use {\tt\textbackslash docdate\{MY DATE\}}.}
\def\doccourse#1{\gdef\@doccourse{#1}}
\let\@doccourse\@empty
\def\docscoring#1{\gdef\@docscoring{#1}}
\let\@docscoring\@empty
\def\docversion#1{\gdef\@docversion{#1}}
\let\@docversion\@empty
\makeatother

% Headers and footers layout.
\makeatletter
\usepackage{fancyhdr}
\pagestyle{fancy}
\fancyhf{} % Clears all headers/footers.
\lhead{\baselineskip 30pt
\emph{\@doctitle\hfill\@docdate}}
%\ifnum \value{page} > 1\relax\else\\
%\emph{\quad}{1pt}\ \hfill
%\fi}
%Class (circle): \ \  Sync. \hfill Online%\@docscoring

\rfoot{\emph{\@docversion}}
\lfoot{\emph{\@doccourse}}
\cfoot{\emph{\thepage}}
\renewcommand{\headrulewidth}{0pt}%
\makeatother

% Paragraph spacing
\parindent 0pt
\parskip 6pt plus 1pt

% A problem is a section-like command. Use \problem{5} to
% start a problem worth 5 points.
\newcounter{probcount}
\newcounter{subprobcount}
\setcounter{probcount}{0}
\newcommand{\problem}[1]{%
\par
\addvspace{4pt}%
\setcounter{subprobcount}{0}%
\stepcounter{probcount}%
\makebox[0pt][r]{\emph{\arabic{probcount}.}\hskip1ex}\emph{[#1 points]}\hskip1ex}
\newcommand{\thesubproblem}{\emph{\alph{subprobcount}.}}

\pgfplotsset{my style/.append style={axis x line=middle, axis y line=middle, 
xlabel=$x$,ylabel=$y$,
every axis x label/.style={
    at={(ticklabel* cs:1)},
    anchor=west,
},
every axis y label/.style={
    at={(ticklabel* cs:1)},
    anchor=south,
},}}


% Subproblems are an enumerate-like environment with a consistent
% numbering scheme. 
% Use \begin{subproblems}\item...\item...\end{subproblems}
\newenvironment{subproblems}{%
\begin{enumerate}%
\setcounter{enumi}{\value{subprobcount}}%
\renewcommand{\theenumi}{\emph{\alph{enumi}}}}%
{\setcounter{subprobcount}{\value{enumi}}\end{enumerate}}

% Blanks for answers in normal and math mode.
\newcommand{\blank}[1]{\rule{#1}{0.75pt}}
\newcommand{\mblank}[1]{\underline{\hspace{#1}}}
\def\emptybox(#1,#2){\framebox{\parbox[c][#2]{#1}{\rule{0pt}{0pt}}}}

% Misc.
\renewcommand{\d}{\displaystyle}
\newcommand{\ds}{\displaystyle}
\def\bc{\begin{center}}
\def\ec{\end{center}}


\doctitle{Math 251: Integration Proficiency Practice}
\docdate{Recitation Week 14}
\doccourse{UAF Calculus I}
%\docversion{v-practice}
%\docscoring{\blank{0.8in} / 12}

\begin{document}
The goal of this recitation is to prepare students for the Integral Proficiency.\\

BEFORE you hand out the practice proficiency, do items (1) - (3) below:\\

\begin{enumerate}
\item Remind students of the schedule / details.
	\begin{enumerate}
	\item Given Wednesday (1 Dec)  in class. Returned Thursday (2 Dec). Retakes Friday (3 Dec) 4:45-5:45 or 6:00-7:00 or by appointment (talk with your teacher).
	\item There will be 12 integrals.
	\item 10/12 is a pass and you are done. Less than 10/12, retakes on Friday. 
	\item On the retake, your score is capped at 10. (Reward students who pass on the first go.)
	\item If you complete a retake, you get the highest of two scores (with second capped at 10). If you do not take the retake, you get the average of two scores.
	\item Integral Proficiency is 7.5\% or grade or one-half a midterm.
	\end{enumerate}
\item Ask students what \emph{they} think they should expect on this. The goal here is two-fold. We want students to recognize that -- having looked at earlier proficiencies -- they have the tools to anticipate what to expect and should be thinking this way about this and all math classes. Second, we want them to put into words explicit guidelines and strategies. By a combination of volunteers and pointed questions, we want the following observations on the board in writing.
	\begin{enumerate}
	\item There will be between 1 and 3 definite integrals. The rest will be indefinite integrals.
	\item Somewhere will be an integrand that integrates as arcsine or arctangent.
	\item Somewhere will be an integrand that integrates as secant, another as tangent. 
	\item You'll have to integrate sine, cosine and $e^x$
	\item Somewhere you will integrate $1/x$ as $\ln |x|$
	\item There will be $u$-substitution problems and at least one of each of the following types:
		\begin{enumerate}
		\item $u$ as something raised to a power 
		\item $u$ as denominator -- which will integrate as natural log
		\item $u$ as exponent of $e$
		\item $u$ inside a trig function
		\end{enumerate} 
	\item Expect a sophisticated $u$-substitution. (Where you have to solve your choice of $u$ for $x$. See problem (d) on the practice to see what I mean here.)
	\item Some integrands will require some algebraic pre-processing.
	\item You will have to use the power rule on fractional exponents and negative exponents.
	\end{enumerate}
\item Point out to students that this list is a big hint. Point out that they should go looking for these things. (Ex: Where is the sophisticated substitution problem? The one where I pick $u$ to be the exponent of $e$?)
\item Hand out the Proficiency but tell them that before starting the time you are giving them 2 minutes to read the directions and then clarify the expectations. Specifically:
	\begin{enumerate}
	\item While they do not have to simplify definite integrals, they have to make sure their parenthesis are correct. If literally copying their writing into WolframAlpha or Desmos would not produce the correct numerical answer, they will be counted wrong -- obviously!!
	\item They need to get the "+C" in the right place at least once. They will be counted wrong for putting the "+C" in the wrong place.
	\item They do not have to simplify indefinite integrals. 
	\item They do not have to use $u$-substitution if they can do something in their head.
	\item If there is time remaining they should check their answers. Especially consider checking by taking the derivative!
	\end{enumerate}
Now give students 30 minutes to complete the practice. Get complete worked solutions on the board in the last 30 minutes.
\end{enumerate}
\end{document}