\documentclass[12pt]{article}

% Layout.
\usepackage[top=1in, bottom=0.75in, left=1in, right=1in, headheight=1in, headsep=6pt]{geometry}

% Fonts.
\usepackage{mathptmx}
\usepackage[scaled=0.86]{helvet}
\renewcommand{\emph}[1]{\textsf{\textbf{#1}}}

% TiKZ.
\usepackage{tikz, pgfplots}
\usetikzlibrary{calc}
\pgfplotsset{my style/.append style={axis x line=middle, axis y line=
middle, xlabel={$x$}, ylabel={$y$}, axis equal }}

% Misc packages.
\usepackage{amsmath,amssymb,latexsym}
\usepackage{graphicx}
\usepackage{array}
\usepackage{xcolor}
\usepackage{multicol,wrapfig,txfonts}

% Commands to set various header/footer components.
\makeatletter
\def\doctitle#1{\gdef\@doctitle{#1}}
\doctitle{Use {\tt\textbackslash doctitle\{MY LABEL\}}.}
\def\docdate#1{\gdef\@docdate{#1}}
\docdate{Use {\tt\textbackslash docdate\{MY DATE\}}.}
\def\doccourse#1{\gdef\@doccourse{#1}}
\let\@doccourse\@empty
\def\docscoring#1{\gdef\@docscoring{#1}}
\let\@docscoring\@empty
\def\docversion#1{\gdef\@docversion{#1}}
\let\@docversion\@empty
\makeatother

% Headers and footers layout.
\makeatletter
\usepackage{fancyhdr}
\pagestyle{fancy}
\fancyhf{} % Clears all headers/footers.
\lhead{\baselineskip 30pt
\emph{\@doctitle\hfill\@docdate}}
%\ifnum \value{page} > 1\relax\else\\
%\emph{\quad}{1pt}\ \hfill
%\fi}
%Class (circle): \ \  Sync. \hfill Online%\@docscoring

\rfoot{\emph{\@docversion}}
\lfoot{\emph{\@doccourse}}
\cfoot{\emph{\thepage}}
\renewcommand{\headrulewidth}{0pt}%
\makeatother

% Paragraph spacing
\parindent 0pt
\parskip 6pt plus 1pt

% A problem is a section-like command. Use \problem{5} to
% start a problem worth 5 points.
\newcounter{probcount}
\newcounter{subprobcount}
\setcounter{probcount}{0}
\newcommand{\problem}[1]{%
\par
\addvspace{4pt}%
\setcounter{subprobcount}{0}%
\stepcounter{probcount}%
\makebox[0pt][r]{\emph{\arabic{probcount}.}\hskip1ex}\emph{[#1 points]}\hskip1ex}
\newcommand{\thesubproblem}{\emph{\alph{subprobcount}.}}

\pgfplotsset{my style/.append style={axis x line=middle, axis y line=middle, 
xlabel=$x$,ylabel=$y$,
every axis x label/.style={
    at={(ticklabel* cs:1)},
    anchor=west,
},
every axis y label/.style={
    at={(ticklabel* cs:1)},
    anchor=south,
},}}


% Subproblems are an enumerate-like environment with a consistent
% numbering scheme. 
% Use \begin{subproblems}\item...\item...\end{subproblems}
\newenvironment{subproblems}{%
\begin{enumerate}%
\setcounter{enumi}{\value{subprobcount}}%
\renewcommand{\theenumi}{\emph{\alph{enumi}}}}%
{\setcounter{subprobcount}{\value{enumi}}\end{enumerate}}

% Blanks for answers in normal and math mode.
\newcommand{\blank}[1]{\rule{#1}{0.75pt}}
\newcommand{\mblank}[1]{\underline{\hspace{#1}}}
\def\emptybox(#1,#2){\framebox{\parbox[c][#2]{#1}{\rule{0pt}{0pt}}}}

% Misc.
\renewcommand{\d}{\displaystyle}
\newcommand{\ds}{\displaystyle}
\def\bc{\begin{center}}
\def\ec{\end{center}}


\doctitle{Math 251: Integration Extra}
\docdate{Recitation Week 13}
\doccourse{UAF Calculus I}
%\docversion{v-practice}
%\docscoring{\blank{0.8in} / 12}

\begin{document}
\begin{enumerate}
\item (Net Change Extra) An airplane is descending.  Its rate of change of height
is $\d r(t) = -4 t + \frac{t^2}{10}$ meters per second.  
\begin{enumerate}
\item If $A(t)$ is the altitude of the airplane in meters, 
how are $A(t)$ and $r(t)$ related?
\vfill
\item What physical quantity
does $\d \int_1^3 r(t)\; dt$ represent?
\vfill
\item  Compute $A(3)-A(1)$.
\vfill
\item Explain why you do not know $A(t)$ exactly.
\vfill
\item Explain how you can find $A(3)-A(1)$ exactly without knowing $A(t)$ exactly?
\vfill
\end{enumerate}

\item Fill out the blanks below:\\
 \begin{multicols}{2}
    \begin{itemize}
    \item $\d \int x^n dx = $
    \item $\d \int \sin x dx =$
    \item $\d \int \cos x dx = $
    \item $\d \int \sec^2 x dx =$
    \item $\d \int \csc^2 x dx =$
   
\columnbreak
%\vspace*{0.32in}
 \item $\d \int \sec x \tan x dx= $
    \item $\d \int \csc x \cot x dx= $
    \item $\d \int \frac 1 x dx$
    \item $\d \int e^x dx$
    \item $\d \int \frac{1}{\sqrt{1-x^2}} dx$
    \item $\d \int \frac{1}{1+x^2} dx$
    \end{itemize}
  \end{multicols}
  \newpage
\item For the integral $\ds \int \sin(x) \cos(x) \: dx$, evaluate it first using $u=\sin (x)$ then using $u=\cos(x).$ Are these really equal? Justify your answer.
\vfill
\item Evaluate the integrals below. 
\begin{multicols}{3}
	\begin{enumerate}
	\item $\ds{\int \frac{1}{x^2+1} \: dx}$
	\item $\ds{\int \frac{x}{x^2+1} \: dx}$
	\item $\ds{\int \frac{x^2+1}{x} \: dx}$
	\end{enumerate}
\end{multicols}
\vfill

\begin{multicols}{3}
	\begin{enumerate}
	\item[(d)] $\ds{\int \frac{\cos( \sqrt{x})}{\sqrt{x}} \: dx}$
	\item[(e)] $\ds{\int \frac{x^3}{\sqrt{x^2+1}} \: dx}$
	\item[(f)] $\ds{\int \frac{x^2+1}{\sqrt{x}} \: dx}$
	\end{enumerate}
\end{multicols}
\vfill

\end{enumerate}
\end{document}