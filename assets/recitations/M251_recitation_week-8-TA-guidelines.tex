
\documentclass[11pt,fleqn]{article} 
\usepackage[margin=0.8in, head=0.8in]{geometry} 
\usepackage{amsmath, amssymb, amsthm}
\usepackage{fancyhdr} 
\usepackage{palatino, url, multicol}
\usepackage{graphicx, pgfplots,xfrac} 
\usepackage[all]{xy}
\usepackage{polynom} 
%\usepackage{pdfsync} %% I don't know why this messes up tabular column widths
\usepackage{enumerate}
\usepackage{framed}
\usepackage{setspace}
\usepackage{array,tikz}

\pgfplotsset{compat=1.6}

\pgfplotsset{soldot/.style={color=black,only marks,mark=*}} \pgfplotsset{holdot/.style={color=black,fill=white,only marks,mark=*}}


\pagestyle{fancy} 
\lfoot{}
\rfoot{}

\begin{document}
\renewcommand{\headrulewidth}{0pt}
\newcommand{\blank}[1]{\rule{#1}{0.75pt}}
\newcommand{\bc}{\begin{center}}
\newcommand{\ec}{\end{center}}
\renewcommand{\d}{\displaystyle}

\vspace*{-0.7in}

%%%%%%%%%intro page
\begin{center}
  \large
  \sc{Recitation: Week 8}\\
  Derivative Proficiency Practice Guidelines for TA's
\end{center}

\begin{enumerate}
\item Remind students that they will take a Derivative Proficiency on Thursday. Ask them if they have questions. The facts are listed below.
	\begin{itemize}
	\item You have 1 hour to complete it.
	\item No notes or other aids.
	\item No simplification is needed.
	\item No partial credit. 
	\item Answers will be read as written. So $(x+1)\sin(x) \not = x+1 \: \cdot \: \sin(x)$.
	\item You have a total of three opportunities to take a Derivative Proficiency. Your high score will count in your grade.
	\item Graded proficiencies will be returned in class on Friday. The retake will be on Thursday (outside of class) after Spring Break.
	\end{itemize} 
\item Ask students what sort of things they will need to know and what to expect. Here are things that should be mentioned.
	\begin{itemize}
	\item You must know how to take the derivative of sine, cosine, tangent and secant. Typically cotangent OR cosecant appear, but you can always convert to sines and cosines.
	\item You must know how to take the derivative of $e^x$ and $\ln(x)$. Alternate bases will not appear (like $2^x$ and $\log_2(x)$).
	\item There are multiple problems that require the chain rule and at least two double chain rules always appear.
	\item The product rule always appears. There is always a problem for which the quotient rule is probably easier then rewriting and doing a product rule.
	\item The last problem is always an implicit differentiation problem.
	\item There are problems for which simplification or some algebraic preprocessing is either required or will make the problem much easier. 
	\end{itemize}
\end{enumerate}
\end{document}

