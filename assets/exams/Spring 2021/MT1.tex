\documentclass[12pt]{article}
\usepackage[top=1in, bottom=1in, left=.75in, right=.75in]{geometry}
\usepackage{amsmath}
\usepackage{fancyhdr}
\usepackage{graphicx}
\usepackage{txfonts}
\usepackage{multicol}
\usepackage[scaled=0.86]{helvet}
\renewcommand{\emph}[1]{\textsf{\textbf{#1}}}
\usepackage{anyfontsize}
% \usepackage{times}
% \usepackage[lf]{MinionPro}
\usepackage{tikz,pgfplots}
\def\degC{{}^\circ{\rm C}}
\def\ra{\rightarrow}

\newcommand{\blank}[1]{\rule{#1}{0.75pt}}

% \setmainfont{Times}
% \def\sansfont{Lucida Grande Bold}
\parindent 0pt
\parskip 4pt
\pagestyle{fancy}
\fancyfoot[C]{\emph{\thepage}}
\fancyhead[L]{\ifnum \value{page} > 1\relax\emph{Math 251: Midterm 1}\fi}
\fancyhead[R]{\ifnum \value{page} > 1\relax\emph{February 10, 2021}\fi}
\headheight 15pt
\renewcommand{\headrulewidth}{0pt}
\renewcommand{\footrulewidth}{0pt}
\let\ds\displaystyle
\def\continued{{\emph {Continued....}}}
\def\continuing{{\emph {Problem \arabic{probcount} continued....}}\par\vskip 4pt}


\newcounter{probcount}
\newcounter{subprobcount}
\newcommand{\thesubproblem}{\emph{\alph{subprobcount}.}}
\def\problem#1{\setcounter{subprobcount}{0}%
\addtocounter{probcount}{1}{\emph{\arabic{probcount}.\hskip 1em(#1)}}\par}
\def\subproblem#1{\par\hangindent=1em\hangafter=0{%
\addtocounter{subprobcount}{1}\thesubproblem\emph{#1}\hskip 1em}}
\def\probskip{\vskip 10pt}
\def\medprobskip{\vskip 2in}
\def\subprobskip{\vskip 45pt}
\def\bigprobskip{\vskip 4in}

\begin{document}
{\emph{\fontsize{26}{28}\selectfont Math F251\hfill
{\fontsize{32}{36}\selectfont Midterm 1}
\hfill Spring 2021}}
\vskip 2cm
\strut\vtop{\halign{\emph#\hskip 0.5em\hfil&#\hbox to 2in{\hrulefill}\cr
\emph{\fontsize{18}{22}\selectfont Name:}&\cr
\noalign{\vskip 10pt}
\emph{\fontsize{18}{22}\selectfont Student Id:}&\cr
\noalign{\vskip 10pt}
\emph{\fontsize{18}{22}\selectfont Calculator Model:}&\cr
}}
\hfill
\vtop{\halign{\emph{\fontsize{18}{22}\selectfont #}\hfil& \emph{\fontsize{18}{22}\selectfont\hskip 0.5ex $\square$ #}\hfil\cr
Section: & FXA (Sus)\cr
\noalign{\vskip 4pt}
         & FXB (Maxwell)\cr
\noalign{\vskip 4pt}
         & UX1 (Jurkowski)\cr}}

\vskip 1.5cm
{\fontsize{18}{22}\selectfont\emph{Rules:}}

You have 60 minutes to complete the exam.  

Partial credit will be awarded, but you must show your work.

A scientific or graphing calculator (without symbolic manipulation) is allowed.

A one page sheet of paper (8 1/2 in. x 11 in.) with handwritten notes on one side is allowed.

No other aids are permitted.

Place a box around your  \fbox{FINAL ANSWER} to each question where appropriate.

If you need extra space, you can use the back sides of the pages.
Please make it obvious  when you have done so.

Turn off anything that might go beep during the exam.

Good luck!
\vskip 1cm
\def\emptybox{\hbox to 2em{\vrule height 16pt depth 8pt width 0pt\hfil}}
\def\tline{\noalign{\hrule}}
\centerline{\vbox{\offinterlineskip
{
\bf\sf\fontsize{18pt}{22pt}\selectfont
\hrule
\halign{
\vrule#&\strut\quad\hfil#\hfil\quad&\vrule#&\quad\hfil#\hfil\quad
&\vrule#&\quad\hfil#\hfil\quad&\vrule#\cr
height 3pt&\omit&&\omit&&\omit&\cr
&Problem&&Possible&&Score&\cr\tline
height 3pt&\omit&&\omit&&\omit&\cr
&1&&9&&\emptybox&\cr\tline
&2&&10&&\emptybox&\cr\tline
&3&&16&&\emptybox&\cr\tline
&4&&10&&\emptybox&\cr\tline
&5&&10&&\emptybox&\cr\tline
&6&&8&&\emptybox&\cr\tline
&Extra Credit&&3&&\emptybox&\cr\tline
&Total&&63&&\emptybox&\cr
}\hrule}}}

\newpage
\problem{9 points}

In the diagram below, sketch the graph
of a function  $f(x)$ defined on all the real numbers
satisfying the following criteria.
 
% \begin{multicols}{3}
    \begin{enumerate}
        \item $\ds \lim_{x\to 2}f(x)=1$
        \item $f(x)$ is not continuous at $x=1$
        \item $\ds \lim_{x\to 4-}f(x)=\infty$
        \item $\ds f(4)=-1$
        \item $\ds \lim_{x\to \infty}f(x)=0$
        \item $\ds \lim_{x\to 0^+}=-1$
        \item $\ds \lim_{x\to 0^-}=2$
        \item $\ds f(-2)=-1$
        \item $\ds f(x)$ is a linear function on $[-2,0]$
    \end{enumerate}
% \end{multicols}
\vfill
\begin{center}
\begin{tikzpicture}[xscale=1.5,yscale=1.5]
\draw [help lines,dashed] (-3,-3) grid (6.2,3.2);
\draw [thick, ->] (-3,0)--(6.2,0) node[right] {$x$};
\draw [thick, ->] (0,-3)--(0,3.2) node[above]{$f(x)$};
\foreach \i in {-2,2,4}
{       \node[below] at (\i,0) {$\i$};
}
\foreach \i in {-2,2}
{       \node[left] at (0,\i) {$\i$};
}
\end{tikzpicture}
\end{center}
\vfill
\newpage
\problem{10 points} For a particular function $f(x)$,
\[
f(1)=4,\quad f(4)=1,\quad f'(1)=0\quad\text{and}\quad f'(4)=2.
\]
\subproblem{} Find the equation of the tangent line
of the graph of $f(x)$ at $x=4$.
\vfill
\subproblem{}
\begin{itemize}
	\item On the axes below make a rough sketch of what the graph $y=f(x)$ might look like, using all of the limited information that you have.
	\item Sketch the tangent line at $x=1$.
	\item Sketch the tangent line at $x=4$.
\end{itemize}

\hfil\begin{tikzpicture}[xscale=2.5,yscale=1.5]
\draw [help lines,dashed] (0,-2) grid (5.2,5.2);
\draw [thick, ->] (0,0)--(5.2,0) node[right] {$x$};
\draw [thick, ->] (0,-2)--(0,5.2) node[above]{$f(x)$};
\foreach \i in {2,4}
{       \node[below] at (\i,0) {$\i$};
}
\foreach \i in {2,4}
{       \node[left] at (0,\i) {$\i$};
}
\end{tikzpicture}


\newpage

\problem{16 points}  Compute, with justification, the following limits, or explain why the given limit does not exist. 
\begin{itemize}
\item \textbf{Use proper limit notation for full credit.}
\item  If a limit is \textbf{positive or negative infinity}, state that explicitly and justify your answer \textbf{without using a sketch}.
\end{itemize}
\vskip 12pt

\subproblem{}$\ds\lim_{h\to 0} \frac{\sqrt{9+h}-3}{h}$
\vfill
\continued
\newpage
\continuing
\vskip 12pt

\subproblem{} $\ds\lim_{x\to 5^+} \frac{x^2-3x}{5-x}$
\vfill
\subproblem{} $\ds\lim_{x\to\infty} \frac{\sqrt{5x^2-2}}{3-2x}$
\vfill
\subproblem{} $\ds\lim_{x\to\infty} \sin\left(\frac{\sqrt{5x^2-2}}{3-2x}\right)$
\vskip 1cm
\newpage
\problem {10 points}  An icicle is growing
at the back of a house.  At time $t\ge 0$
days the length of the icicle is
$$
\ell(t) = \frac{14 t + 4}{t+2}
$$
inches.  Answer the following questions about the
function $\ell(t)$ and be sure to include \textbf{units}
in your answers.

\subproblem{} Compute the average rate of change of
the length of the icicle from time $t=0$ to time $t=2$
days.  \textbf{Do not forget units!}
\vfill

\subproblem{}  It is a known fact that $\ds \ell'(t) = \frac{24}{(t+2)^2}$.
Compute $\ell'(2)$ (\textbf{with units}) and explain what this quantity 
means about the icicle.
\vfill

\subproblem{}
Compute $\lim_{t\to\infty} \ell(t)$.
Then explain what this quantity means in precise
but everyday language that the general public would understand. 
Do not forget units!
\vfill
\newpage
\problem{10 points}
Use the \textbf{definition of the derivative} to compute
$f'(2)$ if $f(x)=7x^2$. No credit will be given if a different method is used.

\newpage
\problem{8 points}

Match the graph of each function (a) - (d) with the
graph of its derivative I-IX. Write your answers in the blanks
provided below.\medskip

\halign{#&\qquad#\cr
1. The derivative of graph (a) is \blank{1in} & 3. The derivative of graph (b) is \blank{1in}\cr
\noalign{\vskip 8pt}
2. The derivative of graph (c) is \blank{1in} & 4. The derivative of graph (d) is \blank{1in}\cr}\par\bigskip

\hrule
\emph{Functions:}
\begin{center}
 \begin{tabular}{llcll}
a)&\includegraphics{Graphics/MT1-matching-quartic}
&\qquad&b)&\includegraphics{Graphics/MT1-matching-atan}\\
&&&&\vspace{.4in}\\
c)&\includegraphics{Graphics/MT1-matching-sqrt}&\qquad&d)&\includegraphics{Graphics/MT1-matching-cubic}
\end{tabular} 
\end{center}

\smallskip
\hrule
\emph{Derivatives:}

\begin{center}
%%Version 1
\begin{tabular}{llcllcll}
I.&\includegraphics{Graphics/MT1-matching-quartic-prime-fake}&\qquad&
II.&\includegraphics{Graphics/MT1-matching-sqrt-prime}&\qquad&
III.&\includegraphics{Graphics/MT1-matching-atan-prime-fake}\\
&&&&\vspace{.3in}\\
IV.&\includegraphics{Graphics/MT1-matching-quartic-prime}&\qquad&
V.&\includegraphics{Graphics/MT1-matching-cubic-prime-fake2}&\qquad&
VI.&\includegraphics{Graphics/MT1-matching-sqrt-prime-fake}\\
&&&&\vspace{.3in}\\
VII.&\includegraphics{Graphics/MT1-matching-cubic-prime-fake}&\qquad&
VIII.&\includegraphics{Graphics/MT1-matching-atan-prime}&\qquad&
IX.&\includegraphics{Graphics/MT1-matching-cubic-prime}
\end{tabular}\end{center}
\newpage


% \begin{axis}[scale=1, axis x line=middle, axis y line=
% middle, xlabel={$x$}, ylabel={$y$}, xtick={-4,-3,...,4}, ytick={-2,-1,...,4},
% xmin=-5, xmax=5, ymin=-3, ymax=5, minor y tick num=1,
%         minor x tick num=1, mark size=3.0pt]
% % \addplot[domain=-4.4:-3,<-,ultra thick] {0.1*((x+3)^2)+2};
% % \addplot[domain=-3:1, smooth, tension=1, ultra thick] coordinates { (-3,2) (-1,1.8) (0,1) (1,0)};
% % \addplot[domain=1:3.8, ultra thick,->] {(x-4)^(-1)+3.3};
% % \addplot[domain=4.2:5,<->, ultra thick]{(x-4)^(-1)};
% % \addplot[dashed,<->] coordinates {(4,4.5) (4,-2)};
% % \addplot[mark=*,only marks] coordinates {(-3,1)(1,0)};
% % \addplot[mark=*,fill=white,only marks] coordinates {(-3,2)(1,3)};
% \end{axis}
% \end{center}

\problem{Extra Credit: 3 points}
In problem \emph{3a} you computed the following limit:
\[
\lim_{h\to 0} \frac{\sqrt{9+h}-3}{h}.
\]
In fact, this limit is a computation, from the definition
of the derivative, of $f'(a)$ for some function $f(x)$
and some $x$-value $a$.  What is $f(x)$ and what is $a$?

\end{document}
