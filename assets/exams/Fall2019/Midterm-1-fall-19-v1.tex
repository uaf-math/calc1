% !TEX TS-program = pdflatexmk
\documentclass[11pt]{article}
\usepackage[margin=.8in]{geometry}
\usepackage{amsmath,amssymb,amsthm, latexsym, mathrsfs, pdfsync, multicol,
%setspace,
%graphics, 
fancybox, fancyhdr,
graphicx, enumerate,
subfig, tikz, pgfplots,array}

%\singlespacing
\def\RR{{\mathbb R}}
\def\NN{{\mathbb N}}
\def\ZZ{{\mathbb Z}}
\def\QQ{{\mathbb Q}}
\def\CC{{\mathbb C}}
\def\bc{\begin{center}}
\def\ec{\end{center}}
\def\be{\begin{enumerate}}
\def\ee{\end{enumerate}}
\def\bi{\begin{itemize}}
\def\ei{\end{itemize}}
\def\bs{\begin{slide}}
\def\es{\end{slide}}
\def\bx{\begin{exercise}}
\def\ex{\end{exercise}}
\def\t{\times}
\newcommand{\ol}[1]{\overline{#1}}
\newcommand{\oimp}[1]{\overset{#1}{\Longleftrightarrow}}
\newcommand{\bv}[1]{\ensuremath{ \mathbf{\vec{#1}}} }
\renewcommand{\d}{\displaystyle}
\newcommand{\blank}[1]{\rule{#1}{0.75pt}}

\usetikzlibrary{calc}

%for tikz pictures
\pgfplotsset{compat=1.6}

\pgfplotsset{soldot/.style={color=black,only marks,mark=*}} \pgfplotsset{holdot/.style={color=black,fill=white,only marks,mark=*}}


%
% Answerbox:
%
%\newcommand\answerbox[3]{#3 \fbox{\rule{#1}{0cm}\rule{0cm}{#2}}}
%
%\setlength{\headsep}{2pt}

\lhead{\sc{Math 251 Calculus I}}
\chead{\large \sc Midterm I} 
\rhead{\sc Fall 2019}
\cfoot{}
\pagestyle{fancy}
%
\begin{document}
\thispagestyle{fancy}

\vspace{.1in}
\begin{tabular}{l@{\hspace{.4in}}l}
Your Name & Your Signature\\
\framebox(200,30){} & \framebox(200,30){} \\
\end{tabular}

%\bigskip

\begin{tabular}{l@{\hspace{.4in}}l}
Instructor Name & \\
\framebox(200,30){}&  \\
\end{tabular}
{
\renewcommand{\baselinestretch}{1.8}
\setlength{\tabcolsep}{.2in}
\normalsize
\begin{center}
\begin{tabular}{|c|c|c|}
\hline
Problem&Total Points&\parbox{.8in}{\hfil Score\hfil}\\
\hline
1&10&\\
\hline
2&12&\\
\hline
3&10&\\
\hline
4&10&\\
\hline
5&10&\\
\hline
6&8&\\
\hline
7&20&\\
\hline
8&10&\\
\hline
\hline
Extra Credit & (5) & \\
%\hline
\hline
Total&100&\\
\hline
%Current Course Grade&\multicolumn{2}{c|  }{}\\
%\hline

\end{tabular}

\end{center}
}
\begin{itemize}
\item 
This test is closed notes and closed book.

\item You may \textbf{not} use a calculator.

\item
In order to receive full credit, you must {\bf show your work.}  
Be wary of doing computations in your head. Instead, write out your
computations on the exam paper.

%\item
%\textbf{PLACE A BOX AROUND \fbox{YOUR FINAL ANSWER} to each question} where appropriate. 
\item
Raise your hand if you have a question.

\end{itemize}

\newpage
\vspace*{-0.3in}
\begin{enumerate}

%%%%%Problem 1: Questions about a graph

\item (10 points) The graphs of two functions $f(x)$, shown thick, and $g(x)$, shown dashed, are given below.  %whose domain is $(-6, 7
Determine the following. If the value does not exist, write ``DNE''.
%
%\begin{figure}[ht]
\begin{center}
\begin{tikzpicture}[scale=.9]
\draw[ultra thin] (-6.9,-3) grid (6.9,3.1);
\draw[thick,<->] (-7.2,0) -- (7.2,0);
\draw[thick,<->] (0,-3.1) -- (0,3.1);
%\draw[dashed] (7,-2) -- (7,3);
%\draw[dashed] (-6,-2)--(-6,3);

%%%%% f(x) %%%%
\draw [ultra thick] (-5, -1) parabola bend (-4,2)  (-3,0) to  (-2,-1) to (1,-1);
\draw [ultra thick, ->] (1, -2)  parabola bend (2,-2) (3,-1)-- (4,1) -- (6,3);
\draw[fill=white, thick] (4,1) circle  (1.2 mm);
%\draw[fill=white] (6, 2) circle (1.2mm);
\draw[fill=black] (1,-2) circle (1.2mm);
\draw[fill=white] (1,-1) circle (1.2mm);
\draw[fill=white] (-5,-1) circle (1.2mm);
%\draw[fill=black] (6,3) circle (1.2mm);

\draw (2.9, -1.5) node[right] {$f(x)$};
\draw (3,2) node[above left] {$g(x)$};

%%%%% g(x) %%%%%
%\draw[ultra thick, dashed, <->](-6.8, -2.9)-- (-6, -2.9)  to[out = 5, in = 180+45](-3,-2) to[bend left = 30] (-1,0) to [out = 45, in = -90] (1.9,2.5)--(1.9,3);

\draw[ultra thick, dashed, <->] plot [smooth] coordinates { (-6.8, -2.9) (-5, -2.8) (-4, -2.6) (-3,-2) (0, .2) %(.5,0) 
(2.4,1) 
(2.95,3)};

\foreach \i in {-6, -5, ..., 6}
{\draw (\i,0) node[below] {\i};
%\draw (\i+.5, -.1) -- (\i+.5, .1);
}
\foreach \i in { -3, -2, -1,1,2, 3}
{\draw (0,\i) node[left] {\i};}
%\foreach \i in { -2, -1,0,1,2}{
%\draw (-.1, \i+.5) -- (.1,\i+.5);}
\end{tikzpicture}

\end{center}
%\end{figure}

\newcommand{\ans}{\rule{1.5cm}{.5 pt}}

\newlength{\mysep} 
\setlength{\mysep}{0.3in}
\begin{enumerate}
\begin{multicols}{2}

\item $\d\lim_{x\to -5^{+}} f(x) = \ans$
\bigskip

\item $ \d \lim_{x\to -3} f(x) = \ans$
\bigskip

\item $\d \lim_{x \to \infty} f(x) = \ans$
\bigskip


\item $\d \lim_{x\to 1} f(x) = \ans $
\bigskip

\item $f'(-4) = \ans $
\bigskip

\item $\d\lim_{x\to 3^{-}} g(x) = \ans$
\bigskip

\item $\d \lim_{x \to -\infty} g(x) = \ans$
\bigskip





\item $\d \lim_{x\to -3} f(g(x)) = \ans $
\bigskip

\vspace{\mysep}
\end{multicols}

\vspace{1cm}

\item What is the domain of $f(x)$? Give your answer in interval form.\\

\vfill

\item Where in the domain of $f(x)$ is the function continuous? Give your answer in interval form.
\vfill

\vspace{\mysep}
\end{enumerate}
\newpage

%%%%Question 2: Derivative using definition

\item (12 points) Consider the function

$$
g(x) = 2x^2 - 3.
$$

Using the \textbf{definition of the derivative}, find $g'(a)$. No credit will be given if a different method is used. [It is recommended you start by writing the definition of the derivative.] 

\newpage
%%Continuity problem
\item (10 points) Let $f$ be the piecewise defined function below.
$$f(x)=\begin{cases} c-\sin (x) & x \not= \pi/2 \\ 5 & x=\pi/2 \\
\end{cases}$$
\begin{enumerate}
	\item Determine a value for $c$ such that the function $f(x)$ is continuous at $x=\pi/2.$
\vfill

	\item Show that your choice for $c$ above is correct using the \emph{definition of continuity at a point}. (A correct answer will involve writing and computing an appropriate limit or limits.)
\vfill
\vfill
\vfill

\end{enumerate}

%%%Tangent line problem
\item (10 points) The function $f(x)=2+4x+e^{3x}$ has derivative $f'(x)=4+3e^{3x}.$ 
\begin{enumerate}
\item Find $f(0).$ 

\vfill
\hfill $f(0) = $ \ans %\framebox(100,30){}

\bigskip

\item Find $f'(0).$ 
\vfill


\hfill $f'(0) = $ \ans %\framebox(100,30){}

\bigskip

\item Write an equation of the line tangent to the graph of $f(x)$ when $x=0$ in slope-intercept form. (That is, in the form $y=mx+b$.) 
\vfill

\hfill Equation of Tangent Line: \ans\ans\ans%\framebox(150,30){}

\bigskip

\end{enumerate}
\newpage

%\item (5 points) Let $g(x)=\frac{1}{x+3}+2.$
%\begin{enumerate}
%
%\item List all vertical asymptotes of $g(x)$: \hrulefill
%
%\item What equation(s) involving limits show(s) that the vertical asymptote(s) you found actually exist?
%
%\vfill
%
%\item List all horizontal asymptotes of $g(x)$: \hrulefill
%
%\item What equation(s) involving limits show(s) that the horizontal asymptote(s) you found actually exist?
%
%\vfill
%
%\end{enumerate}
%


%\item (5 points) Graph the function $q(x) = \cos(\pi x) + 2$ on the interval $[-1, 3]$, on the given axes.
%
%\begin{tikzpicture}[xscale=3, yscale=.9]
%\draw[ultra thick , <->] (-1.1,0) -- (3.1,0);
%\draw[ultra thick , <->] (0,-5.1) -- (0,5.1);
%\draw[ultra thin] (-1, -5) grid[step=.5] (3, 5);
%\draw[thick] (-1, -5) grid (3, 5);
%\foreach \i in {-1,1,2,3}{\draw (\i-.07,0) node[below]{$\i$};}
%\foreach \i in {-5,-4,-3,-2,-1,1,2,3,4}{\draw[ ] (-.1,\i) node[inner sep=1 pt, fill=white, left]{$\i$};}
%%\foreach \i in {-1,0,1,2,3}{\draw[ ] (\i-.5, .1) -- (\i-.5, -.1);}
%\coordinate (0, -.1) node [below left] {0};
%
%
%\end{tikzpicture}

%graphs and asymptotes and limits problem
\item (10 points) Let $g(x)=\frac{1}{x+3}+2.$
	\begin{enumerate}
	

		\item Evaluate the limit: $\d{\lim_{x \to -3^-} f(x)}.$  Show your work or justify your computation with a few words.	
	\vfill
	
	\hfill $\d{\lim_{x \to -3^-} f(x)} = \ans$
	
	\bigskip
	
	\item What does your answer to part (a) tell you about the graph of $f(x)?$ (What feature does the graph have?)
	\vspace{0.5in}
	\item Evaluate the limit: $\d{\lim_{x \to - \infty} f(x)}.$  Show your work or justify your computation with a few words.
	
		\vfill
		
		\hfill$\d{\lim_{x \to - \infty} f(x)} = \ans$
		
		\bigskip
	\item What does your answer to part (c) tell you about the graph of $f(x)?$ (What feature does the graph have?)
	\vspace{0.5in}
	
	\item Sketch the graph of $f(x)$ on the axes below. If the graph has any asymptotes, draw them as dashed lines and label them with their equation. \\
	
\begin{tikzpicture}[xscale=1.25]
\draw[ , <->] (-5.5,0) -- (5.5,0);
\draw[ , <->] (0,4.5) -- (0,-4.5);
%\foreach \i in {-5,-4,-3,-2,-1,1,2,3,4,5}{\draw[ ] (\i, .1) -- (\i, -.1) node[below]{$\i$};}
%\foreach \i in {-4,-3,-2,-1,1,2,3,4}{\draw[ ] (0.1,\i) -- (-0.1,\i) node[left]{$\i$};}
%\coordinate (0, -.1) node [below left] {0};
\end{tikzpicture}
\vspace{.25in}
	
	
	\end{enumerate} 
	\newpage
	


\newpage



%%EASY
%\item {\bf Easy version} Sketch the derivative of the following function, on the second set of axes.
%
%\begin{tikzpicture}
%\draw[ thick, <->] (-6.5,0) -- (6.5,0);
%\draw[ thick, <->] (0,3.5) -- (0,-3.5);\foreach \i in {-6, -5, -4,-3,-2,-1,1,2,3,4, 5}{\draw[ thick] (\i, .1) -- (\i, -.1) node[below]{$\i$};}
%\coordinate (0, -.1) node [below left] {0};
%\draw[ultra thin] (-6,-3) grid (5,3);
%
%\draw[ultra thick] (-5, 2) parabola bend (-3,-2)  (-2,-1) -- (1,-1) -- (4,2);
%
%
%\end{tikzpicture}
%
%\begin{tikzpicture}
%\draw[ , <->] (-6.5,0) -- (6.5,0);
%\draw[ , <->] (0,3.5) -- (0,-3.5);\foreach \i in {-6, -5, -4,-3,-2,-1,1,2,3,4, 5}{\draw[ ] (\i, .1) -- (\i, -.1) node[below]{$\i$};}
%\coordinate (0, -.1) node [below left] {0};
%
%\end{tikzpicture}
%sketch the derivative
\item (8 points) Sketch the derivative of the following function, on the second set of axes.
\begin{center}
\begin{tikzpicture}
\draw[ thick, <->] (-6.5,0) -- (6.5,0);
\draw[ thick, <->] (0,3.5) -- (0,-3.5);\foreach \i in {-6, -5, -4,-3,-2,-1,1,2,3,4, 5}{\draw[ thick] (\i, .1) -- (\i, -.1) node[below]{$\i$};}
\coordinate (0, -.1) node [below left] {0};
\draw[ultra thin] (-6,-3) grid (5,3);

%\draw plot[smooth] coordinates{ (-5,-2) (-4,1) (-3,-1) (-2,-2) (-1,1)}
\draw[ultra thick] (-6, -1) parabola bend (-4.5,1) (-3,-1) parabola bend (-2,-2)  
(-1,-1) -- (1,-1) -- (4,2);


\end{tikzpicture}
\end{center}

\begin{center}
\begin{tikzpicture}
\draw[ , <->] (-6.5,0) -- (6.5,0);
\draw[ , <->] (0,3.5) -- (0,-3.5);\foreach \i in {-6, -5, -4,-3,-2,-1,1,2,3,4, 5}{\draw[ ] (\i, .1) -- (\i, -.1) node[below]{$\i$};}
\coordinate (0, -.1) node [below left] {0};

\end{tikzpicture}
\end{center}

\newpage
 %%Limit problems
\item (20 points) Find the limit or show it doesn't exist. Use proper limit notation for full credit. If a limit is positive or negative infinity, state that explicitly instead of writing ``does not exist." Answers with little work will be accepted here but the wrong answer with no work will receive no partial credit.

\begin{enumerate}
	\item $\d \lim_{t\to-2^+} \frac{t - 2}{2 + t}$\hfill answer: \ans\ans%\framebox(100,30){}
	\vfill
%	\item $\d \lim_{\theta\to\pi}\frac{\theta^2}{\cos\theta}$
%	\vfill
	%2.5 # 36
%	\item $\d \lim_{x\to\infty}\frac{2x^2 + x}{\sqrt{16 + x^4}}$
%	\vfill
	\item $\d \lim_{x\to -2}\frac{x^2+3x+2}{x^2-x-6}$\hfill answer: \ans\ans %\framebox(100,30){}
	\vfill
	\item $\d \lim_{x\to4}\frac{\frac{1}{x^2} - \frac{1}{16}}{x - 4}$\hfill answer: \ans\ans %\framebox(100,30){}
	\vfill
	%2.7 #34
	\item $\d \lim_{x\to\infty}\left[\ln{(2x)} - \ln(5x + 2)\right]$\hfill answer: \ans\ans %\framebox(100,30){}
	\vfill
	%2.6 #42
\end{enumerate}

\newpage
%interpretataion of the derivative
\item (10 points) The temperature $F$, in degrees Fahrenheit, was measured $t$ seconds after a hot water faucet was turned on.
	\begin{enumerate}
	\item It is found that $F(2)=51.$ Interpret this fact in the context of the problem. (That is, what does $F(2)=51$ mean?)  Include units in your answer.
	\vspace{1in}	
	\item It is found that $F'(2)=5.$ Interpret this fact in the context of the problem. (That is, what does $F'(2)=5$ mean?) Include units in your answer.
	\vspace{1in}
	\item It is found that $\displaystyle{\lim_{t \to \infty} F(t)=105}.$ Interpret this fact in the context of the problem. (That is, what does $\displaystyle{\lim_{t \to \infty} F(t)=105}$ mean?)  Include units in your answer.
	\vspace{1in}
	\end{enumerate}


\end{enumerate}


\textbf{Extra Credit (5 points)} \textbf{Use the Intermediate Value Theorem} to prove that there exists some $x$-value in the interval $(0,100)$ such that the function $g(x)=18e^{-x}-\sqrt{x}+8$ takes the value $y=20.$
\vfill

\end{document}

%%%%%%%%%%
%%%%%%%%%%
End Document


