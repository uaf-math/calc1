\documentclass[12pt]{article}
\usepackage[top=1in, bottom=1in, left=.75in, right=.75in]{geometry}

\usepackage{amsmath}
\usepackage{fancyhdr}
\usepackage{graphicx}
\usepackage{txfonts}
\usepackage{multicol}
\usepackage{wrapfig}
\usepackage[scaled=0.86]{helvet}
\renewcommand{\emph}[1]{\textsf{\textbf{#1}}}
\usepackage{anyfontsize}
% \usepackage{times}
% \usepackage[lf]{MinionPro}

\usepackage{tikz,pgfplots}
\usetikzlibrary{shapes}
%\pgfplotsset{my style/.append style={axis x line=middle, axis y line=middle, xlabel={$x$}, ylabel={$y$}, axis equal }}
\pgfplotsset{my style/.append style={axis x line=middle, axis y line=middle, xlabel={$x$}, ylabel={$y$}}}

\def\degC{{}^\circ{\rm C}}
\def\ra{\rightarrow}

\newcommand*\circled[1]{\tikz[baseline=(char.base)]{
            \node[shape=ellipse,draw,inner sep=2pt] (char) {#1};}}

\newcommand{\blank}[1]{\rule{#1}{0.75pt}}

\parindent 0pt
\parskip 4pt
\pagestyle{fancy}
\fancyfoot[C]{\emph{\thepage}}
\fancyhead[L]{\ifnum \value{page} > 1\relax\emph{Math 251: Midterm 2}\fi}
\fancyhead[R]{\ifnum \value{page} > 1\relax\emph{31 March 2021}\fi}
\headheight 15pt
\renewcommand{\headrulewidth}{0pt}
\renewcommand{\footrulewidth}{0pt}
\let\ds\displaystyle
\def\continued{{\emph {Continued....}}}
\def\continuing{{\emph {Problem \arabic{probcount} continued....}}\par\vskip 4pt}

\newcounter{probcount}
\newcounter{subprobcount}
\newcommand{\thesubproblem}{\emph{\alph{subprobcount}.}}

\def\problem#1{\setcounter{subprobcount}{0}%
\addtocounter{probcount}{1}{\emph{\arabic{probcount}.\hskip 1em(#1)}}\par}

\def\ecproblem#1{{\emph{Extra Credit. \hskip 1em(#1)}}\par}

\def\subproblem#1{\par\hangindent=1em\hangafter=0{%
\addtocounter{subprobcount}{1}\thesubproblem\emph{#1}\hskip 1em}}

\def\probskip{\vskip 10pt}
\def\medprobskip{\vskip 2in}
\def\subprobskip{\vskip 45pt}
\def\bigprobskip{\vskip 4in}


\begin{document}
{\emph{\fontsize{26}{28}\selectfont Math F251\hfill
{\fontsize{32}{36}\selectfont Midterm 2}
\hfill Spring 2021}}
\vskip 1cm
\strut\vtop{\halign{\emph#\hskip 0.5em\hfil&#\hbox to 2in{\hrulefill}\cr
\emph{\fontsize{18}{22}\selectfont Name:}&\cr}}
\hfill
\vtop{\halign{\emph{\fontsize{18}{22}\selectfont #}\hfil& \emph{\fontsize{18}{22}\selectfont\hskip 0.5ex $\square$ #}\hfil\cr
Section: & FXA (Sus)\cr
\noalign{\vskip 4pt}
         & FXB (Maxwell)\cr
\noalign{\vskip 4pt}
         & UX1 (Jurkowski)\cr}}

{\fontsize{18}{22}\selectfont\emph{Rules:}}

You have 60 minutes to complete the exam.  

Partial credit will be awarded, but you must show your work.

A scientific or graphing calculator (without symbolic manipulation) is allowed.

A one page sheet of paper (8 1/2 in. x 11 in.) with handwritten notes on one side is allowed.

No other aids are permitted.

Place a box around your  \fbox{FINAL ANSWER} to each question where appropriate.

Turn off anything that might go beep during the exam.
\vskip 0.5cm

{\fontsize{16}{13}\selectfont\emph{Academic Integrity Statement:}}

{\fontsize{13}{13}\selectfont\emph{All students must affirm the following statements by initialing in the blanks provided. Students using their own paper must write out the statements in full.}}

\underline{\hspace{1in}} I will not seek or accept help from anyone.  

\underline{\hspace{1in}} I will not use books, the internet or other disallowed aids.

\underline{\hspace{1in}} I understand correct answers without sufficient supporting work will be marked incorrect.


\vskip 1cm
\def\emptybox{\hbox to 2em{\vrule height 16pt depth 8pt width 0pt\hfil}}
\def\tline{\noalign{\hrule}}
\centerline{\vbox{\offinterlineskip
{
\bf\sf\fontsize{18pt}{22pt}\selectfont
\hrule
\halign{
\vrule#&\strut\quad\hfil#\hfil\quad&\vrule#&\quad\hfil#\hfil\quad
&\vrule#&\quad\hfil#\hfil\quad&\vrule#\cr
height 3pt&\omit&&\omit&&\omit&\cr
&Problem&&Possible&&Score&\cr\tline
height 3pt&\omit&&\omit&&\omit&\cr
&1&& 10 &&\emptybox&\cr\tline
&2&& 10 &&\emptybox&\cr\tline
&3&& 10 &&\emptybox&\cr\tline
&4&& 12 &&\emptybox&\cr\tline
&5&& 10 &&\emptybox&\cr\tline
&6&& 10 &&\emptybox&\cr\tline
&Extra Credit&&3&&\emptybox&\cr\tline
&Total&& 62 &&\emptybox&\cr
}\hrule}}}

\newpage


\problem{10 points}
Consider the function
$$
f(t) = \frac{t}{1+t^2}.
$$
\subproblem{} Find the critical number(s) of $f(t)$ on the interval $[0,4]$
\vskip 2.in

\subproblem{} Find the absolute maximum and absolute minimum values of $f(t)$ on the interval $[0,4]$.
\vskip 2.in



 \problem{10 points}    % like 3.10 #5

\subproblem{} Find the linearization of $g(x)=\sqrt{x}$ at $a=100$.
\vskip 2.in
\subproblem{} Use your result in part \textbf{a.} to approximate $\sqrt{100.5}$.
\vskip 1.in


\newpage

\problem{10 points}
Evaluate the following limits.
[Note: You should be careful to apply L'Hospital's rule \emph{only} when appropriate.]
\subproblem{} $\ds \lim_{x\to 0} \frac {x^2}{1-\cos(x)}$
\vskip 1.5in
\subproblem{} $\ds \lim_{x \to 0^+} x\ln(x)$
\vskip 2.in
\subproblem{} $\ds \lim_{\theta \to \pi} \frac {1-\cos(2\theta)}{\cos( \theta)}$
\vskip 2.in
\newpage

\problem{10 points}
Consider a secret function $f(x)$. We have computed for you $$ f'(x)=\frac{2(x^2-1)}{(x^2+1)^2}\: \quad \text{and} \: \quad f''(x)=\frac{4x(3-x^2)}{(x^2+1)}.$$
\begin{center} \textbf{For full credit, show your work.} \end{center}
\subproblem{} Find the intervals where $f(x)$ is increasing and decreasing.
\vfill
\subproblem{} Find the intervals where $f(x)$ is concave up and concave down.
\vfill
\subproblem{} Classify, with justification, all critical points of $f(x).$
\vfill

\newpage
\continued
\subproblem{} In fact, for this secret function $f(0)=1$ and $f(1)=0$. Sketch the graph of any function that is consistent with
this fact and the data from parts \emph{a}-\emph{c}.  Denote any \textbf{points of inflection} on your graph with a square box.
\vskip 1cm

\strut\hfil\includegraphics{axes.pdf}


\newpage
\problem{10 points} An open faced metal box with a square bottom is to be constructed satisfying with the following constraints:
\begin{enumerate}
\item The volume is 1200 cm$^2$.
\item The bottom is made of silver costing 6 dollars per square centimeter, and the sides are made of copper costing 2 dollars per square centimeter.  
\end{enumerate}
\subproblem{} If the base of the box is 10cm in width and the height
of the box is 12cm, how much do the materials cost to build the box?
\vskip 2cm

\subproblem{} What dimensions of the box minimize the material costs? Include units in your answer.
\vfill

\newpage
\problem{10 points}
A hot air balloon is rising straight up at a rate of 40 meters per minute. You are 200 meters away from the balloon's launch site.
How fast is the distance between you and the balloon increasing when the balloon is 80 meters in the air?
\vfill

\vfill
\ecproblem{3 points}    
Compute $\ds \lim_{x\to \infty} x^{1/x}$.

\vfill

\end{document}

